\documentclass[a4paper]{article}

\usepackage{amsthm, amsmath, latexsym, amssymb}
\usepackage{mathtools}
\usepackage{verbatim}
\usepackage[dvipsnames]{xcolor}

\theoremstyle{definition} \newtheorem*{definition}{Definition}
\theoremstyle{definition} \newtheorem*{definitions}{Definitions}
\theoremstyle{plain} \newtheorem{theorem}{Theorem}[section]
\theoremstyle{plain} \newtheorem{proposition}[theorem]{Proposition}
\theoremstyle{plain} \newtheorem{corollary}[theorem]{Corollary}
\theoremstyle{plain} \newtheorem{lemma}[theorem]{Lemma}
\theoremstyle{plain} \newtheorem{example}[theorem]{Example}

\newcommand{\checkCorrect}[1]{\textcolor{red}{#1}}
\newcommand{\understandBetter}[1]{\textcolor{red}{#1}}
\newcommand{\question}[1]{\textcolor{red}{#1}}
\newcommand{\explainFurther}[1]{\textcolor{red}{#1}}
\newcommand{\finish}[1]{\textcolor{red}{#1}}
\newcommand{\defn}[1]{\textbf{#1}}
\newcommand{\realnos}{\mathbb{R}}
\newcommand{\complexnos}{\mathbb{C}}


\begin{document}
\title{Cobordism}
\author{Vatsal Kanoria}
\date{August 2022}
\maketitle
\begin{abstract}
An exposition on the proof that cobordant manifolds have the same characteristic classes.
\end{abstract}
\tableofcontents

\section{Introduction}


\section{Manifolds}

We begin by recalling some important facts from differential geometry. Point-set topology and familiarity with these notions are assumed.

A (smooth) $n$-dimensional \defn{manifold} $M$ is a second countable, Hausdorff space with a smooth maximal atlas. An \defn{atlas} consists of a collection of charts, that is, an open cover $U_\alpha$ ($\alpha\in I$, $I$ \checkCorrect{finite} index set) of $M$ and homeomorphims $\phi_\alpha:U_\alpha \to \phi_\alpha(U_\alpha) \subseteq k^n$, such that the \defn{transition maps} $\phi_\alpha \circ \phi_\beta^{-1}$ are smooth (on $\phi_\beta(U_\alpha \cap U_\beta)\subseteq k^n$) for all $\alpha, \beta\in I$. Generally we set $k=\realnos$ and assume that our manifolds are smooth and \checkCorrect{without boundary}, unless otherwise specified. 

\understandBetter{Note that every atlas uniquely gives rise to a maximal atlas, so it will generally be sufficient to define any convenient atlas on a given manifold. Recall also the non-trivial fact that the dimension of a manifold is a \explainFurther{topological invariant}. It is also easy to show that the coordinate maps $\phi_\alpha$ are not only homeomorphisms but also diffeomorphisms.} \question{Do these statements also hold for complex manifolds/manifolds with boundary?}

\understandBetter{In certain cases, we will require our manifold to be a \defn{complex manifold} in which case $k=\complexnos$, and transition maps will be required to be \understandBetter{holomorphic}.}
\explainFurther{Say more on this?}

A manifold has a \defn{boundary} if some of its charts are \defn{boundary charts}. That is, a boundary chart $\phi:U\to \phi(U)\subseteq \mathbb{H}^n$ takes values in the closed upper half space $\mathbb{H}^n := \{(x_1, \ldots, x_n): x_i\geq 0, \forall i\}$ (so that $\partial \mathbb{H}^n \cap \phi(U)$ is non-empty, where $\partial \mathbb{H}^n := \{(x_1, \ldots, x_n): x_n=0\}$). Note that $\mathbb{H}^0 = \{0\}$ and $\partial \mathbb{H}^0 = \{\}$ by definition. The remaining charts $\phi:U\to \phi(U)\subseteq \realnos^n$ are called \defn{interior charts}. The \defn{interior} of a manifold $M$ are the points that come from an interior chart, $\texttt{Int} M := \{p\in M: \exists \textrm{interior chart } \phi:U\to \phi(U)\subseteq \realnos^n, p\in U \}$. The \defn{boundary} of $M$ are the points that come from a boundary chart, $\partial M:=\{p\in M: \exists \textrm{boundary chart } \phi:U\to \phi(U)\subseteq \mathbb{H}^n, p\in U \textrm{ and } \phi(p)\in \partial \mathbb{H}^n \}$. \checkCorrect{Confusingly, the boundary of a manifold is not the same as its topological boundary in general}. It is non-trivial to show that $M$ can be decomposed into a disjoint union of its interior and boundary, $M = \texttt{Int} \ M \mathbin{\dot{\cup}} \partial M$, hence every point of $M$ is either an interior point or exclusively, a boundary point.

\subsection{Examples}

Manifolds with boundaries: 

\section{Fibre Bundles}

A \defn{fibre bundle} consists of three topological spaces $E, B, F$, and a continuous \defn{projection} map $\pi:E\rightarrow B$, such that \defn{`local trivialisations`} exist (i.e. for every $p\in B$, there exists a neighbourhood $U_p\subseteq B$ of $p$ and a homeomorphism $\psi: E_p \rightarrow U_p \times F$, such that $\pi \vert_{E_p} = \pi_1 \circ \psi$ where $\pi_1$ is the projection on to the first component, and $E_p:=\pi^{-1}(p)$ is called the \defn{fibre} over $p$. $E$ is called the \defn{total space}, $B$ the \defn{base space}, and $F$ the \defn{typical fibre}. 

\finish{PUT IN A COMMUTATIVE DIAGRAM/PICTURE FOR LOCAL TRIVS? -- DO THIS FOR VECTOR BUNDLE CASE. SHOULD I MAKE OBSERVATION ABOUT LOCAL TRIVS AT A POINT, OR ONLY FOR VECTOR BUNDLES?}

A continuous map $U:E\rightarrow E'$ is a \defn{(fibre) bundle morphism} between two fibre bundles $E\xrightarrow{\pi} B$ and $E'\xrightarrow{\pi'} B'$ \checkCorrect{with the same typical fibre $F$}, if there exists a map $u:B\rightarrow B'$ such that $\pi' \circ U = u \circ \pi$

\finish{PUT IN COMMUTATIVE DIAGRAM.}

\checkCorrect{What is difference between fibre bundle morphism, vector bunlde morphism - (such that the restricted map $u\vert_{E_p}: E_p \rightarrow E_{f(p)}$ is linear for each $p\in B$??. Do I need to distinguish these??}

\finish{Define Bundle iso. smooth bundle iso. etc. (see Lee.)}

We will work with two types of fibre bundles, namely, vector bundles and principle $G$-bundles. Roughly speaking, in the case of vector bundles, each fibre is isomorphic to $\realnos^k$, whereas for $G$-bundles, the fibres are isomorphic to the Lie group $G$. The notion of a bundle morphism also generalises to both these structures, with some extra conditions to preserve the additional structure. Note that we will use the same terminology `bundle morphism` for morphisms betweeen vector bundles and morphisms between fibre bundles and this will have to be understood from the context. In the case of principle G-bundles, we will always specify `principle G-bundle morphism`.

\section{Vector Bundles}


\subsection{De Rham Cohomology}

\subsection{Integration??}

\subsection{Examples (of vect bundles)}

\section{Principle G-Bundles}

\subsection{Examples}

\section{Connections on Vector Bundles}

\subsection{Examples}

\section{Connections on Principle G-Bundles}

\section{Equivalence of Connections}

\section{Curvature}

\subsection{Flat connections and more general De Rham Cohomology}

\subsection{Examples}

\section{Torsion??}

\section{Characteristic classes}

\section{Bordism and Cobordism}

\section{Chern Simons}

\section{Topological Quantum Computing}

\section{Appendix: Functors and Categories}

\section*{References}
\begin{enumerate}
\item J.M.Lee; \textit{Introduction to Smooth Manifolds, 2nd ed.} Springer (2012)
\item I.Madsen, J.Tornehave; \textit{From Calculus to Cohomology, De Rham Cohomology and characteristic classes.} Cambridge University Press (1997)
\end{enumerate}
\end{document}
