\documentclass[a4paper]{article}

\usepackage{amsthm, amsmath, latexsym, amssymb}
\usepackage{mathtools}
\usepackage{bbm}
\usepackage{verbatim}
\usepackage[dvipsnames]{xcolor}

\theoremstyle{definition} \newtheorem*{definition}{Definition}
\theoremstyle{definition} \newtheorem*{definitions}{Definitions}
\theoremstyle{plain} \newtheorem{theorem}{Theorem}[section]
\theoremstyle{plain} \newtheorem{proposition}[theorem]{Proposition}
\theoremstyle{plain} \newtheorem{corollary}[theorem]{Corollary}
\theoremstyle{plain} \newtheorem{lemma}[theorem]{Lemma}
\theoremstyle{plain} \newtheorem{example}[theorem]{Example}

\newcommand{\checkCorrect}[1]{\textcolor{red}{#1}}
\newcommand{\understandBetter}[1]{\textcolor{orange}{#1}}
\newcommand{\question}[1]{\textcolor{orange}{#1}}
\newcommand{\explainFurther}[1]{\textcolor{blue}{#1}}
\newcommand{\finish}[1]{\textcolor{green}{#1}}

\newcommand{\defn}[1]{\textbf{#1}}
\newcommand{\realnos}{\mathbb{R}}
\newcommand{\complexnos}{\mathbb{C}}
\newcommand{\canonicaliso}{\cong}
\newcommand{\id}{\mathtt{id}}
\newcommand{\smoothCmaps}{C^\infty_\complexnos (U)}
\newcommand{\Hom}{\text{Hom}}
\newcommand{\End}{\text{End}}
\newcommand{\tr}{\text{tr}}
\newcommand{\smooth}{C^\infty}


\begin{document}
\title{Chern Classes}
\author{Vatsal Kanoria}
\date{2023}
\maketitle

\begin{abstract}
A literature review on the theory of Chern classes from the perspective of differential geometry. We study in particular the example of complex projective space in detail. 
\end{abstract}

\tableofcontents

\section{Introduction}

We are interested in learning about topological invariants of vector bundles on manifolds. Namely those properties that are the `same' for isomorphic vector bundles. A particular type of global invariant on vector bundles is known as Characteristic classes. Loosely speaking, Characteristic classes \understandBetter{measure the difference between a global product structure and a local product structure on a manifold.} \finish{[OBSTRUCTION to defining independent sections on the bundle?? -- or is this just Stiefal-whitney?]}

There are four different types of Characteristic classes: Stiefal-Whitney classes, Pontryagin classes, \checkCorrect{Euler classes} and Chern classes. Stiefal-Whitney classes are defined as characteristic classes of real vector bundles. \understandBetter{Pontryagin classes are characteristic classes of .... Euler classes.....}
Chern classes characterise complex vector bundles. 
Note that one can consider Chern classes of complex vector bundles over smooth (real) manifolds, and also of K{\"a}hler manifolds (the latter of which we do not discuss in detail). Note further that the theory of Chern classes can be applied equally if we replace vector bundles with principle $G$-bundles, since these structures are equivalent. 

The theory of Characteristic classes is a cohomology theory, and thus requires techniques from algebraic topology to compute these in general. However in the case of Chern classes of complex vector bundles, one can use techniques from complex differential geometry to compute these. This makes computing Chern classes comparatively more explicit than computing the other characteristic classes. In particular the Chern classes (which are elements in cohomology classes) can be defined in terms of an invariant polynomial of its curvature form. Recall that curvature is associated to a connection on a vector bundle. We will find that in fact the choice of connection does not impact the Chern class, \checkCorrect{so it is indeed an invariant.}

Hence we spend a large portion of the thesis  reproducing important facts and examples from the theory of manifolds, vector bundles, connections on manifolds and the associated curvature. We then go on to define Chern classes and their properties. We compute all of the theory with a focus on complex projective space, whilst alluding to various other simple examples throughout the exposition. We also do not hesitate to recall theory relating to smooth and Riemannian manifolds, due to their familiarity and similarity in many respects to the respective theory on complex manifolds.

Complex projective space happens to be a particularly important example in the theory of Chern classes, since one can show.... \finish{[SEE HUYBRECHTS]}.

The theory of Characteristic classes have applications for example in Topological Quantum computing. Chern classes are an example of `primary' characteristic class. One may also define `secondary' characteristic classes \finish{[HOW ROUGHLY?]} leading to what is known as Chern-Simons theory. Chern-Simons theory is a topological quantum field theory. \checkCorrect{In the simplest case, $SU(2)$-Chern-Simons theory (which is known to physicists as the Yang-Lee-Fibonnaci model) gives rise to non-abelian anyons describing a model for Topological Quantum computing.} A formal treatment of Chern-Simons theory is out of the scope of this exposition, \checkCorrect{however it is defined in terms of the Chern-Weil homomorphism which we do discuss.} 
\finish{Calabi-yau manifolds?}
\finish{Say something about topological quantum computing? Why its interesting?}

\section{Manifolds}

\subsubsection{Smooth Manifolds}

A (smooth) $n$-dimensional \defn{manifold} $M$ is a second countable, Hausdorff space with a smooth maximal atlas. An \defn{atlas} consists of a collection of charts, that is, an open cover $U_\alpha$ ($\alpha\in I$, $I$ \checkCorrect{finite} index set) of $M$ and homeomorphims $\phi_\alpha:U_\alpha \to \phi_\alpha(U_\alpha) \subseteq k^n$, such that the \defn{transition maps} $\phi_\alpha \circ \phi_\beta^{-1}$ are smooth (on $\phi_\beta(U_\alpha \cap U_\beta)\subseteq k^n$) for all $\alpha, \beta\in I$. Generally we set $k=\realnos$ and assume that our manifolds are smooth and \checkCorrect{without boundary}, unless otherwise specified. 

\understandBetter{Note that every atlas uniquely gives rise to a maximal atlas, so it will generally be sufficient to define any convenient atlas on a given manifold.} Recall also the non-trivial fact that the dimension of a manifold is a topological invariant (in other words given two charts $\phi$, $\tilde{\phi}$ of $M$ with $\mathtt{im}(\phi)\subseteq \realnos^m$, $\mathtt{im}(\tilde{\phi})\subseteq \realnos^n$, we have $m=n$). \question{Is this true? In milnor seems to allow locally constant function as dimension, and terms this kind of vector bundle as a special case, i.e. an $R^n$-bundle!?} It is also easy to show that the coordinate maps $\phi_\alpha$ are not only homeomorphisms but also diffeomorphisms. \question{Do these statements also hold for complex manifolds/manifolds with boundary etc.?}

\understandBetter{In certain cases, we will require our manifold to be a \defn{complex manifold} in which case $k=\complexnos$, and transition maps will be required to be \understandBetter{holomorphic}.}
\explainFurther{Say more on this?}

A manifold has a \defn{boundary} if some of its charts are \defn{boundary charts}. That is, a boundary chart $\phi:U\to \phi(U)\subseteq \mathbb{H}^n$ takes values in the closed upper half space $\mathbb{H}^n := \{(x_1, \ldots, x_n): x_i\geq 0, \forall i\}$ (so that $\partial \mathbb{H}^n \cap \phi(U)$ is non-empty, where $\partial \mathbb{H}^n := \{(x_1, \ldots, x_n): x_n=0\}$). Note that $\mathbb{H}^0 = \{0\}$ and $\partial \mathbb{H}^0 = \{\}$ by definition. The remaining charts $\phi:U\to \phi(U)\subseteq \realnos^n$ are called \defn{interior charts}. The \defn{interior} of a manifold $M$ are the points that come from an interior chart, $\texttt{Int} M := \{p\in M: \exists \textrm{interior chart } \phi:U\to \phi(U)\subseteq \realnos^n, p\in U \}$. The \defn{boundary} of $M$ are the points that come from a boundary chart, $\partial M:=\{p\in M: \exists \textrm{boundary chart } \phi:U\to \phi(U)\subseteq \mathbb{H}^n, p\in U \textrm{ and } \phi(p)\in \partial \mathbb{H}^n \}$. \checkCorrect{Confusingly, the boundary of a manifold is not the same as its topological boundary in general}. It is non-trivial to show that $M$ can be decomposed into a disjoint union of its interior and boundary, $M = \mathtt{Int} \ M \mathbin{\dot{\cup}} \partial M$, hence every point of $M$ is either an interior point or exclusively, a boundary point.

\subsubsection{Examples: new manifolds from old.}

Subsets of $R^n$?? Quotient spaces.
topology on $T^2$, $S^n$...smooth manifolds.

\subsubsection{Example: An atlas on $S^1$}

On the circle $S^1:=\{(x, y): x^2+y^2 = 1\}\subseteq \realnos^2$, we can define an atlas \question{What is this kind of projection called?} $\mathfrak{A}_1 :=\cup_{i=1}^4 U_i$ given by 
\begin{align*}
&& U_1 = S^1 \cap \{(x, y): x>0 \} \\
\phi_1 & : U_1\to \realnos &
\phi_1^{-1} & :(-1, 1)\to U_1 \\
& (x,y) \xmapsto{\phi_1} x 
& & x \xmapsto{\phi_1^{-1}} (x, \sqrt{1-x^2})
\end{align*}
\begin{align*}
&& U_2 = S^1 \cap \{(x, y): x<0 \} \\
\phi_2 & : U_2\to \realnos &
\phi_2^{-1} & :(-1, 1)\to U_2 \\
& (x,y) \xmapsto{\phi_2} x 
& & x \xmapsto{\phi_2^{-1}} (x, - \sqrt{1-x^2})
\end{align*}
\begin{align*}
&& U_3 = S^1 \cap \{(x, y): y>0 \} \\
\phi_3 & : U_3\to \realnos &
\phi_3^{-1} & :(-1, 1)\to U_3 \\
& (x,y) \xmapsto{\phi_3} y
& & y \xmapsto{\phi_3^{-1}} (\sqrt{1-y^2}, y)
\end{align*}
\begin{align*}
&& U_4 = S^1 \cap \{(x, y): y<0 \} \\
\phi_4 & : U_4\to \realnos &
\phi_4^{-1} & :(-1, 1)\to U_4 \\
& (x,y) \xmapsto{\phi_4} y
& & y \xmapsto{\phi_4^{-1}} (- \sqrt{1-y^2}, y)
\end{align*}
and with transition functions given by (noting that $\phi_i(U_i\cap U_j) = (0, 1) \text{ or } (-1, 0)$), 
\begin{align*}
\phi_3 \circ \phi_1^{-1} & :\phi_1(U_1\cap U_3)\to \phi_3(U_1 \cap U_3) \\
\phi_3 \circ \phi_1^{-1} & : (0,1) \to (0,1) \\
& : x \xmapsto{\phi_1^{-1}} (x, \sqrt{1-x^2}) \xmapsto{\phi_3} \sqrt{1-x^2}
\end{align*}
\begin{align*}
\phi_1 \circ \phi_3^{-1} & : (0,1) \to (0,1) \\
& : y \xmapsto{\phi_1 \circ \phi_3^{-1}}  \sqrt{1-y^2}
\end{align*}
\begin{align*}
\phi_1 \circ \phi_4^{-1} & : (-1,0) \to (0,1) \\
& : y \mapsto -\sqrt{1-y^2}
\end{align*}
\begin{align*}
\phi_4 \circ \phi_1^{-1} & : (0,1) \to (-1, 0) \\
& : x \mapsto \sqrt{1-x^2}
\end{align*}
\begin{align*}
\phi_2 \circ \phi_3^{-1} & : (0,1) \to (-1, 0) \\
& : y \mapsto \sqrt{1-y^2}
\end{align*}
\begin{align*}
\phi_3 \circ \phi_2^{-1} & : (-1,0) \to (0,1)\\
& : x \mapsto -\sqrt{1-x^2}
\end{align*}
\begin{align*}
\phi_2 \circ \phi_4^{-1} & : (-1,0) \to (-1,0)\\
& : y \mapsto -\sqrt{1-y^2}
\end{align*}
\begin{align*}
\phi_4 \circ \phi_2^{-1} & : (-1,0) \to (-1,0)\\
& : x \mapsto -\sqrt{1-x^2}
\end{align*}

This atlas can easily be generalised to $S^n$.

\subsubsection{Example: Stereographic Projection on $S^n$}
\finish{We now give a useful atlas on $S^n$}

\subsubsection{Examples: Projective space.}

\question{Is $\realnos^{n+1}\setminus \{0\}$ endowed with subspace topology of $\realnos^{n+1}$?}
\defn{Real projective space} $\realnos P^n := \frac{\realnos^{n+1}\setminus \{0\}}{\sim}$ is the quotient space of $\realnos^{n+1}\setminus \{0\}$ under the relation $x\sim y \iff \exists \lambda \in \realnos \setminus \{0\} : \lambda x=y$. We write the equivalence class of $(x_1, \ldots, x_{n+1})\in \realnos^{n+1}$ as $[x_1, \ldots, x_{n+1}]\in \realnos P^n$. (In other words, under the quotient map $q:\realnos^{n+1}\setminus \{0\}\to \realnos P^n$, $[x_1, \ldots, x_{n+1}]:=q(x_1, \ldots, x_{n+1})$ and recall $U \subseteq \realnos P^n \textrm{ open} \iff q^{-1}(U) \subseteq \realnos^{n+1}\setminus \{0\} \textrm{ open}$.) This is an $n$-dimensional smooth manifold with atlas $\{(U_i, \phi_i)\}_{i=1, \ldots, n+1}$ given by $U_i=\{[x_1, \ldots, x_{n+1}]: x_i\neq 0\}$ and $\phi_i([x_1, \ldots, x_{n+1}])=(\frac{x_1}{x_i}, \ldots , \frac{x_{i-1}}{x_i}, \frac{x_{i+1}}{x_i}, \ldots, \frac{x_{n+1}}{x_i})\in \realnos^n$. \checkCorrect{Note that $q^{-1}(U_i)^C=\realnos^n$ is a closed hyperplane in $\realnos^{n+1}\setminus \{0\}$, hence the complement $q^{-1}(U_i)$ is the union of two hyperspaces $(\realnos^{n+1})^+$,$(\realnos^{n+1})^-$, which is open in $\realnos^{n+1}\setminus \{0\}$}, hence $U_i$ is open in $\realnos P^n$, and one can also show $\phi_i$ are indeed homeomorphisms. \finish{Appendix: show $\phi_i$ homeos?} Note it is easy to see that the inverse of $\phi_i$ is given by $\phi_i^{-1}(a_1, \ldots, a_n)=[a_1, \ldots, a_{i-1}, 1, a_{i}, \ldots, a_n]$ (since $[a_1, \ldots, a_{i-1}, 1, a_{i}, \ldots, a_n]=[\lambda a_1, \ldots, \lambda a_{i-1}, \lambda, \lambda a_{i}, \ldots, \lambda a_n]$, and can \checkCorrect{set $\lambda=x_i$ accordingly}). It is also easy to show that the transition maps $\phi_i\circ \phi_j^{-1}$ are smooth. \finish{appendix: show this}

\finish{Note that $RP^1 \cong S^1$. check and explain.} \question{generalisation? do we get some similar isos to spheres for projective space over $\complexnos$ etc.?} (This is all the hopf fibrations...see in that section)

 Recall the following division algebras: the complex numbers $\complexnos = \langle 1, i: i^2=-1 \rangle$, the quaternions $\mathbb{H} = \langle 1, i, j, k: i^2=j^2=k^2=-1, ij=k=-ji \rangle$ , and the octonions \checkCorrect{$\mathbb{O} = \mathbb{H}\times \mathbb{H}=\langle 1, i, j, k, l: i^2=j^2=k^2=l^2=-1 \rangle$}. \question{def of product $\mathbb{H}\times \mathbb{H}$?} As vector spaces, $\complexnos \simeq \realnos^2$, $\mathbb{H} \simeq \realnos^4$, $\mathbb{O} \simeq \realnos^8$. The \defn{projective spaces} $\complexnos P^n$, $\mathbb{H} P^n$, $\mathbb{O} P^n$ are defined similarly to $\realnos P^n$ (replacing $\realnos$ with the appropriate division algebra), and are also \question{are they all n-dimensional?} smooth manifolds with the \checkCorrect{atlas defined similarly}.
\question{is quotient topology defined on $HP^n, OP^n$ in the same way? Are they second countable/hausdorff?}
\question{Is $CP^n$ a complex manifold?}

\subsubsection{Examples: Grassmanians \checkCorrect{[check spelling]}}


\subsubsection{Examples: manifolds with boundaries.}
\finish{Manifolds with boundaries (see Lee)}

\subsubsection{Riemannian Manifolds}
A \defn{Riemannian manifold} is a smooth manifold $M$ with a \checkCorrect{smooth map $g:p\mapsto g_p$, $p\in M$}, \question{is smooth over all of M or just an open subset?} called a \defn{Riemannian metric}, where $g_p:T_pM\times T_pM\to \realnos$ is an inner product (i.e.\ a symmetric, positive-definite, bilinear form).
Note that it is always possible to find a Riemannian metric on any smooth manifold.

Given a chart $(U, x^1, \ldots, x^n)$ of $M$, $g$ is smooth is \checkCorrect{equivalent} \question{[is it iff or only in one direction?]} to saying that $g_{ij}:=g(\frac{\partial}{\partial x^i}, \frac{\partial}{\partial x^j}):U\to \realnos$ is smooth $\forall i,j=1,\ldots, n$. Note $g_{ij}(p):=g_p(\frac{\partial}{\partial x^i}\vert_p, \frac{\partial}{\partial x^j}\vert_p) \in \realnos$ and we will make use of the shorthand notation $\partial_i:=\frac{\partial}{\partial x^i}$. We can now view $(g_{ij})$ as an $n^2$ matrix in $M_n(C^\infty (U))$, and since at each point $p$, $g_p$ is an inner product (so symmetric and positive definite) this also implies the matrix $(g_{ij})$ is symmetric and \checkCorrect{positive, and also in particular is non-degenerate, thus invertible.} \finish{Explain why the inverse exists more precisely!}

Since $(g_{ij}(p))$ is an invertible matrix over $\realnos$, we write its inverse as $(g^{kl}(p))$, and recall matrix multiplication of a matrix with its inverse gives us the following \checkCorrect{\defn{`contraction'}}
$\sum_j g_{ij}(p)g^{jk}(p)=\delta_i^k(p)$ (where $\delta_i^k\in C^\infty(U)$ such that $\delta_i^k(p)=1$ if $i=k$ or $0$ otherwise). In other words, as smooth functions over $U$ we have $\sum_j g_{ij}g^{jk}=\delta_i^k$. 

Recall also that the vector space of bilinear forms on $V$ can be canonically identified with $V^\ast \otimes V^\ast$, by the map $\eta\otimes \psi \mapsto ((v,w)\mapsto \eta(v)\psi(w))$ \finish{PROVE}. Since $dx^i\vert_p\otimes dx^j\vert_p$, $i,j=1,\ldots, n$, is a basis of $T_pM^\ast \otimes T_pM^\ast$, we can write the bilinear form $g_p=\sum_{i,j} g_{ij}(p)dx^i\vert_p \otimes dx^j\vert_p$, for some $g_{ij}(p)\in \realnos$ as an element of $T_pM^\ast \otimes T_pM^\ast$ under the identification. \checkCorrect{Furthermore it is easy to show, that the coefficients $g_{ij}(p)\in \realnos$ in this expression agrees with the previous definition $g_{ij}(p)=g_p(\frac{\partial}{\partial x^i}\vert_p, \frac{\partial}{\partial x^j}\vert_p)$ under this identification \finish{SHOW}.} Further, this all makes sense if we forget the point dependency, and may write (within a local chart), $g=\sum_{i,j}g_{ij} dx^i\otimes dx^j$. \question{Does symmetry of bilinear form allow us to compress this further, using symmetric product of forms? See Lee, Riem Manifolds.} 
\finish{Define norm, isometry}

\subsubsection{Complex manifolds}
\finish{[WRITE UP!!]}

On an open subset $U$ of a complex manifold $M$, we have local coordinates $z^1,\ldots, z^n\in \smooth_\complexnos (U)$, where the transition functions between different `patches' $U, V\subseteq M$ of the manifold are holomorphic functions of these variables. Then we can write $z^k=x^k+iy^k, k=1,\ldots, n$, where $n=dim_\complexnos M$, and $x^i, y^i$ are \checkCorrect{smooth real functions}. 

\section{Bundles and Sections}

\subsubsection{Fibre Bundles}

A \defn{fibre bundle} consists of three topological spaces $E, B, F$, and a continuous surjection $\pi:E\rightarrow B$ (called the \defn{projection}), such that \defn{`local trivialisations'} exist (i.e. for every $p\in B$, there exists a neighbourhood $U_p\subseteq B$ of $p$ and a homeomorphism $\psi: \pi^{-1}(U_p) \rightarrow U_p \times F$, such that $\pi \vert_{\pi^{-1}(U_p)} = \pi_1 \circ \psi$ where $\pi_1$ is the projection on to the first component). $E$ is called the \defn{total space}, $B$ the \defn{base space}, $F$ the \defn{typical fibre}, and $E_p:=\pi^{-1}(p)$ the \defn{fibre} over $p$. As a shorthand we may denote a fibre bundle as $F\hookrightarrow E\xrightarrow{\pi} B$.

\finish{PUT IN A COMMUTATIVE DIAGRAM/PICTURE FOR LOCAL TRIVS? -- DO THIS FOR VECTOR BUNDLE CASE. SHOULD I MAKE OBSERVATION ABOUT LOCAL TRIVS AT A POINT, OR ONLY FOR VECTOR BUNDLES?}

A continuous map $U:E\rightarrow E'$ is a \defn{\checkCorrect{(fibre)} bundle morphism} between two fibre bundles $E\xrightarrow{\pi} B$ and $E'\xrightarrow{\pi'} B'$ \checkCorrect{with the same typical fibre $F$}, if there exists a map $u:B\rightarrow B'$ such that $\pi' \circ U = u \circ \pi$

\finish{PUT IN COMMUTATIVE DIAGRAM.}

\checkCorrect{What is difference between fibre bundle morphism, vector bunlde morphism - (such that the restricted map $u\vert_{E_p}: E_p \rightarrow E_{f(p)}$ is linear for each $p\in B$??. Do I need to distinguish these??}

\finish{Define Bundle iso. smooth bundle iso. etc. (see Lee.)}

We will work with two types of fibre bundles, namely, vector bundles and principle $G$-bundles. Roughly speaking, in the case of vector bundles, each fibre is isomorphic to $\realnos^k$, whereas for $G$-bundles, the fibres are isomorphic to the Lie group $G$. The notion of a bundle morphism also generalises to both these structures, with some extra conditions to preserve the additional structure. Note that we will use the same terminology `bundle morphism' for morphisms betweeen vector bundles and morphisms between fibre bundles and this will have to be understood from the context. In the case of principle G-bundles, we will always specify `principle bundle morphism'.

\subsection{Examples}
We can view the torus $T^2=S^1\times S^1$ as a fibre bundle over the base manifold $S^1$. We define the projection $T^2 \xrightarrow{\pi} S^1$ as $\pi(x, y)=x$. Hence a fibre at $p\in S^1$ is $E_p := T^2_p = \pi^{-1}(p) = \{(p, y): y\in S^1\} = \{p\}\times S^1 \cong S^1$. Furthermore, we can define a global trivialisation on this space as follows. Given a point $p\in S^1$, $S^1$ is an open neighbourhood of $p$ \question{WHY? in subspace topology of $R^2$?? Answer: no, open in the base $S^1$ itself, which is endowed with subspace topology of $R^2$}, and $\psi:T^2=\pi^{-1}(\{p\}\times S^1)\to S^1 \times S^1$ given by $\psi(x, y)=(x, y)$. Hence $T^2 \xrightarrow{\pi} S^1$ is a trivial bundle $T^2\cong S^1\times S^1$. 
\finish{draw a picture of the torus sitting over $S^1$.} \question{Is this an example of a hopf fibration? If yes, why and what is a hopf fibration formally? Answer: No...ignore.}
\finish{move stuff on local and global trivialisations to fibre bundle section for more generality and for above example to make sense?}

Mobius band (see sinha youtube + milnor ch2)

Double twisted mobius band (see sinha youtube)

Hopf line over $CP^1$. 
Hopf line bundle. 
line bundle on $RP^1$ isomorphic to $S^1$. 

\subsection{Examples: Hopf fibrations.}
\understandBetter{(Hopf invarient is invarient of homotopy group).}
The 4 hopf fibrations from the division algebras R, C, H, O.
double cover of $S^1$ (figure 8). 
$\mathbb{Z}^2 = S^0 \to S^1 \xrightarrow{2:1} S^1$; note last $S^1$ comes from $RP^1$, middle one is boundary of mobius strip;; twisted fibration, so non trivial.
$(S^1\to) S^3$ over $S^2$; ($S^2$ comes from $CP^1$. $S^3$ all lines in $C^2 - \{0\}$).
(gives homotopy groups of spheres. (homotopy group is equivalence class of loops in a space -- throw a loop (sphere) into space and can you catch something with it))

 Consider $E:=S^1=\{(x,y)\in \realnos^2:x^2+y^2=1\}$. We obtain a fibre bundle $S^1\xrightarrow{\pi} \realnos P^1$ with typical fibre $S^0$, and projection $\pi(x, y)=[x, y]$. \checkCorrect{We obtain a global trivialisation $\psi:S^1\to \realnos P^1\times S^0$ given by $p\mapsto ([p], p)$.} \finish{(show it is global triv (homeo), if it is.). (is this correct?? does a global triv exist? why send to $p$ instead of $-p$ in $S^0$? what if we sent to $-p$ instead?). Does this give the double cover vs figure 8 stuff? Need to understand properly.} \checkCorrect{Since we found a homeomorphism $S^1\overset{\psi}{\simeq} \realnos P^1\times S^0$, we can conclude that $\realnos P^1 \simeq \frac{S^1}{S^0}$ \question{why?? true?} (i.e. one dimensional real projective space is the same as the circle with antipodal points identified). \question{why?}} \question{where does the double cover stuff come in?}

Consider $\frac{S^1}{\sim}$ with $p\sim -p \forall p\in S^1$ (i.e. the circle with antipodal points identified). \question{How to consider this as $\frac{S^1}{\mathbb{Z}}$?}


Consider $S^3\xrightarrow{\pi} S^2$ with typical fibre $S^1$, where $\pi(x, y, z)=(x, y)$. \finish{We will see later that this is a principle G-bundle}

\subsubsection{Vector Bundles}
A $k$-dimensional (real, \checkCorrect{smooth}) \defn{vector bundle} is a fibre bundle with typical fibre $\realnos^k$, such that the projection map $E\xrightarrow{\pi} B$ is a smooth surjection of smooth manifolds, and the local trivialisations $\psi:\pi^{-1}(U_p) \rightarrow U_p \times \realnos^k$ are diffeomorphisms. Moreover, the fibres $E_p:=\pi^{-1}(p)$ are required to be $k$-dimensional vector spaces and restricting any local trivialisation $\psi_p:=\psi \vert_{E_p}$ should give a linear isomorphism $\psi_p:\pi^{-1}(p)\rightarrow \{p\} \times \realnos^k \canonicaliso \realnos^k$ at each point $p\in B$. \question{Do we require only a single local triv at each point $p, (U_p)$ in def of fibre bundle, for the notation in last statement to make sense? Or is it because when you restrict, you always get the same iso?...not sure if this is true...see transition functions. Perhaps the notation makes sense up to some transition function.}

Consider two collections of local trivialisations $\{\psi_i \}_i$, $\{\tilde{\psi}_j\}_j$ (defined on $\pi^{-1}(U_i)$, $\pi^{-1}(\tilde{U}_j)$ respectively) 
for a vector bundle $E\xrightarrow{\pi} B$. Assume $U_i \cap \tilde{U}_j\neq \phi$ and given a 
point $p\in U_i \cap \tilde{U}_j$, the map $\psi_i \circ \tilde{\psi}_j^{-1}:=(\psi_i)_p \circ (\tilde{\psi}_j^{-1}\vert_{\{p\}\times \realnos^k})$ 
is a linear isomorphism $\psi_i \circ \tilde{\psi}_j^{-1} : \realnos^k \rightarrow \realnos^k$, 
which can be seen from the following diagram.
[DRAW COMM DIAGRAM]
Hence we can represent $\psi_i \circ \tilde{\psi}_j^{-1}$ by an invertible matrix $(g_{mn})$ \finish{[Write lin alg identification explicitly, \question{choosing standard basis?}]} which depends on the point $p$ (in other words, each entry is a smooth map $g_{mn}\in C^\infty(U_i\cap \tilde{U}_j)$). Whence we obtain a map, called a \defn{transition function}
$g:U_i\cap \tilde{U}_j\rightarrow \mathrm{GL}_k(\realnos)$ such that $g:p\mapsto (g_{mn}(p))$. \question{How to extend domain of the map g to the whole manifold M, using the full collection of trivialisations? Or do we just do this on a single collection of trivialisations, so i should get rid of the i, j everywhere?} \checkCorrect{The image of $g$ which can be shown to be a subgroup of $\mathrm{GL}_k(\realnos)$ is called the \defn{structure group} of the transition function.} \question{When is it a Lie group? is it always a subgroup?, is it always the same subgroup no matter which trivialisations we choose. How to prove these results if true?} \question{Is it possible to define the structrue group in terms of sections instead of local trivialisations?}
\question{...[Do this for fibre bundles instead?? make a point for complex vect bundles/euclidean vect bundles etc that $\realnos^k$ can be replaced by $\complexnos^k$ etc. if true]}
\question{Is there alt def of vector bundles using the tranistion maps instead?}

An \defn{orientation} of an $n$-dimensional \checkCorrect{real?? (does it hold for general v.spaces?)} vector space $V$ is an equivalence class of ordered bases $[e_1, \ldots, e_n]$ under the relation $(e_1, \ldots, e_n)\sim (e_1', \ldots, e_n')$ if and only if \checkCorrect{the change of basis matrix} $(a_{ij})$ (so $e_j=\sum_i a_{ji}e_i'$) has positive determinant $\mathrm{det}(a_{ij})>0$. So to any non-zero vector space we can have exactly two possible orientations $\pm 1$ (corresponding to the two equivalence classes corresponding to the strictly positive and negative determinants). \checkCorrect{Note given two orderings of a basis $\{e_i\}$, with $e_{\sigma(j)}=e_i$ for a permutation $\sigma\in S^n$, have the same orientation if and only if $\mathrm{sign}(\sigma)=1$.} Furthermore an orientation of $+1$ is associated to the standard ordered basis of $\realnos^k$.  
A vector bundle $E\xrightarrow{\pi} M$ with local trivialisations $\psi_U:\pi^{-1}(U)\to U\times \realnos^k$ is \defn{oriented} if the fibres $\pi^{-1}(p)$ have the same orientation as $\realnos^k$ for every $p \in U$ (in other words the local trivialisations are orientation preserving).\explainFurther{We could equivalently say that a vector bundle is oriented if at each point $p\in U$ of \checkCorrect{any} local frame $s_1, \ldots, s_k\in \Gamma(U, E)$, the basis $s_1(p), \ldots, s_k(p)$ \checkCorrect{has the same} orientation.} 
\finish{[MOVE ORIENTATION STUFF TO AFTER SECTIONS ETC]} \explainFurther{Note that the structure group of an oriented vector bundle can be restricted to the subgroup $\mathrm{GL}_k(\realnos)^+=\{A\in \mathrm{GL}_k(\realnos): \mathrm{det}(A)>0\}$.} \checkCorrect{check!} 

A \defn{complex vector bundle} has typical fibre $\complexnos^k$. Hence the fibres $E_p$ are $k$-dimensional vector spaces over $\complexnos$. \finish{Note that complex vector bundles are canonically oriented vector bundles since...} Note also that we may have complex vector bundles where the \checkCorrect{total space and} base space are real manifolds \question{Example of this?}.

[MOVE FOLLOWING TWO PARAS TO EXAMPLES-- NEW BUNDLES FROM OLD.]

\question{How to construct a vector bundle in practice by gluing together some specific fibres to give the total space E, and then specifying local trivs or otherwise sections of E? What is the theory behind this?}

Note that given a real vector space $V$ we can construct another real vector space $V^\complexnos := V\otimes_\realnos \complexnos$, \checkCorrect{where $\complexnos$ is considered as a real vector space}. $V^\complexnos$ can be made into a complex vector space by defining complex multiplication $\beta(v\otimes \alpha) = v\otimes (\beta \alpha)$ for $v\in V$, $\alpha, \beta \in \complexnos$. The \checkCorrect{complex} vector space $V^\complexnos$ is called the \defn{complexification} of $V$. Similarly we can obtain a complex vector bundle $E^\complexnos := E\otimes \complexnos$ from a real vector bundle $E$, by complexifying each fibre. \checkCorrect{Note that this is a special case of the tensor product of bundles. \question{How can $\complexnos$ be considered as a vector bundle?}} To be explicit, the fibres of $E^\complexnos$ are given by $(E\otimes \complexnos)_p := E_p\otimes \complexnos$ and \finish{local trivialisations/sections ??? Why is the new bundle a vector bundle?---SEE tensor product of bundles example and if this is just a special case...do the tensor product example first}

Conversely we can obtain a real vector space from a complex vector space by \finish{restriction of scalars. We can obtan a real vector bundle from a complex vector bundle by restricting. (See Milnor for more general restrictions ---check if it specialises to this case)}

\question{[DO ALL FOLLOWING DEFS/RESULTS on VECT BUNDLES HOLD IN SAME WAY on complex vect bundles etc?]}. (Do i need oriented, riemannian, hermitian vector bundle? -- salamon)

Define euclidean vect bundles (and riem manifolds)? (milnor pg21-23)

The \defn{tangent space} $T_p M$ at a point $p\in M$ is  the \explainFurther{algebra of point derivations of $C^\infty (M)$}. \finish{Is product in the algebra composition?}
In particular a \defn{tangent vector} $X_p:C^\infty (M)\to \realnos \in T_p M$ satisfies the following Liebniz rule $X_p(fg)=f(p)X_p(g) + X_p(f)g(p)$, for $f,g\in C^\infty (M)$. Equivalently we can define a tangent vector at $p\in M$ \explainFurther{as the derivative to a curve $\gamma: (-\epsilon, \epsilon)\to M$ in the manifold with $p\in \gamma (-\epsilon, \epsilon)$}. 
 \finish{Other def of tangent vectors?}

Given a chart $(U, \phi)$ of $M$, we can write $\phi = (x^1, \ldots, x^n)$, where $x^i := \pi^i\circ \phi$, and $\pi^i:\realnos^n \to \realnos$ is the canonical projection. We obtain a basis $\{\frac{\partial}{\partial{x^i}} \vert_p\}_{i=1,\ldots, n}$ 
of $T_p M$, where we define $\frac{\partial}{\partial{x^i}} \vert_p f := \frac{\partial}{\partial{\pi^i}}\vert_{\phi (p)} f\circ \phi^{-1}$. Note in particular that 
$\mathtt{dim} (T_p M) = n$ where 
$n$ is the dimension of $M$. \finish{Write coordinate formula for applying differential?}

The \defn{tangent bundle} over an $n$-dimensional manifold $M$ is the $n$-dimensional vector bundle $TM:=\cup_p T_p M\xrightarrow{\pi} M$ with $\pi(X_p)=p$ for $X_p\in T_p M$. Note that given a chart $(U, \phi)$ of $M$ with coordinates $\phi =(x_1, \ldots, x_n)$, we obtain a local frame $\{\frac{\partial}{\partial{x^i}}\}_i$ for the tangent bundle, \explainFurther{and the corresponding local trivialisation}. \finish{MOVE TO lower section after global frames are discussed. Perhaps tangent bundle etc. should be an example.}

Go into derivatives/jacobians etc??

A \defn{section} of a vector bundle $E\xrightarrow{\pi} B$ is a smooth map $\sigma:B\rightarrow E$, such that $\pi \circ \sigma=\id_B$ (in other words, $\sigma(p)\in E_p, \forall p\in B$). The set of all sections of a vector bundle $E$ is denoted $\Gamma (E)$ and is a module over smooth functions $C^\infty(M)$. \question{true in complex case, functions into complex nos also?}

Given an open subset $U\subseteq M$, we denote the \checkCorrect{$C^\infty (U)$}-module of sections on $U$ rather than on the whole base manifold, as $\Gamma (U, E)$. A \defn{local frame} for a vector bundle $E$ is a collection of local sections $s_1, s_2, \ldots, s_k\in \Gamma (U, E)$, such that for every $p\in U$, $s_1(p),\ldots, s_k(p)$ is a basis of the fibre $E_p$. \question{Do I need a collection of U that covers M? for a local frame?} Hence we can \understandBetter{smoothly} decompose any section $s\in \Gamma(U, E)$ in terms of a local frame $s(p)=\sum_{i=1}^k f_i(p)s_i(p)$, for some $f_i(p)\in \realnos$, giving us a local expression $s=\sum_{i=1}^k f_i s_i$ for $f_i\in C^\infty(U)$. 

We can identify local trivialisations with local frames. Given a local frame $s_1, \ldots, s_k\in \Gamma(U, E)$, define a local trivialisation $\psi^{-1}:U\times \realnos^k \to \pi^{-1}(U)$ by
$\psi^{-1}(p, a_1, \ldots, a_k) = \sum_{i=1}^k a_is_i(p)$. \finish{show this is indeed a local triv? - appendix?} Conversely, given a local trivialisation $\psi:\pi^{-1}(U) \to U\times \realnos^k$, define a local frame $s_1,\ldots , s_k\in \Gamma(U, E)$ by $s_i(p)=\psi^{-1} (p, e_i)$, where $e_i = (0, \ldots, 0, 1, 0, \ldots, 0)$ with the $1$ in the $i$'th position. \finish{show this is indeed a local frame? - appendix?}

Furthermore a \defn{global frame} (i.e. a set of sections $s_1, \ldots s_k\in \Gamma(E)$ which form a basis at each point $p\in B$) corresponds to a \defn{global trivialisation} $\psi:E \to M\times \realnos^k$ \finish{[Milnor, T2.2]}. If such a global trivialisation exists, we write $E\cong M\times \realnos^k$ (noting that global trivialisations are diffeomorphisms) and say that $E$ is the \defn{trivial vector bundle} of dimension $k$. 
Note that if in particular, we cannot find $k$ non-vanishing sections in a vector bundle (in which case we cannot find $k$ linearly independent vectors in $E_p$ at every point $p\in B$), then a global frame does not exist, hence the vector bundle is non-trivial.  

\finish{[See later: G-bundles can have vanishing sections with corresponding global trivs --- true?? --check.].}

A \defn{vector field} is a section of the tangent bundle $TM$. The $C^\infty (M)$ module of vector fields is denoted $\mathfrak{X}(M):=\Gamma(TM)$.
\question{Q: is there a name for sections of general vector bundle. ANSWER: NO. (like is it called a tensor field or something)?} \question{Are there any results on conditions on the base manifold of a space for when the tangent bundle gives a global trivialisation? } ANSWER: yes, see parallelizable manifolds.

\subsection{Complex Bundles}

We will adopt the following notation throughout this section. Let $E\to M$ be a $\complexnos$-vector bundle over a smooth manifold of rank $r$.
Let $f=(e_1,\ldots, e_r)$ be a local frame of the bundle over some open neighbourhood $U\subseteq M$. (Hence $e_j\in \Gamma(U, E)$ are linearly independent at each $p\in U$).

\subsubsection{Local frames}
A smooth map $g:U\to GL_r(\complexnos)$ is called a \defn{change of frame mapping}, since $fg$ is another local frame on $U$ given by 
$$fg=(\sum_{k=1}^r g_{k1}e_k,\ldots, \sum_{k=1}^r g_{kr}e_k)$$
noting that $g_{kj}\in \smoothCmaps$. We may also express this in terms of matrix multiplication as $(fg)(p):=f(x)g(x)$. Note if we wrote $f$ as a column vector instead of a row vector then we would write $gf$ instead of $fg$.

One can show that given any two local frames $f, f'$ of $E$ over $U$, there always exists a change of frame mapping $g:U\to GL_r(\complexnos)$ such that $f'=fg$. 

\subsubsection{Local representation of sections}
Given a local frame $f=(e_1,\ldots, e_r)$ we may express a section $s\in \Gamma(U, E)$ locally as
$$s=\sum_{i=1}^r s^i(f)e_i$$
for some unique $s^i(f)\in \smoothCmaps$ ($i=1,\ldots, r$), and we define 
$$s(f):= \begin{pmatrix}s^1(f) \\ \vdots \\ s^r(f)\end{pmatrix} \in \smoothCmaps^r$$
Hence locally we have shown that $\Gamma(E, U) \simeq \smoothCmaps^r$.

We can view this same fact in terms of local trivialisations.  Let $\psi: E\vert_U \to U\times \complexnos^r$ be a local trivialisation (which we can assume is associated with the frame $f$,  but the following holds in more generality also).  Then we may write
\begin{align*}
\Gamma(E, U) & \simeq \smoothCmaps^r \\
s & \mapsto \psi\circ s = (s^1, \ldots,  s^r)
\end{align*}
for $s^i\in \smoothCmaps$.  
The last equality holds since $\psi\circ s$ is a section of $U\times \complexnos^r$,
\explainFurther{WHY?} 
\checkCorrect{ARE we saying that the bundles are essentially $\psi$-related, which is why composition is also a section?}
and $\Gamma(U\times \complexnos^r)$ are essentially smooth maps $U\to \complexnos^r$ [GO OVER WHY], which by taking components is the same as $r$ smooth maps,  i.e.  $\smoothCmaps^r$.  

Suppose $g:U\to GL_r(\complexnos)$ is a change of frame mapping, so that $f'=fg$ is another local frame. Then we obtain the following transformation law $s(f')=s(fg)=g^{-1}s(f)$, or in other words
$$gs(f')=s(f)$$
which follows from direct computation since $s^i(fg)=\sum_{j=1}^r g_{ij}^{-1} s^j(f)$.

\subsubsection{Hermitian Vector bundles}
On a complex vector space $V$, a \defn{Hermitian inner product} is a map $(\cdot, \cdot):V\times V\to \complexnos$ such that for all $u,v\in V$
\begin{align*}
& (u,v)=\overline{(v, u)} \\
& (\lambda u + v, w) = \lambda (u, w) + (v, w) \\
& (u, v)\geq 0 \\
& (u,v)=0, \forall v \implies u=0
\end{align*}
Or in other words $(\cdot, \cdot)$ is a conjugate-symmetric, sesquilinear, positive, non-degenerate map. 

For example on $V=\complexnos^n$, $(x, y)=\sum_i x_i\overline{y_i}$, $\forall x,y\in \complexnos^n$, is a Hermitian inner product. Note in this case we have that for a matrix $A\in M_n(\complexnos)$ we have $(Au, v)=(u, \overline{A}^Tv)$.

A \defn{Hermitian vector bundle} assigns a Hermitian inner product to each fibre $E_p$ of the vector bundle. More formally, for every open neighbourhood $U\subseteq M$, and for every pair of sections $s, \gamma \in \Gamma(U, E)$, \defn{a Hermitian metric} $h$ on $E$ is a map $h(s, \gamma):=\langle s, \gamma \rangle:U\to \complexnos$ such that $\langle s, \gamma \rangle(p)=\langle s(p), \gamma (p) \rangle$ is a Herimitian inner product on $E_p$, and $\langle s, \gamma \rangle$ is smooth. 

One may show that it is possible to construct a Hermitian metric on any complex vector bundle (the construction of which involves a partitions of unity). 

\subsubsection{Hermitian metric local representation}
Given a local frame $f=(e_1, \ldots, e_r)$ over $U$, define 
$$h(f)_{ij}=\langle e_i, e_j \rangle \ (:U\to \complexnos)$$
Whence $h(f)\in M_r(\smoothCmaps)$ is a positive definite, Hermitian, symmetric matrix, that represents $h$ locally with respect to the frame $f$. 

In terms of a local trivialisation $\psi:E|_U\to U\times \complexnos^r$ a Hermitian metric locally on $E|_U$ can be expressed as requiring
$$\langle \cdot, \cdot \rangle_p := h_p(\psi^{-1}_p(\cdot), \psi^{-1}_p(\cdot)):\complexnos^r \times \complexnos^r \to \complexnos$$
to be a Hermitian inner product at each $p\in U$. 

Recall given $s, \gamma\in \Gamma(U, E)$, we can write $s(f)=(s^1(f), \ldots, s^r(f))^T$, $\gamma(f)=(\gamma^1(f), \ldots, \gamma^r(f))^T$ locally with respect to the frame $f$. Then one may show that 
$$\langle s, \gamma \rangle = \overline{\gamma(f)^T}h(f)s(f)$$
(where the product here is matrix multipilicaiton).

Furthermore, if $g:U\to GL_r(\complexnos)$ is a change of frame mapping with $f'=fg$. Then we get the transformation law
$$h(f')=\overline{g^T}h(f)g$$
for the local representation of $h$ with respect to the frames $f, f'$.

\subsubsection{Holomorphic line bundle}
\finish{WRITE up and move to chapter on bundles}

\subsubsection{Hermitian metric on holomorphic line bundle}
\finish{UNDERSTAND THIS SECTION BETTER and rewrite!}

Let $E$ be a holomorphic line bundle, with global holomorphic sections $s^1, \ldots, s^r$, \question{[Is this in fact a global holomorphic frame??]} and let $\psi:E|_U\to U\times \complexnos^r$ be a local trivialisation over $U$ (so that for $p\in U$, $\psi_p:E_p\to \{p\}\times \complexnos^r \simeq \complexnos^r$). Then $\forall p\in U$ define
$$h(v_p):=\frac{|\psi_p(v_p)|^2}{\sum_{i=1}^r |\psi_p (s^i_p)|^2}$$
for $v_p\in E_p$. This is a Hermitian inner product at $p$, and hence gives a hermitian metric on $E$. Note by an abuse of language, this can be written in shorthand as $h(v_p)=(\sum_i |s^i|^2)^{-1}$.

\understandBetter{In particular, the line bundle on $\complexnos P^n$ (with its standard globally generating sections $s^1, \dots, s^r$) has this Hermitian metric.}

\subsubsection{Induced Hermitian metrics}
Suppose the bundles $E, F$ are endowed with hermitian structures. Then
$$E\oplus F$$
$$E\otimes F$$
$$\Hom(E, F)$$
all have Hermitian structures. If $(M, g)$ is a Hermitian manifold \finish{DEF OF HERMITIAN MANIFOLD} then
$$TM$$
$$TM^*$$
$$\bigwedge^{p,q} M$$
all have Hermitian structures.
\finish{REWRITE SECTION WITH HOW THESE ARE DEFINED PRECISELY - FIND REF or figure out.}

Suppose $f:M\to N$ smooth map between manifolds, and $E$ is a bundle on $N$, and $N$ has a hermitian structure $h$. Then the pull back bundle $f^*E$ has a hermitian structure given by the hermitian inner product
$$(f^*h)_p:=h_{f(p)}$$
on the fibres $(f^*E)_p=E_{f(p)}$.

\subsubsection{Multilinear algebra}
We state here some isomorphisms of vector spaces without much justification, to review concepts from multi-linear algebra, which will be useful to us. Let $V, W$ be finite dimensional vector spaces over a field, say the real numbers for simplicity. 

Some notation:
$$T^{(r,s)}:=\underbrace{V^*\otimes \cdots \otimes V^*}_{s \text{ times}} \otimes \underbrace{V \otimes \cdots \otimes V}_{r \text{ times}} $$
$$\Hom(V_1, \ldots V_n; \realnos):= \{\text{Multilinear maps } V_1\times \cdots \times V_n \to \realnos\}$$
$$\Hom(V_1, \ldots V_n; W):= \{\text{Multilinear maps } V_1\times \cdots \times V_n \to W\}$$

Some isomorphisms:
\begin{align*}
V^*\otimes W & \simeq \Hom_\realnos(V, W) \\
V^*\otimes V & \simeq \Hom_\realnos(V, V) = \text{End}_\realnos (V) 
\end{align*}
Note in particular that the dimension of $V^*\otimes W$ is $\text{dim}(V)\text{dim}(W)=\text{dim}(\Hom(V, W))$.

\begin{align*}
V_1^* \otimes V_2^* \otimes \cdots \otimes V_n^* & \simeq (V_1\otimes \cdots \otimes V_n)^* \\
& \simeq \Hom_\realnos (V_1, \ldots, V_n; \realnos) 
\end{align*}

\begin{align*}
V_1^* \otimes V_2^* \otimes \cdots \otimes V_n^* \otimes W & \simeq \Hom_\realnos((V_1\otimes \cdots \otimes V_n), W) \\
& \simeq \Hom_\realnos (V_1, \ldots, V_n; W) 
\end{align*}

Note that linear maps $V_1\otimes \cdots \otimes V_n\to \realnos$ are the same as multilinear maps $V_1\times \cdots \times V_n \to \realnos$.

Another useful identity:
$$V\otimes (W\oplus U)=(V\otimes W)\oplus (V\otimes U)$$

\subsubsection{Tensoriality}
Let $\pi_E:E\to M, \pi_F:F\to M$ be two (complex) vector bundles over $M$, ($p\in M$).

\finish{Write more bundles: symmetric product etc? See Huybrechts or any other ref for useful bundles}

Suppose $(\pi_E, E, M)$, $(\pi_F, F, M)$ are two (complex) vector bundles over the same (smooth) base manifold $M$. A \defn{bundle morphism} is a smooth map $\phi:E\to F$ that satisfies
\begin{enumerate}
    \item $\pi_F\circ \phi = \pi_E$ \finish{[DRAW DIAGRAM!]}
    \item $\phi \vert_{E_p}=:\phi_p:E_p\to F_p$ is linear, $\forall p\in M$
\end{enumerate}
Note that the first property in particular implies that $\phi(E_p)\subseteq F_p$ at every $p\in M$ (and so the codomain in the second property makes sense).  

\finish{WHAT IF ISOMORPHISM OF BUNDLES: GIVES ISO?}

We denote the set of all bundle morphisms from $E$ to $F$ as \defn{Hom($\mathbf{E, F}$)}. One can also multiply a bundle morphism $\phi \in \text{Hom}(E, F)$ by a smooth function $f\in C^\infty (M)$ to produce another bundle morphism $f \phi :E\to F$,
\understandBetter{given by $v\mapsto f(p)\phi(v)$ for all $v\in E_p$.} \finish{PROVE this is indeed a morphism}. 
\checkCorrect{This makes $\text{Hom}(E, F)$ into a $C^\infty(M)$-module.} 

A useful and important fact is that if $\phi:E\to F$ is a bundle morphism, then 
\begin{align*}
\tilde{\phi}:\ & \Gamma(E)\to \Gamma(F) \\
& s \mapsto \phi \circ s
\end{align*}
is a homomorphism of $C^\infty(M)$-modules. (In other words, $\tilde{\phi}(fs)=f\tilde{\phi}(s)$ for all $f\in C^\infty (M), s\in \Gamma(E)$.) \checkCorrect{Note that at each point $p\in M$, $(\phi\circ s)(p)=\phi(p)s(p)$.}

Moreover, this induces a \checkCorrect{module} isomorphism 
\begin{align*}
\text{Hom}(E, F) & \simeq \text{Hom}_{C^\infty (M)}(\Gamma(E), \Gamma(F)) \\
\phi & \mapsto \tilde{\phi} 
\end{align*}

Conversely if a map $L:\Gamma(E)\to \Gamma(F)$ is $C^\infty(M)$-linear (i.e. $L_p(fs)=fL_p(s)$ for all $f\in C^\infty (M), s\in \Gamma(E), p\in M$), then there exists a unique bundle morphism $\phi:E\to F$ such that $L=\tilde{\phi}$, in which case we say that $L$ is \defn{tensorial}. 

As an example, we show that 1-forms are tensorial. Consider $\omega\in \Omega(M) = \Gamma(TM^*)$. So $\omega:\Gamma(TM)\to \smooth (M)\simeq \Gamma(M\times \realnos)$. By definition, note that for $x\in M$, $X\in \Gamma(TM)$, we have $\omega(X)_x = \omega_x(X_x)\in \realnos$. Now for $f\in \smooth(M)$, we have $\omega(fX)_x = \omega_x((fX)_x)= \omega(f(x)X_x) = f(x)\omega_x(X_x)$, hence $\omega(fX)=f \omega(X)$. So $\omega$ is tensorial and comes from a bundle morphism $TM\to M\times \realnos$ over the base manifold $M$. In particular $\omega_p:T_p(M)\to \{p\} \times \realnos \simeq \realnos$ is linear.  

We can generalise this definition of tensoriality to say that a $\smooth(M)$-multilinear map $\phi:\Gamma(E_1)\times \Gamma(E_2)\times \ldots \times \Gamma(E_k)\to \Gamma(F)$ is tensorial, (where $E_i, F$ are bundles over $M$). Or in other words, $\phi$ is tensorial as a $\smooth(M)$-linear map $\Gamma(E_1)\otimes_{\smooth(M)} \cdots \otimes_{\smooth(M)} \Gamma(E_k)\to \Gamma(F)$. Note that $\Gamma(E_1)\otimes_{\smooth(M)} \cdots \otimes_{\smooth(M)} \Gamma(E_k)\simeq \Gamma(E_1\otimes \cdots \otimes E_k)$, hence if $\phi$ is tensorial, it comes from a bundle morphism $E_1\otimes \cdots \otimes E_k\to F$.

As an example of this consider tensor fields, namely 
\begin{align*}
T & \in \Gamma(T^*M\otimes \cdots \otimes T^*M \otimes TM) = \Gamma(T^{(1, k)}(TM)) \\
& \ \ \simeq \text{Hom}_{\smooth(M)}(\Gamma(TM), \Gamma(TM), \ldots, \Gamma(TM); \Gamma(TM)) \\
& \ \ = \text{Hom}_{\smooth(M)}(\mathfrak{X}(M), \mathfrak{X}(M), \ldots, \mathfrak{X}(M); \mathfrak{X}(M))
\end{align*}
which are $\smooth(M)$-multilinear, which is seen from the last isomorphism (which is just a generalisation of $\Gamma(E^*\otimes F)\simeq \Hom_{\smooth (M)}(E, F)$). Hence every tensor field  $T:\mathfrak{X}(M)\times \mathfrak{X}(M) \ldots \times \mathfrak{X}(M)\to \mathfrak{X}(M)$, comes from a bundle morphism $TM\otimes \cdots \otimes TM\to TM$.

As an example, we will see that \checkCorrect{Riemannian} curvature $R:\Gamma(TM)\times \Gamma(TM) \times \Gamma(E)\to \Gamma(E)$ is $\smooth(M)$-trilinear (i.e. is tensorial), and hence comes from a bundle morphism $TM\otimes TM\otimes E\to E$. Furthermore torsion $T:\Gamma(TM)\times \Gamma(TM)\to \Gamma(TM)$ is $\smooth (M)$-bilinear, and hence comes from a bundle morphism $TM\otimes TM\to TM$.

\understandBetter{Another example of a tensorial/not tensorial map are composition of vector fields?? But Lie bracket is tensorial?? [true?]....}

Note that if $L$ is not tensorial, then it does not in general give rise to a bundle morphism. We will see that some non-examples include the connection operator (which is of the form $L_p(fs)=fL_p(s)+X f\cdot s$, where we have added an extra term due to a Liebniz rule), \checkCorrect{and the Lie bracket of vector spaces}. However we will also see that any two connections differ by a tensorial map, i.e. that $\nabla^1-\nabla^2$ is tensorial for connections $\nabla^1, \nabla^2$, which will allow us to express connections locally in terms of this operator. 

Another non-example is the exterior derivative $d:\Gamma(M\times \realnos)=\smooth(M)\to \Omega^1(M)=\Gamma(T^*M)$. For $d$ to be tensorial we would require $d(fg)=f\cdot dg$ for $f\in \smooth(M)$, $g\in \Gamma(M\times \realnos)\simeq \smooth(M)$. However this is in general not the case, due to the Liebniz rule, i.e. $d(fg)=g\cdot df + f\cdot dg$ and $g\cdot df\neq 0$ in general. 


\subsubsection{Isomorphisms and bundle morphisms}
\checkCorrect{The technicalities of viewing bundles in a variety of ways will become useful in later exposition. We also observe here that constructions relating to vector spaces (local constructions) can be generalised into global statements relating spaces of smooth sections with bundle morphisms.}

Let $\pi_E:E\to M, \pi_F:F\to M$ be two (complex) vector bundles over $M$, ($p\in M$).

The \defn{dual bundle} of $E$ is $\pi_{E^\ast}:E^\ast \to M$ with fibres $(E^*)_p = (E_p)^*$.

The \defn{tensor product} of $E$ and $F$ is a bundle $\pi_{E\otimes F}:(E\otimes F) \to M$ with fibres $(E\otimes F)_p:=(E_p \otimes F_p)$. Furthermore given sections $s_E\in \Gamma(E), s_F\in \Gamma(F)$ then define $s_E\otimes s_F\in \Gamma(E\otimes F)$ as $(s_E\otimes s_F)(p)=s_E(p)\otimes s_F(p)$ for all $p\in M$. Lastly note that if $\{s^1, \ldots, s^k\}$ is a local frame for $E$, and $\{\sigma^1,\ldots , \sigma^t\}$ is a local frame for $F$, then $\{s^i\otimes s^j : i=1,\ldots, k, j=1,\ldots , t\}$ is a local frame for $E\otimes F$, (since $s^i(p)\otimes s^j(p)$ gives a basis of $E_p\otimes F_p$).  

We now wite another useful $C^\infty(M)$-module isomorphism. Let $\pi_{E^\ast}:E^\ast \to M$ be the dual bundle of $\pi_E:E \to M$. Then 
\begin{align*}
\Gamma(E^\ast) & \simeq \text{Hom}(E, M\times \complexnos) \\
s & \xmapsto{\simeq} (v_p\mapsto (p, s_p(v_p))
\end{align*}
for all $v_p\in E_p$, $p\in M$. Note this makes sense as for $s\in \Gamma(E^*)$, $p\in M$, then $s_p:E_p\to \complexnos$ is a linear map (i.e. an element of $(E_p)^*$).
Note also that $M\times \complexnos$ is the trivial line bundle. \checkCorrect{Is this really true globally, or do we need to do this in a chart and have a local trivialisation?? (Effects next para too in this case)}

As a corollary of the previous two isomorphisms we have 
$$\Gamma(E^\ast)\simeq \text{Hom}_{C^\infty (M)}(\Gamma(E), C^\infty (M))$$
since $\Gamma(E^\ast) \simeq \text{Hom}(E, M\times \complexnos) \simeq \text{Hom}_{C^\infty (M)}(\Gamma(E), \Gamma(M\times \complexnos))\simeq \text{Hom}_{C^\infty (M)}(\Gamma(E), C^\infty (M))$. The last equality holds since we have a one-dimensional global trivialisation $M\times \complexnos$, and we saw that we can write a section locally as smooth functions given a local trivialisation (i.e. $\Gamma(U\times \complexnos^r)\simeq C^\infty (M)^r$, with $r=1, U=M)$.

As an application, if we set $E=TM$ to be the tangent bundle on $M$ in this isomorphism, then $\Gamma(TM^*) = \Omega^1(M) \simeq \text{Hom}_{C^\infty (M)}(\Gamma(TM), C^\infty (M))$. So given any one-form $\omega\in \Omega^1(M)$ and vector field $X\in TM$, we have that $\omega(X)\in C^\infty(M)$ is a smooth function.

\checkCorrect{
It is easy to see that we can generalise the previous two isomorphisms as follows
\begin{align*}
\Gamma(\otimes^r E^\ast) & \simeq \text{Hom}(E, M\times \complexnos^r) \\
 & \simeq \text{Hom}_{C^\infty (M)}(\Gamma(E), C^\infty (M)^r)
\end{align*}
} \checkCorrect{IS THIS TRUE/CORRECT?? JUST MADE A GUESS.}

Let us investigate another bundle, namely $E^*\otimes F$ which has fibres 
$$(E^*\otimes F)_p=E_p^*\otimes F_p\simeq \text{Hom}_\complexnos (E_p, F_p)$$
where the last isomorphism is a fact from linear algebra. To be explicit, given vector spaces $V, W,$ recall the vector space isomorphism
\begin{align*}
V^*\otimes W & \simeq \text{Hom}(V, W) \\
\mu \otimes w & \xmapsto{\simeq} (v \mapsto \mu(v)\cdot w)
\end{align*}
which can be applied pointwise at each fibre of $E^*\otimes F$. This can be made into a global statement as follows  \understandBetter{WHY?}
$$\Gamma(E^*\otimes F)\simeq \text{Hom}(E,F)$$
Using that $\text{Hom}(E,F)\simeq \text{Hom}_{C^\infty (M)}(\Gamma(E), \Gamma(F))$ (as seen previously), we get
\begin{align*}
\Gamma(E^*\otimes F) & \simeq \text{Hom}_{C^\infty (M)}(\Gamma(E), \Gamma(F)) \\
\epsilon \otimes \eta & \xmapsto{\simeq} (\sigma \mapsto \epsilon(\sigma) \eta)
\end{align*}
where $\sigma\in \Gamma(E)$, and note that $\epsilon(\sigma)\in C^\infty (M)$. \checkCorrect{To understand why $\epsilon \otimes \eta\in \Gamma(E^*\otimes F)$ here, think of $(\epsilon \otimes \eta)_p=\epsilon_p \otimes \eta_p, \forall p\in M$ giving a section of $E^*\otimes F$; or more formally, note the canonical $C^\infty(M)$-module isomorphisms}
\begin{align*}
\Gamma(E\otimes F) & \simeq \Gamma(E)\otimes_{C^\infty (M)} \Gamma(F) \\
\Gamma(E^*) & \simeq \Gamma(E)^* 
\end{align*}

\section{Differential Forms}

\subsubsection{Differential forms}
\finish{[WRITE UP]}

\subsubsection{Complex differential forms}
Complex differential forms on a manifold are essentially differential forms which are allowed to take complex coefficients. We can write any complex $k$-form as a $(p,q)$-form. Let us make this more precise.

Let $M$ be a complex manifold, with local coordinates $z^1, \ldots, z^n$ over $U$, and $z^k=x^k+iy^k$. We note the following two complex one-forms defined as
$$dz^k := dx^k + i dy^k \in \Omega^1_\complexnos (M)$$
$$d\bar{z}^k := dx^k - i dy^k \in \Omega^1_\complexnos (M)$$
Then any complex one-form $\omega\in \Omega^1_\complexnos (M)$ can be written uniquely as 
$$\omega = \sum_{k=1}^n (f_kdz^k + g_kd\bar{z}^k)$$

\finish{We define $\Omega^k_\complexnos(M)$ as....??}

We define $\Omega^{1, 0}(M)$ to be the forms generated by $dz^k, k=1, \ldots, n$. In other words $\Omega^{1, 0}(M)$ are all possible $\smooth(M)$-linear combinations of the $dz^k$. Define $\Omega^{0, 1}(M)$ to be the forms generated by $d\bar{z}^k, k=1, \ldots, n$. 
Then we define 
$$\Omega^{p, q}_\complexnos (M) := \underbrace{\Omega^{1, 0} \wedge \Omega^{1, 0} \wedge \cdots \wedge \Omega^{1, 0}}_{p \text{ times}} \wedge \underbrace{\Omega^{0, 1} \wedge \Omega^{0, 1} \cdots \wedge \Omega^{0, 1}}_{q \text{ times}}$$
\understandBetter{(Note that one can show these forms transform tensorially - under a change of holomorphic coordinates - hence give rise to complex vector bundles).} An element $\alpha\in \Omega^{p,q}_\complexnos(U)$ can be written locally as 
$$\alpha := \sum_{|I|=p, |J|=q} f_{I,J} dz^I \wedge d\bar{z}^J \in \Omega^{p,q}_\complexnos(U)$$
where $f_{I, J}\in \smooth_\complexnos(U)$, and $I, J$ represent the appropriate multi-indices. 
\understandBetter{UNDERSTAND LOCAL formula better.}

Now one can then show that \checkCorrect{OR is it by definition??}
\begin{align*}
\Omega^k_\complexnos(M) & =\bigoplus_{k=p+q} \Omega^{p, q}_\complexnos (M) \\
& = \Omega^{k, 0}\oplus\Omega^{k-1, 1}\oplus \cdots \oplus \Omega^{1, k-1}\oplus \Omega^{0, k}
\end{align*}
This induces a vector bundle with projection 
$$\pi^{p,q}:\Omega^k_\complexnos (M)\to \Omega^{p, q}_\complexnos(M)$$

\subsubsection{Dolbeault operators}
\checkCorrect{The usual exterior derivative $d:\Omega^r \to \Omega^{r+1}$} gives \understandBetter{$d(\Omega^{p,q}_\complexnos)\subseteq_{r+s=p+q+1} \Omega^{r, s}$}. Hence we define the following operators
$$\partial := \pi^{p+1, q}\circ d:\Omega^{p,q}\to \Omega^{p+1, q}$$
$$\bar{\partial} := \pi^{p, q+1}\circ d:\Omega^{p,q}\to \Omega^{p, q+1}$$
called \defn{Dolbeault operators}, where $\pi^{p, q}:\Omega^k_\complexnos (M)\to \Omega^{p, q}_\complexnos(M)$ is the projection map defined previously. 

We can express the action of the Dolbeault operators locally, given $\alpha := \sum_{|I|=p, |J|=q} f_{I,J} dz^I \wedge d\bar{z}^J \in \Omega^{p,q}_\complexnos(U)$, we have
$$\partial \alpha = \sum_{|I|=p, |J|=q} \sum_l \frac{\partial f_{I,J}}{\partial z^l} dz^l\wedge dz^I \wedge d\bar{z}^J$$
$$\bar{\partial} \alpha = \sum_{|I|=p, |J|=q} \sum_l \frac{\partial f_{I,J}}{\partial \bar{z}^l} d\bar{z}^l\wedge dz^I \wedge d\bar{z}^J$$

One can also show the following properties of Dolbealt operators
$$\partial +\bar{\partial} = d$$
$$\partial^2 = \bar{\partial}^2 = \partial\bar{\partial} + \bar{\partial}\partial = 0 $$
Note the second property $\bar{\partial}^2=0$ gives rise to a cohomology theory called \defn{Dolbeault cohomology} defined as
$$H^{p,q}_\complexnos (M) := \frac{\text{ker}(\bar{\partial}:\Omega^{p,q}\to \Omega^{p, q+1})}{\text{im}(\bar{\partial}:\Omega^{p,q-1}\to \Omega^{p, q})}$$

\subsubsection{Differential forms with vector coefficients}
Define $\bigwedge^p_\complexnos(T^\ast M):=\bigwedge^p (T^\ast M)\otimes \complexnos$, i.e.\ the complexification of $\bigwedge^p_\realnos(T^\ast M)$. Note that we can express the elements of $\Omega^p_\complexnos(M)=\Gamma(\bigwedge^p_\complexnos(T^\ast M))$ as $\omega+i\eta \in \Omega^p_\complexnos(M)$ for $\omega, \eta \in \bigwedge^p_\realnos(T^\ast M)$, and $i$ the imaginary unit. 

We define the differential $p$-forms with coefficients in $E$ as 
\begin{align*}
\Omega^p(M, E) & :=\Gamma(M, \bigwedge {}^p_\complexnos(T^\ast M) \otimes_\complexnos E) \\
& \simeq \Hom(\bigwedge {}^p_\realnos (TM), E)
\end{align*}
\understandBetter{UNDERSTAND THE ISO PROPERLY! Check $R$ vs $C$}
In other words the elements $\omega\in \Omega^p(M, E)$ at a point $x\in M$ are skew-symmetric multilinear forms 
$$\omega_x:\underbrace{T_xM\times T_xM\times \ldots T_xM}_{p \text{ times}} \to E_x$$
assigning a vector in the fibre $E_x$ to the $p$-tuples of tangent vectors at $x$. 

Note in particular that 
$$\Omega^0(M, E) = \Gamma(E)$$
$$\Omega^1(M, E) = \Gamma(T^\ast M \otimes E) \simeq \Hom(TM, E)$$

Note that locally (over $U$),  $0$-forms with coefficients in $E$ are local sections $s\in \Gamma(E, U)$ and hence can be expressed as an $r$-tuple of smooth maps $s=s(f)\in \smoothCmaps^r$ once a frame $f$ is fixed (or alternatively a local trivialisation is chosen),  as seen previously.  

Moreover,  $1$-forms with coefficients in $E$ can be expressed locally (over $U$) as an $r$-tuple of one forms.  \explainFurther{Let us explain this formally. }

\section{Connections}

\subsubsection{Definitions of Connection}

There are a few different equivalent ways to view connections.
We can define a \defn{connection} to be a bilinear map 
\begin{align*}
\nabla: & \Gamma(TM)\times \Gamma(E)\to \Gamma(E) \\
& (X, s)\mapsto \nabla_X s
\end{align*}
such that for all $f\in \smooth (M)$ we have
\begin{align*}
& \nabla_{fX}s=f\nabla_X s \\
& \nabla_X(fs)=(Xf)s + f\nabla_X s 
\end{align*}
Recall that $Xf\in \smooth(M)$ where $(Xf)(p)=X_p(f)\in \realnos$ for $p\in M$. (Note that the order here matters, as $fX\in \Gamma(TM)$ (as $\Gamma(TM)$ module over $\smooth(M)$) and $(fX)(p) = f(p)X(p)$, $p\in M$.) Note that if we fix 
$X\in \mathfrak{X}(M)$, we can think of the connection as an operator $\nabla_X:\Gamma(E)\to \Gamma(E)$.

We have an \defn{affine (or linear) connection} if $E=TM$ in the above definition.

We can also define a \defn{connection} as a $\complexnos$-linear map $\nabla : \Gamma(E)\to \Gamma(T^\ast M\otimes E)$ such that 
$$\nabla(fs)=df \otimes s + f\nabla s$$ 
for all $f\in C^\infty(M), s\in \Gamma(E)$.   

Note that $\Gamma(T^*M\otimes E)\simeq \Hom(TM, E)\simeq \Hom_{\smooth(M)}(\Gamma(TM), \Gamma(E)) = \Hom_{\smooth(M)}(\mathfrak{X}(M), \Gamma(E))$. Hence due to the last isomorphism, we can write for $s\in \Gamma(E), X\in \mathfrak(M)$, that $\nabla s(X)\in \Gamma(E)$. Moreover $\nabla s$ is $\smooth(M)$-linear, i.e. $\nabla s(fX)=f(\nabla s (X))$ for $f\in \smooth (M)$.

To show that these two definitions of a connection are equivalent, let us write $\nabla^1$ for the definition $\nabla^1:\Gamma(E)\to \Gamma(T^*M\otimes E)$ and $\nabla^2$ for the definition $\nabla^2:\Gamma(TM)\times \Gamma(E)\to \Gamma(E)$. If we are given $\nabla^1$, we can define $\nabla^2_X s:=\nabla^1 s(X)$, and it is easy to check the axioms of $\nabla^2$ are satisfied, namely $\complexnos$-bilinearity, the Liebniz rule, and that it is $\smooth(M)$-linear in the first argument. Conversely given $\nabla^2$, we can define $\nabla^1 s(X):=\nabla^2_X s$ and show that the axioms for $\nabla^1$ to be a connection are satisfied. 

Notice that we could have equivalently expressed the connection as $\nabla:\Omega^0(M, E)\to \Omega^1(M, E)$ in the language of differential $p$-forms with coefficients in $E$ (since we showed previously that $\Omega^0(M, E)=\Gamma(E)$, $\Omega^1(M, E)=\Gamma(T^*M\otimes E)$ by definition).  \checkCorrect{This view is helpful to generalise differentiation to higher order forms.}

\subsubsection{Trivial connection on trivial line bundle}
As an example consider the trivial vector bundle of rank 1 (trivial line bundle)
$\pi:E\to M$, with $E=M\times \realnos$.
Note in this case a section on $E$ is a smooth function $\Gamma(E)=\Gamma(M\times \realnos)\simeq \smooth(M)$, since $s(p)\in \{p\}\times \realnos \simeq \realnos$ for $s\in \Gamma(E)$, $\forall p\in M$.
We claim that $\nabla: \Gamma(TM)\times \smooth(M)\to \smooth(M)$ such that $\nabla_X f := Xf$ is a connection on this bundle (called the \defn{trivial connection}). For this we need to show, it is $\realnos$-bilinear, and satisfies the two properties of a connection. 

$\nabla$ is $\realnos$-linear in the first argument since $\nabla_{X+Y}f=(X+Y)f=Xf+Yf=\nabla_Xf + \nabla_Yf$. Note that it is enough to consider $\nabla_{X+Y}$ instead of $\nabla_{aX+bY}$, $a,b\in \realnos$, since real numbers can be treated as constant functions and set $a=b=1$, and use the property $\nabla_{fX}s=f\nabla_X s$ of connections (we are yet to show). Also note that the equality $(X+Y)f=Xf+Yf$ follows from the fact that at every $p\in M$,
$((X+Y)f)(p)=(X+Y)_pf=(X_p+Y_p)f=X_pf+Y_pf=(Xf)(p)+(Yf)(p)=(Xf+Yf)(p)$.

To show that $\nabla$ is $\realnos$-linear in the second argument, we have $\nabla_X(af_bg)=X(af+bg)=aXf+bXg=a\nabla_Xf + b\nabla_X g$, ($f, g \in \smooth (M), a,b\in \realnos$) where $X(af+bg)=aXf+bXg$ is due to the $\realnos$-linearity of the space of point derivations. 

Further $\nabla_{fX}g=(fX)g=f(Xg)=f\nabla_X g$, and to make explicit the middle equality, for $p\in M$, $((fX)g)(p)=(fX)_p(g)=(f(p)X_p)(g)=f(p)X_p(g)=f(p)(Xg)(p)$.

Finally we see that $\nabla_X(fs)=X(fs)=fX(s)+X(f)s=f\nabla_Xs + (Xf)s$ where we have used the Liebniz rule on the product of two functions (since $s\in \Gamma(E)\simeq \smooth(M)$.

Hence we have shown that $\nabla$ is indeed a connection on the trivial line bundle. 

Let us also view this connection in terms of the definition $\nabla:\Gamma(E)\to \Gamma(T^*M\otimes E)$. Recall since $E=M\times \realnos$ in this case, then $\Gamma(E)\simeq \smooth (M)$. Now $\Gamma(T^*M\otimes E)\simeq \Hom_{\smooth(M)}(\Gamma(TM), \Gamma(E)) = \Hom_{\smooth(M)}(\Gamma(TM), \smooth(M)) \simeq \Gamma(TM)^* \simeq \Gamma(T^*M)=\Omega^1(M)$. So we have shown $\nabla:\smooth(M)\to \Omega^1(M)$ (or in other words $\nabla:\Omega^0(M)\to \Omega^1(M)$) on the trivial line bundle. Defining $(\nabla f)X=Xf$ for $f\in \smooth(M)$ as above, we observe this is simply the definition of the exterior derivative $d:\Omega^0(M)\to \Omega^1(M)$ (since $df(X)=Xf$ by definition). So we have that the trivial connection on the trivial line bundle is the exterior derivative (acting on smooth functions $\nabla f=df \in \Omega^1(M)$). 

\subsubsection{Trivial connection on trivial bundle}
We now consider the trivial rank $k$ vector bundle, $E=M\times \realnos^k$ on $M$. In this case we notice that $\Gamma(E)=\Gamma(M\times \realnos^k)\simeq \smooth (M)^k$ since for a section $s\in \Gamma(E)$ and for $p\in M$ we have $s(p)\in E_p=\{p\}\times \realnos^k\simeq \realnos^k$, hence we can write $s(p)=(s^1(p),\ldots, s^k(p))\in \realnos^k$, and $s^i\in \smooth(M)$. Note that $s^i$ is smooth since it is a composition of smooth maps $s^i=\pi_i\circ s$, where $\pi_i$ is the $i$'th projection. Note also that $fs=(fs^1, \ldots, fs^k)$ for $f\in \smooth(M)$. Now we claim that $\nabla:\Gamma(TM)\times \smooth(M)^k\to \smooth(M)^k$ such that $\nabla_X s = (Xs^1, Xs^2, \ldots, Xs^k)$, where $s=(s^1, \ldots, s^k)\in \smooth (M)^k$ and $X\in \Gamma(TM)$ is a connection on the trivial bundle. The proof follows similarly to that of the trivial connection on the trivial line bundle. We call this connection the \defn{standard trivial connection} on the trivial $k$-bundle. 

Let us also view this connection in the definition $\nabla:\Gamma(E)\to \Gamma(T^*M\otimes E)$. Now $\Gamma(E)\simeq \smooth(M)^k$ on the trivial bundle. So  $\Gamma(T^*M\otimes E)\simeq \Hom_{\smooth(M)}(\Gamma(TM), \smooth(M)^k)\simeq \Gamma(TM^*)^k \simeq \Omega^1(M)^k$. Note that we have used the fact here that if $M$ is a module over a commutative ring $R$, then 
$$\Hom(M, R^k)\simeq \underbrace{M^*\oplus \cdots \oplus M^*}_{k \text{ times}} \simeq (M^*)^{k}$$
So we have shown that $\nabla:\smooth(M)^k\to \Omega^1(M)^k$. As before, for $f=(f^1, \ldots, f^k)\in \smooth(M)^k, X\in \Gamma(TM)$, let us define $(\nabla f)(X)=(Xf^1, \ldots, Xf^k)=(df^1(X), \ldots, df^k(X))=(df^1, \ldots, df^k)(X)$, noting that $(df^1, \ldots, df^k)\in \Omega^1(M)^k$.

\subsubsection{Affine connection on $\realnos^n$}
Consider $M=\realnos^n$, $T_pM=\realnos^n$ for all $p\in M$.
The atlas consisting of a single chart $\text{id}=(u^1, \ldots, u^n)$ on $\realnos^n$, gives us a global frame $\{\frac{\partial}{\partial u^i}\}_i$ on $\realnos^n$. Hence we can write any sections $X, Y\in \Gamma(TM)$ as $X=\sum_{i=1}^n x_i \frac{\partial}{\partial u^i}$, $Y=\sum_{i=1}^n y_i \frac{\partial}{\partial u^i}$, for $x_i, y_i \in \smooth(M)$. We define an affine connection $\nabla:\Gamma(TM)\times \Gamma(TM)\to \Gamma(TM)$ such that 
\begin{align*}
\nabla_X Y & := \sum_{i=1}^n X(y_i) \frac{\partial}{\partial u^i} \\
& = \sum_{i, j} x_j\frac{\partial}{\partial u^j}(y_i)\frac{\partial}{\partial u^i}
\end{align*}
called the \defn{standard connection} on the tangent bundle to $\realnos^n$. It is straight forward to show that the axioms for a connection hold. \understandBetter{In fact this is an example of a Levi-Civita connection.}

\subsubsection{Levi-Civita connection}
Let $(M, g)$ be a Riemannian manifold. There exists a unique affine connection $\nabla:\mathfrak{X}(M)\times \mathfrak{X}(M)\to \mathfrak{X}(M)$ with the properties, for $X,Y,Z\in \mathfrak{X}(M)$,
$Xg(Y,Z)=g(\nabla_XY, Z)+g(Y, \nabla_XZ) (\in C^\infty(U))$ (compatibility with the metric) and $[X,Y]=\nabla_XY-\nabla_YX$ (torsion-free), called the \defn{Levi-Civita connection}. \checkCorrect{The uniqueness of the connection can be seen from the second christoffel [CORRECT SPELLING] formula (below).} \finish{GO OVER different formulation of compatibility with metric, and recall lie bracket def/or in local coord. SEE also questions in handwritten notes}.

Note that given a chart $(U, x^1, \ldots, x^n)$ of $M$, any affine connection $\nabla:\mathfrak{X}(M)\times \mathfrak{X}(M)\to \mathfrak{X}(M)$ is completely determined by $n^3$ coefficients $\Gamma^k_{ij}\in C^\infty (U)$ where
$$\nabla_{\partial_i}\partial_j=\sum_{k=1}^n \Gamma^k_{ij} \partial_k \in \mathfrak{X}(U)$$
Note that this fully determines the affine connection since any vector field can be written as $X=\sum_{i=1}^n \alpha_i \partial_i\in \mathfrak{X}(U)$ where $\alpha_i\in C^\infty (U)$, and since \checkCorrect{affine connections are $C^\infty(U)$-bilinear}. In the case that $\nabla$ is a Levi-Civita connection, the $\Gamma^k_{ij}$ are called \checkCorrect{\defn{Christoffel symbols} (CHECK spelling)}.

Let $\nabla$ be a Levi-Civita connection. Let us rewrite the compatiblity with a metric and torsion-free conditions in terms of local coordinates, with respect to the local frame $\{\partial_i\}_{i=1}^n$ of $TU$. Firstly, that $\nabla$ is compatible with the metric:
\begin{align*}
    \partial_i g(\partial_j, \partial_k) & = g(\nabla_{\partial_i}\partial_j, \partial_k) + g(\partial_j, \nabla_{\partial_i}\partial_k) \\
    & = g(\sum_s \Gamma^s_{ij}\partial_s, \partial_k)+g(\partial_j, \sum_s \Gamma^s_{ik}\partial_s)\\
    &= \sum_s (\Gamma^s_{ij}g(\partial_s, \partial_k)+\Gamma^s_{ik}g(\partial_j, \partial_s)) \\ & \textrm{\checkCorrect{(since $g$ is $C^\infty(U)$ bilinear and $\Gamma^s_{ij}\in C^\infty(U)$)}}
\end{align*}
In short $\partial_i g_{jk} = \sum_s (\Gamma^s_{ij}g_{sk}+\Gamma^s_{ik}g_{js})$. Now we find the torsion-free condition:
\begin{align*}
    [\partial_i, \partial_j] & = \nabla_{\partial_i}\partial_j - \nabla_{\partial_j}\partial_i \\
    & = \sum_s (\Gamma^s_{ij} - \Gamma^s_{ji})\partial_s
\end{align*}
and noting that the lie bracket $[\partial_i, \partial_j]=0$ for all $i,j$ \checkCorrect{[WHY? TRUEE??]}, we get $(\Gamma^s_{ij} - \Gamma^s_{ji})=0$, i.e. $\Gamma^s_{ij}=\Gamma^s_{ji}$.

Our goal now is to obtain a formula for the Christoffel symbols of a Levi-Civita connection, called the second Christoffel identity. We will first derive the first Christoffel identity to help us in this end:
\begin{align*}
    & \partial_i g_{jk} + \partial_j g_{ik} - \partial_k g_{ij} \\
    & = \sum_s [(\Gamma^s_{ij}g_{sk}+\Gamma^s_{ik}g_{js})+
    (\Gamma^s_{ji}g_{sk}+\Gamma^s_{jk}g_{is})-
    (\Gamma^s_{ki}g_{sj}+\Gamma^s_{kj}g_{is})] \\
    & = \sum_s 2\Gamma^s_{ij}g_{sk}
\end{align*}
where we used compatibility with the metric in the first step, and torsion-free and symmetry of the metric in the second step. Now to get the second Christorffel identity, we can contract with the inverse $(g^{tk})$ of the metric $g_{ij}$. Applying the contraction to the right hand side of the first Christoffel identity, we get
\begin{align*}
    \sum_k g^{tk}(2\sum_s \Gamma^s_{ij}g_{sk}) 
    & = 2\sum_s \Gamma^s_{ij}\sum_k g_{sk}g^{kt} \\
    & = 2 \sum_s \Gamma^s_{ij}\delta^t_s \\
    & = 2\Gamma^t_{ij}
\end{align*}
Whence we obtain the second Christoffel identity,
$$\Gamma^t_{ij} = \frac{1}{2}\sum_k g^{tk}(\partial_i g_{jk} + \partial_j g_{ik} - \partial_k g_{ij})$$

Note that an invarient formulation (not depending on a local chart) of the second Christoffel identity also exists called the Koszul formula (stated below), which is also easy to derive from the properties of the Levi-Civita connection.
\begin{align*}
2g(\nabla_XY, Z) = & X(g(Y,Z))+Y(g(X,Z))-Z(g(X,Y)) \\
& -g([Y,X],Z)-g([X,Z],Y)-g([Y,Z],X)    
\end{align*}

\subsubsection{Levi-Civita connection on $S^2$}
\finish{WRITE UP}

\finish{WRITE MORE EXAMPLES OF CONNECTIONS ON BUNDLES, PROJECTIVE SPACE, SPHERES ETC}

\subsubsection{Is a connection tensorial?}
We ask does the connection $\nabla:\Gamma(E)\to \Gamma(T^*M\otimes E)$ come from a bundle morphism $E\to T^*M\otimes E$? For it to be tensorial, we would require $\nabla(fs)=f\nabla s$. However in general (by the axiom for connections), $\nabla(fs)=f\nabla s + df\otimes s$, and $df\otimes s \neq 0$ in general, hence a connection is not tensorial.

Although a connection is not tensorial, we will see now that the difference of two connections is.

Let $\nabla^1, \nabla^2$ be two connections on $E$. Then $\nabla^1-\nabla^2:\Gamma(E)\to \Gamma(T^*M\otimes E)$. Now for $f\in \smooth(M)$, $s\in \Gamma(E)$
\begin{align*}
(\nabla^1 - \nabla^2)(fs)
& = \nabla^1(fs) - \nabla^2(fs) \\
& = (f\nabla^1 s + df \otimes s)
- (f\nabla^2 s + df \otimes s) \\
& = f\nabla^1 s - f\nabla^2 s \\
& = f(\nabla^1 - \nabla^2) s
\end{align*}
so $\nabla^1-\nabla^2$ is tensorial and comes from a bundle morphism $E\to T^*M\otimes E$. Loosely we have shown that $\nabla^1-\nabla^2\in \Hom (E, T^*M\otimes E)$ has a pointwise nature, which will allow us to express connections locally.

Now note that 
\begin{align*}
\Hom (E, T^*M\otimes E) & \simeq \Gamma(E^* \otimes T^*M\otimes E) \\
& \simeq \Gamma(T^*M\otimes E^* \otimes E) \\
& \simeq \Gamma(T^*M \otimes \Hom(E, E)) \\
& = \Gamma(T^*M \otimes \text{End}(E)) \\
& = \Omega^1(M, \text{End}(E))
\end{align*}
\understandBetter{is $End(E)$ a bundle itself?? How is it defined?}
so $(\nabla^1-\nabla^2)_p\in T^*_pM\otimes \Hom(E_p, E_p)$ at $p\in M$. Or we can write $A:=\nabla^1-\nabla^2 \in \Omega^1(M, \text{End}(E))$, so that 
$A_p:T_p(M)\to \text{End}(E_p)$ for all $p\in M$. Or in other words for $s\in \Gamma(E), X\in \mathfrak{X}(M)$,
\begin{align*}
& (\nabla^1_X - \nabla^2_X)s = A_X s \\
\Rightarrow & \nabla^1_X s = \nabla^2_X s + A_X s \\
\end{align*}
where $A_X s$ at a point $p\in M$ reads as $(A_p)_{X_p} (s_p)$.

\subsubsection{Local expression of Connections}
Let $E\to M$ be a (complex) vector bundle with a connection $\nabla:\Gamma(E)\to \Gamma(T^*M\otimes E)$.  Let $f=(e_1, \ldots, e_r)$ be a frame over $U\subseteq M$, with corresponding local trivialisation $\psi:\pi^{-1}(U)\to U\times \complexnos^r$. 
We want to find a local formula for $\nabla s$,  for an arbitrary section $s\in \Gamma(E, U)$.
Recall we can write $s$ locally as $s=(s^1, \ldots,  s^r)$,  for $s^i\in \smoothCmaps$,  with respect to the local frame $f=(e_1, \ldots e_r)$ (or equivalently a local trivialisation $\psi$).

Note that over $U$, any bundle is isomorphic to the trivial bundle $U\times \complexnos^r$. We have seen that the exterior derivative $d:\smooth(U)\to \Omega^1(U)$ is a connection on the trivial line bundle. We extend this definition of $d$ to the trivial rank $r$ bundle, by defining $d: \smooth(U)^r\to \Omega^1(U)^r$ componentwise as $d(s^1, \ldots, s^r) = (ds^1, \ldots, ds^r)$, which we also saw was a connection on $U\times \complexnos^r$. 

Hence over $U$, we have that $A:=\nabla-d\in \Omega^1(U, \End(E|_U)) = \Omega^1(U, \End(U\times \complexnos^r))$ is tensorial. Note that $\End(U\times \complexnos^r)=U\times M_r(\complexnos)$ (the trivial bundle of rank $r^2$). So we have that $A\in \Omega^1(U, \End(U\times \complexnos^r)) = \Omega^1(U, U\times M_r(\complexnos)) \simeq \Gamma(T^*U\otimes (U\times M_r(\complexnos))) \simeq \Hom(TU, U\times M_r(\complexnos)) \simeq \Hom_{\smooth(M)}(\mathfrak{X}(U), M_r(\smoothCmaps))$. Thus for any $p\in U$, we have a linear map
\begin{align*}
A_p : T_pM & \to \{p\}\times M_r(\complexnos) \simeq M_r(\complexnos) \\
v & \mapsto \begin{pmatrix}
A_{11}(v) & \cdots & A_{11}(v)\\
\vdots & \ddots & \vdots \\
A_{r1}(v) & \cdots & A_{rr}(v)
\end{pmatrix}
\end{align*}
with $A_{ij}\in T_p^*M$. Hence $A\in M_r(\Omega^1(U))$ is a matrix of one-forms on $U$. 

To see this same fact from a slightly different perspective, the connection $\nabla$ can be written locally with respect to the frame $f=(e_1, \ldots, e_r)$ as 
$$\nabla e_j = \sum_{i=1}^r A_{ij}(\nabla, f)\otimes e_i \in \Gamma(T^*U\otimes E|_U)$$
for some $A_{ij}:=A_{ij}(\nabla, f) \in \Omega_\complexnos^1(U)$. Then to find a local formula for $\nabla s$ for an arbitrary section $s=(s^1, \ldots,  s^r)\in \smoothCmaps\simeq \Gamma(E, U)$, we compute
\begin{align*}
\nabla s & = \nabla (\sum_{i=1}^r s^i e_i) \\
& = \sum_{i=1}^r (d s^i \otimes e_i + s^i \nabla e_i) \\
& = \sum_{i=1}^r (d s^i \otimes e_i + s^i (\sum_{k=1}^r A_{ki}\otimes e_k )) \\
& = \sum_{i=1}^r (d s^i + \sum_{j=1}^r s^j A_{ij})\otimes e_i 
\end{align*}
whence we have (in terms of matrix multiplication) that 
$\nabla s = \sum_{i=1}^r (d s(f) + As(f))\otimes e_i$.

To summarise, we have shown that we can write 
$$\nabla = d + A$$ 
where for $s\in \Gamma(E|_U)$,
\begin{align*}
\nabla s & = ds + As \\ 
(d+A)\begin{pmatrix}
    s^1 \\
    \vdots \\
    s^r
\end{pmatrix}
& = \begin{pmatrix}ds^1 \\ \vdots \\ ds^r\end{pmatrix} + \begin{pmatrix}
s^1A_{11} & \cdots & s^rA_{11}\\
\vdots & \ddots & \vdots \\
s^1A_{r1} & \cdots & s^rA_{rr}
\end{pmatrix} 
\end{align*}
for some $A_{ij}\in \Omega^1(U)$. 

Note using this formula in practice to apply $\nabla s$ to a (local) vector field $X\in \Gamma(TU)$, we get
\begin{align*}
    \nabla s (X) & = \left( \begin{pmatrix}ds^1 \\ \vdots \\ ds^r\end{pmatrix} + \begin{pmatrix}
s^1A_{11} & \cdots & s^rA_{11}\\
\vdots & \ddots & \vdots \\
s^1A_{r1} & \cdots & s^rA_{rr}
\end{pmatrix} \right) X \\
& = \begin{pmatrix}
ds^1 & s^1A_{11} & \cdots & s^rA_{11}\\
\vdots & \vdots & \ddots & \vdots \\
ds^r & s^1A_{r1} & \cdots & s^rA_{rr}
\end{pmatrix} X \\
& = \begin{pmatrix}
ds^1(X) & s^1A_{11}(X) & \cdots & s^rA_{11}(X)\\
\vdots & \vdots & \ddots & \vdots \\
ds^r(X) & s^1A_{r1}(X) & \cdots & s^rA_{rr}(X) 
\end{pmatrix} \in \Gamma(U\times \complexnos^r) = \smooth(U)^r
\end{align*}
If we want to compute the $A_{ij}(X)$ given $\nabla$, note that 
$$s^i = \begin{pmatrix}0 \\ \vdots \\ 0 \\ \mathbbm{1} \\ 0 \\ \vdots \\ 0\end{pmatrix}$$
under the isomorphism $\Gamma(E, U)\simeq \smooth(U)^k$, where $\mathbbm{1}:x\mapsto 1$ is the constant function, and is in the $i$'th position of the vector. Whence
\begin{align*}
    \nabla_X s^i & = d_X s^i + A_X s^i \\
    & = d_X \begin{pmatrix}0 \\ \vdots \\ 0 \\ \mathbbm{1} \\ 0 \\ \vdots \\ 0\end{pmatrix} +
     A_X \begin{pmatrix}0 \\ \vdots \\ 0 \\ \mathbbm{1} \\ 0 \\ \vdots \\ 0\end{pmatrix} 
     = \begin{pmatrix}A_{i1}(X) \\ A_{i2}(X) \\ \vdots \\ X\mathbbm{1} + A_{ii}(X)\\ A_{i, (i+1)}(X) \\ \vdots \\ A_{ik}(X)\end{pmatrix} 
     = \begin{pmatrix}A_{i1}(X) \\ A_{i2}(X) \\ \vdots \\ A_{ii}(X)\\ A_{i, (i+1)}(X) \\ \vdots \\ A_{ik}(X)\end{pmatrix} 
\end{align*}
where the last equality holds since the deriviative of a constant function is zero ($X\mathbbm{1}$=0).

\subsubsection{Local expression for connection on the trivial line bundle}
As an example let us consider an arbitrary connection $\nabla$ on the trivial bundle $M\times \complexnos$ and understand its local formula. 

Recall that the exterior derivative $d$ is a connection on $M\times \complexnos$, and hence $\nabla - d$ is tensorial, and comes from a bundle morphism $M\times \complexnos\to T^*M\otimes E \simeq T^*M\otimes (M\times \complexnos)\simeq T^* M$, where the last isomorphism is a globally analogous to the linear isomorphism $V\otimes \complexnos\simeq V$ (where $V$ is a $\complexnos$-vector space). 
Now $\Hom(M\times \complexnos, T^*M)\simeq \Gamma((M\times \complexnos)^*\otimes T^*M)\simeq \Gamma((M\times \complexnos)\otimes T^*M) \simeq \Gamma(T^*M) = \Omega^1(M)$, (where we have used $M\times \complexnos \simeq (M\times \complexnos)^*$). So we have shown that $\nabla - d\in \Omega^1(M)$. To conclude,
$$\nabla = \omega + d$$
for $\omega \in \Omega^1(M)$, \understandBetter{where $\omega(f)=f\omega$, for $f\in \smooth(M)$.} 

\subsubsection{Local expression for connection on the trivial rank $r$-bundle}
\finish{TO DO}

\subsubsection{Hermitian Connection}
Given a Hermitian vector bundle $(E, h)$, a connection $\nabla$ is a \defn{hermitian connection} with respect to $h$ if
$$d(h(s, \sigma))=h(\nabla s, \sigma)+h(s,\nabla\sigma)$$
for $s, \sigma\in \Gamma(E, U)$. Note that $h(s, \sigma)\in \smooth_\complexnos (U)$, and $d:\smooth_\complexnos (U)\to \Omega_\complexnos^1(U)$ is the exterior differential map.

\understandBetter{UNDERSTAND BETTER AND REWRITE PARAGRAPH. THINK INCORRECT. SEE PG 176 HUYBRECTHS}
Note that if $\alpha \in \Omega_\complexnos^1(U)=\Gamma(TU)\checkCorrect{\simeq \smooth(U)^r}$, $s\in \Gamma(U, E)\simeq \smooth_\complexnos (U)^r$, then $\alpha\otimes s\in \Omega_\complexnos^1(U) \otimes \smooth(U)^r \checkCorrect{\simeq \smooth(U)^r \otimes \smooth(U)^r \simeq \smooth(U)^r \simeq \Gamma(E, U)}$. Then we define 
$$h(\alpha \otimes s, \sigma):=\alpha h(s, \sigma)$$
$$h(s, \alpha \otimes \sigma) = \bar{\alpha}h(s, \sigma)$$
\question{HOW IS $\bar{\alpha}$ defined??}

For a hermitian connection $\nabla$ and $A\in \Omega^1(M, \End(E))$ we know $\nabla ' := \nabla + A$ is a connection (recall tensoriality discussion). Now $\nabla'$ is a hermitian connection if and only if
$$h(As, \sigma)+h(s, A\sigma)= 0$$
for all sections $s, \sigma \in \Gamma(E, U)$. (One can show from this that $A$ is contained in the Lie algebra of skew-hermitian matrices at each point, after  diagonalisation of $h$).

\subsubsection{Chern Connection}
\finish{DEFINE holomorphic vector bundle, what it means for $h$ to be compatible. Does existance/uniquencess not hold if not holomorphic? }

On a holomorphic vector bundle $E$ \checkCorrect{over a complex manifold $M$}, wih hermitian structure $h$, there exists a unique hermitian connection $\nabla$ compatible with the holomorphic bundle, called the \defn{Chern connection}. 

Note the Chern connection $\nabla:\Omega^0(E, M)\to \Omega^1(E, M)=\Omega^{1,0}(E, M) \oplus \Omega^{0,1}(E, M)$ splits as
$\nabla = \nabla' + \nabla''$ where
$$\nabla ': \Omega^0(E, M)\to \Omega^{1,0}(E, M)$$
$$\nabla '': \Omega^0(E, M)\to \Omega^{0,1}(E, M)$$
Locally over a holomorphic frame $f$, we have that if $\nabla=d+A$, then 
$$\nabla ' = \partial + A^{1,0}$$
$$\nabla '' = \bar{\partial} + A^{0,1} \checkCorrect{= A^{0,1}}$$
where $A^{0,1}$, $A^{1, 0}$ \finish{[HOW TO DEFINE??]}. \checkCorrect{Note here the last equality holds since  $\bar{\partial}s=0$ for any holomorphic section $s$. Note further that if $s$ is a holomorphic section of $E$ then $\nabla '' s =0$. [REPEATED FACT!!??]}

\understandBetter{SHOULD WE HAVE
$$\nabla ' = \partial + A$$
$$\nabla '' = \bar{\partial}$$
AND IS CHERN CONNECTION JUST DEFINED AS 
$\nabla ''$. Seems to be a typo in WELLS???
}

Further, the condition that $\nabla$ is compatible with the metric $h$ can be written locally as 
$$dh = hA + \bar{A}^Th$$
which can in turn gives 
$$\partial h = h A$$
$$\bar{\partial} h = \bar{A}^T h$$
by comparing types. From this we get that
$$A= h^{-1}\partial h$$

Finally we state without proof that if $A$ is the connection matrix of $\nabla$ then
$$\partial A= -A \wedge A$$
\understandBetter{UNDERSTAND BETTER. WELLS PROP 2.2}

\checkCorrect{CHECK previous few paras correct and understand better. 
[REF:WELLS PG 78]. Is it in relation to chern connection or hermitian connection?} 

Note that the Chern connection (a connection conpatible with the holmorphic structure), is not the same as a holomorhpic connection, which has a more restrictive definition. 

We remark without explanation that on the tangent bundle of a Kahler manifold, the Chern connection coincides with the Levi-civita connection of the underlying Riemannian metric.  

\subsubsection{Chern connection on the holomorphic line bundle}
Let $E$ be a holomorphic line bundle. \understandBetter{One can show the hermitian structure locally $h\in \smooth(U)$ is a positive real function on this bundle.} Locally the chern connection on this bundle is given by 
$$\nabla = d + \partial \ \text{log} h$$
\finish{[UNDERSTAND THIS FORMULA!! AND WHY?]}

\subsubsection{Chern connection on the complex torus}
\finish{[SEE your notes on HUYBRETCHS AND WRITE]}


\subsubsection{Fubini study metric on tangent bundle of $\complexnos P^n$}
\finish{MOVE THIS SECTION TO APPROPRIATE PLACE}

Let $(U_j, \phi_j)$ be the standard atlas of $\complexnos P^n$, where we recall that $\phi_j:U_j\simeq \complexnos$ such that $\phi_j((z_0 : \ldots : z_n))=(\frac{z_0}{z_j}, \ldots, \frac{z_{j-1}}{z_j}, \frac{z_{j+1}}{z_j}, \ldots, \frac{z_n}{z_j})\sim (1, w^1, \ldots, w^n)$. This gives a local frame for the holomorphic tangent bundle of $\complexnos P^n$, namely $\{\partial_1, \ldots, \partial_n\}$, where $\partial_i=\frac{\partial}{\partial w^i}$ over $U_j$. 

We define a hermitian metric on the tangent bundle for $\complexnos P^n$ as follows. On $U_j$ let
$$h_{i\bar{j}}:=h(\partial_i,\bar{\partial}_j):=\frac{((1+|w|^2))\delta_{ij}-\bar{w}
_i w_j}{(1+|w|^2)^2}$$
where $\delta_{ij}$ is the Kronecker delta and $|w|^2 = |w^1|^2 + \ldots + |w^n|^2$. Note that $h_{i\bar{j}}$ is a positive definite hermitian matrix, and defines the \defn{Fubini-study metric}. 

We define the \defn{Fubini-study form} locally over $U_j$
\begin{align*}
    \omega_{FS} & := \checkCorrect{\frac{i}{2\pi} \sum_{i, j} h_{i\bar{j}} dw^i \wedge d\bar{w}^j} \\
 & = \checkCorrect{ \frac{i}{2\pi} \partial \bar{\partial} \text{log} \left( 1 + \sum_{i=1}^k |w_k|^2  \right) \in \Omega^{1,1}(U_j)} 
\end{align*}


Note one can show that in fact $\omega_{FS}$ can be defined as a global form in $\Omega^{1,1}(M)$ since it is compatible over all intersections $U_i\cap U_j$. 

One can show that $\omega_{FS}$ is a closed, \checkCorrect{real}, positive definite $(1,1)$-form. \question{Definition of a real (1,1) form?? is this correct?} 

Furthermore we show that on $1$-dimensional projective space, $\int_{\complexnos P^1} \omega_{FS}=1$. This is since
\begin{align*}
  \int_{\complexnos P^1} \omega_{FS} & = \int_\complexnos \frac{i}{2\pi}\frac{1}{(1+|z|^2)^2}dz\wedge d\bar{z}\\
  & = \frac{1}{\pi}\int_{\realnos^2}\frac{1}{(1+||(x,y)||^2)^2}dx\wedge dy \\
  & = 2 \int_0^\infty \frac{rdr}{(1+r^2)^2} \\
  & = 2 \left( \frac{1}{2} \int_1^\infty \frac{du}{u^2 }\right) \\
  & = \int_1^\infty \frac{du}{u^2}  = \left[ - \frac{1}{u} \right]_1^\infty = 1
\end{align*}

\subsubsection{Chern connection on $\complexnos P^n$}

Recall the Fubini-study metric $h_{i\bar{j}}$ defined previously on a standard chart $U\subseteq \complexnos P^n$ with coordinates $(w_1, \dots, w_n)$. We define the matrix representation of this metric as
\begin{align*}
H & := \checkCorrect{\frac{1}{2\pi}} \left[  h_{i\bar{j}} \right] \\
& = \checkCorrect{\frac{1}{2\pi}} \frac{1}{(1+|w|^2)^2}\begin{pmatrix}
1+|w|^2-|w_1|^2 & -\bar{w}_1w_2 & \cdots & -\bar{w}_1w_n \\
-\bar{w}_2w_1 & 1+|w|^2-|w_1|^2 & \cdots & -\bar{w}_2w_n \\
\vdots & \vdots & \ddots & \vdots \\
-\bar{w}_nw_1 & \cdots & \cdots & 1+|w|^2-|w_n|^2
\end{pmatrix} \\
& \in M_n(\smooth_\complexnos (U))
\end{align*}
where $|w|^2 = |w_1|^2 + \ldots + |w_n|^2$. 
Then the Chern connection locally on $U_i$ is given by 
$$\nabla = d + \bar{H}^{-1}(\partial \bar{H})$$
\checkCorrect{where $\partial$ is the Dolbeault operator. \understandBetter{OR IS IT JUST componentwise partial derivitave, on each component of the matrix??}}

\subsubsection{Chern connection on holomorphic bundle with hermitian structure}

Note that one can show in fact for any holomorphic vector bundle $E$ with hermitian structure locally given by $H$, the Chern connection is given by $$\nabla = d+\bar{H}^{-1}\partial(\bar{H})$$. \checkCorrect{CHECK THIS IS TRUE}

\subsubsection{Connection on the dual bundle}
Let $\pi:E\to M$ be a vector bundle with connection $\nabla$.

We introduce some notation. For $\eta\in \Gamma(E^*), \sigma\in \Gamma(E)$, write
$$\langle \eta, \sigma \rangle := \eta(\sigma)\in \smooth (M)$$
(and recall that $\Gamma(E^*)=\Gamma(E)^* = \Hom_{\smooth(M)}(\Gamma(E), \smooth(M))$ is the dual module of $\Gamma(E)$). Note also that $\langle \eta, \sigma \rangle = \langle \sigma, \eta \rangle $ since it is analogous to thinking of vectors in $V$ as functionals on $V^*$, i.e. $V^{**}=V$.

We define a \defn{connection on the dual bundle} $E^*\to M$ as 
\begin{align*}
    \nabla^* :  \Gamma(TM)\times \Gamma(E^*) & \to \Gamma(E^*) \\
    (X, \eta) & \mapsto \nabla^*_X \eta
\end{align*}
such that for $X\in \Gamma(TM), \eta\in \Gamma(E^*), \sigma\in \Gamma(E)$,
$$X \langle \eta, \sigma \rangle := \langle \nabla^*_X \eta , \sigma \rangle + \langle \eta, \nabla^*_X \sigma \rangle$$
or in other words
$$ \langle \nabla^*_X \eta , \sigma \rangle := X \langle \eta, \sigma \rangle - \langle \eta, \nabla^*_X \sigma \rangle$$
To show this is indeed a connection, one would need to show that $\nabla^*_X \eta \in \Gamma(E^*)$ (i.e. that it is $\smooth(M)$-linear as a map $\Gamma(E)\to \smooth(M)$) and that $\nabla^*$ is $\smooth(M)$-linear in the first argument and satisfies Liebniz rule. \finish{WRITE PROOF? - see exposition notes}

Note that locally if $\nabla^1 = \nabla^2 + A$, then $(\nabla^1)^* = (\nabla^2)^* + A^*$.

\subsubsection{Connection on tensor product of bundles}
Let $E, F$ be vector bundles over $M$, with connections $\nabla^E, \nabla^F$ respectively.
Define a connection $\nabla$ on $E\otimes F$ as
\begin{align*}
    \nabla : \Gamma(TM)\times \Gamma(E\otimes F) & \to \Gamma(E\otimes F) \\
    (X, s\otimes \sigma) & \mapsto \nabla_X(s\otimes \sigma)
\end{align*}
such that
$$\nabla_X(s\otimes \sigma):=(\nabla^E_X s)\otimes \sigma + s \otimes \nabla^F_X \sigma $$

\subsubsection{More induced connections}
Let $E, F$ be vector bundles over $M$, with connections $\nabla^E, \nabla^F$ respectively. 

For $s\in \Gamma(E), \sigma \in \Gamma(F)$ define \defn{a connection $\nabla$ on $E\oplus F$} as
$$\nabla(s\oplus \sigma)=\nabla^E(s)\oplus \nabla^F(s)$$

\finish{On $\Hom(E, F)$ - GO OVER AND WRITE UP. see your huybrecths notes} 

\finish{On pull back bundle - GO OVER AND WRITE UP. in huybrechts?}. If $\nabla = d+A$ is a connection on $U\subseteq N$ locally and $f:M\to N$ is smooth map of manifolds, then the \defn{pull back connection} is a connection on $M$ locally given by $f^*\nabla |_{f^-1(U)}=d+f^*A$.

\section{Curvature and Torsion}

To motivate curvature, we not that a connection $\nabla$ is not a differential (in other words $\nabla^2\neq 0$ in general). We note that the obstruction to $\nabla$ being a differential is the curvature, which is defined with respect to a connection. 
 
\subsubsection{Definition of Torsion}
Given an affine connection $\nabla:\mathfrak{X}(M) \times \mathfrak{X}(M)\to \mathfrak{X}(M)$, the \defn{torsion} is defined
\begin{align*}
T_\nabla & : \mathfrak{X}(M)\times \mathfrak{X}(M)\to \mathfrak{X}(M)\\
& (X, Y)\mapsto \nabla_X Y - \nabla_Y X - [X, Y]
\end{align*}
and it is easy to check that $T_\nabla$ is $\smooth (M)$-bilinear.

Note that we can express $T_\nabla$ as a $\smooth (M)$-linear map $\Gamma(TM\otimes TM)\to \Gamma(TM)$ and show that it is tensorial. 

\subsubsection{Torsion of Levi-Civita connection on $\realnos^n$}
\finish{WRITE UP - see exposition notes (towards end), and for local formula of lie bracket}

\subsubsection{Classical Curvature form}
Given a connection $\nabla$ on $E$, define the \defn{curvature}
\begin{align*}
R^\nabla : & \Gamma(TM)\times \Gamma(TM)\times \Gamma(E)\to \Gamma(E) \\
& (X, Y, s) \mapsto \nabla_X(\nabla_Y s) - \nabla_Y(\nabla_X s) - \nabla_{[X, Y]} s
\end{align*}
which one can show is $\smooth (M)$-trilinear.

A few useful identities in relation to curvature (which follow from direct computation) are
$$R^\nabla (X, Y, s)=-R^\nabla (Y, X, s)$$
\finish{WRITE more identities useful}

\subsubsection{Curvature of the trivial line bundle}
Consider the trivial line bundle $E=M\times \complexnos$. Recall that $\nabla:=d:\smooth(M)\to \Omega^1(M)$ is a connection on this bundle, where $\nabla_X f = \nabla f(X) = df(X) = Xf$, for $f\in \smooth(M)$. 
So for all $s\in \Gamma(M\times \complexnos)\simeq \smooth(M)$, we have
\begin{align*}
    R^\nabla(X, Y, s) & = \nabla_X \nabla_Y s - \nabla_Y \nabla_X s - \nabla_{[X, Y]}s \\
    & = XYs - YXs - [X, Y]s \\
    & = 0
\end{align*}
Since the curvature is identically zero, we say that this connection on the trivial bundle is a \defn{flat connection}.

\subsubsection{Curvature of Levi Civita connection on $T\realnos^n$}
On $\realnos^n$ we have global coordinates $(x^1, \ldots, x^n)$. 
Let
$X=\sum_i X^i \frac{\partial}{\partial x^i}$, $Y=\sum_i Y^i \frac{\partial}{\partial x^i}$, $Z=\sum_i Z^i \frac{\partial}{\partial x^i}$ be vector fields on $T\realnos^n$. 
Recall we define the Levi-Civita connection on $T\realnos^n$ as
$\nabla_X Y = \sum_i X(Y^i) \frac{\partial}{\partial x^i}$. Then 
\begin{align*}
      R^\nabla(X, Y, Z) & = \nabla_X \nabla_Y Z - \nabla_Y \nabla_X Z - \nabla_{[X, Y]} Z \\
      & = \sum_i (X(Y(Z^i))-Y(X(Z^i))-[X, Y](Z^i))\frac{\partial}{\partial x^i} \\
      & = 0
\end{align*}
Hence the Levi-Civita connection on $T\realnos^n$ is flat. 

\subsubsection{Covarient derivative of differential forms with vector coefficients}
We define the \defn{covarient derivative} $D:\Omega^k(E, M)\to \Omega^{k+1}(E, M)$ such that for $\alpha\in \Omega^k(M)$, $s\in \Gamma(E, U)$,
$$D(\alpha \otimes s)=d\alpha \otimes s + (-1)^k \alpha \wedge Ds$$

We note that for any $f\in \smooth(U)$,
$$D(\alpha \otimes fs) = D (f\alpha \otimes s)$$

Furthermore we obtain a general Liebniz rule
$$D(\alpha \wedge \gamma)=d\alpha \wedge \gamma + (-1)^k \alpha \wedge D\gamma$$
for $\alpha \in \Omega^k(M)$, $\gamma\in \Omega^l(E, M)$. 

\subsubsection{General curvature form}
Given a connection $\nabla$, define the \defn{curvature form} as
$$F_\nabla := D^2 = D\circ D: \Omega^0(E, M)\to \Omega^1(E, M) \to \Omega^2(E, M)$$
where $D:\Omega^k(E, M)\to \Omega^{k+1}(E, M)$ is the covarient derivative. 

One can show that $F_\nabla$ is $\smooth(M)$-linear and so note $F_\nabla \in \Omega^2(\End(E), M)$. 

One can show that the general curvature form $F_\nabla$ corresponds with the classical curvature form $R_\nabla$, defined previously.
\understandBetter{WRITE in more detail}

\subsubsection{Example: Curvature of connections on the trivial bundle}
Recall that $d$ is the trivial connection on the trivial bundle $M\times \complexnos^r$. Then the curvature associated with this connection is
$F_d = d^2 = 0$.

Note that any other connection on $M\times \complexnos^r$, is of the form $\nabla = d+A$ (globally) for $A\in M_r(\Omega^1(M))$. Hence (by applying the discussion on local expressions below on $M\times \complexnos^r$), the curvature form is (globally) given by $F_\nabla = dA+A\wedge A$. 

\subsubsection{Curvature form local expression}
Let $A\in M_r(\Omega^1(U))$ be the connection matrix associated to a connection $\nabla$, over a local frame $f=(f^1, \ldots , f^r)$. In other words say $\nabla=d+A$ locally over $f$. Then locally, the curvature form $F_\nabla$ is given by a matrix $\theta \in \checkCorrect{M_r(\Omega^2(M))}$ such that
$$\mathbb{\theta}_{ij} := dA_{ij} + \sum_k A_{ik}\wedge A_{kj}$$
or in short
$$\mathbb{\theta} := dA + A\wedge A$$
\understandBetter{how is $A\wedge A$ defined/computed?} To prove this we note that $F_\nabla = \nabla^2 = \nabla \circ \nabla$ and so locally for $s\in \Gamma(E, U)=\Gamma(U\times \complexnos^r)\simeq \smooth(U)^r$
\begin{align*}
    (d+A)(d+A)v & = d^2 v + A dv + d(Av) + (A\wedge A) v \\
    & = A dv + (dA) v - A  dv + (A\wedge A) v \\
    & = (dA) v  + (A\wedge A) v \\
    & = (dA + A\wedge A) v
\end{align*}



Note that if $g:U\to GL_r(\complexnos)$ is a change of frame, so that $f'=fg$ is a new frame, we have the following transformation law
$$\mathfrak{\theta}(fg)=g^{-1} \mathbb{\theta}g$$

\subsubsection{Bianchi Identity}
\finish{WRITE UP BIANCHI IDENTITY}

Note that for the Chern connection on a holomorphic hermitian bundle, the Bianchi identity allows us to define a cohomology theory with cohomology classes $[F_\nabla]$ belonging to a Dolbeault cohomology class. One can further show that this does not depend on the choice of connection, i.e. $[F_\nabla]=[F_{\nabla+a}]$, where $\nabla, \nabla+a$ are Chern connections with respect to the hermitian metric. 

\subsubsection{Curvature of induced bundles}
Let $E_1, E_2$ be vector bundles with connections $\nabla^1, \nabla^2$ respectively, and associated curvature forms $F_{\nabla^1}, F_{\nabla^2}$. We can define curvature forms of various bundles in terms of the curvature of these bundles. 

Namely, on $E_1\oplus E_2$ we get a  curvature form
$$F = F_{\nabla^1}\oplus F_{\nabla^2}$$

On $E_1\otimes E_2$ we get the curvature form
$F=F_{\nabla^1}\otimes 1 + 1\otimes F_{\nabla^2}$

On $E^*$ (with induced connection $\nabla^*$) we get
$$F_{\nabla^*}=-F_{\nabla}^T$$

On the pull back connection $f^*\nabla$ \finish{[HAVE I DEFINED THIS PROPERLY BEFORE?]} we get
$$F_{f^* \nabla} = f^* F_\nabla$$
(where $f:M\to N$ map of smooth manifolds).

\subsubsection{Curvature of trivial bundle with constant hermitian structure}
We state without proof that if $\nabla$ is a hermitian connection on a hermitian vector bundle $(E, h)$, then $F_\nabla$ satisfies $h(F_\nabla s, \sigma)+h(s, F_\nabla \sigma)=0$. In other words $F_\nabla\in \Omega^2(\understandBetter{\End(E,h)}, M)$.

Consider the trivial bundle $E=M\times \complexnos^r$, with constant hermitian structure \checkCorrect{$h_p(s_p, \sigma_p)=1$ for all $p\in M$, for $s, \sigma\in \Gamma(M\times \complexnos^r)$}. \checkCorrect{Then we have a hermitian connection is $\nabla=d+A$ with $\bar{A}^T=-A$}. Then the \checkCorrect{curvature} is given by $d(A+\bar{A}^T)+(A\wedge A +\overline{(A\wedge A)}^T)=0$ by the result mentioned above. Hence in particular when $r=1$, the real part of $A$
 is constant. 
 \checkCorrect{[DO NOT UNDERSTAND. UNDERSTAND, CHECK CORRCET AND REWRITE]}

\subsubsection{Example: Cuvature associated with Chern connection on $T (\complexnos P^n)$}

Recall that for a holomorphic bundle $E$ with hermitian structure $h$, and Chern connection $\nabla$, the associated curvature $F_\nabla$ is a $(1,1)$-type, real, skew-hermitian form. In other words $F_\nabla\in \Omega^{1,1}_\realnos (\End(E), M)$.

Recall the Chern connection on $\complexnos P^n$ with hermitian structure $H$ given by the fubini-study metric, is given locally by $\nabla = d+\bar{H}^{-1}\partial(\bar{H})$. Let $A=\bar{H}^{-1}\partial(\bar{H})$, then the curvature associated to this Chern connection is locally 
\begin{align*}
    F_\nabla & = dA + A\wedge A \\
    & = \bar{\partial} A + \partial A + A\wedge A \\
    & = \bar{\partial}(\bar{H}^{-1}\partial(\bar{H})) + \partial(\bar{H}^{-1}\partial(\bar{H})) + \bar{H}^{-1}\partial(\bar{H}) \wedge \bar{H}^{-1}\partial(\bar{H}) \\
    & = \bar{\partial}(\bar{H}^{-1}\partial(\bar{H}))
\end{align*}
where the last equality holds, by comparing types. More precisely recall that $A$ is \checkCorrect{a matrix of} $(1, 0)$ forms, so $\partial A$ is a $(2,0)$-form, and $A\wedge A$ is a $(2,0)$-form, and so are zero as curvature is a $(1,1)$ form.

\subsubsection{Line bundle on $\complexnos P^n$}
\understandBetter{[SEE EXAMPLE 4.3.12 HUYBRECHTS, and understand]}
Let $(z^0, \ldots, z^n) \checkCorrect{\sim (1: w^1: \ldots : w^n)}$ standard coordinates on $\complexnos P^n$ \understandBetter{as sections of the holomorphic line bundle}. Recall we have a hermitian metric \checkCorrect{$h=(\sum_i |z^i|^2)^{-1}=(\sum_i 1+|w^i|^2)^{-1}$} on this bundle. Then one can show that the associated curvature to the Chern connection $\nabla$ on this bundle is
$$F_\nabla = \frac{2\pi}{i} \omega_{FS} $$
where $\omega_{FS}$ is the Fubini-study form. Locally we get 
$$F_\nabla = \bar{\partial}\partial \text{log}(h) = \bar{\partial}\partial \text{log}(\sum_i 1+|w^i|^2)$$

Note that this can be generalised to any holomorphic line bundle on any complex manifold $M$ using the pull-back of $\phi:M\to \complexnos P^n$ to give
$$\frac{i}{2\pi}F_\nabla = \phi^* \omega_{FS}$$
where $\omega_{FS}$ is the Fubini-study form on $\complexnos P^n$.

\understandBetter{[UNDERSTAND BETTER AND REWRITE!]}

\section{De Rham Cohomology}
\finish{WRITE DEFINITION of de rham cohom - see Nakahara}

$H^*_{dR}(M) = \bigoplus_i^n H^i_{dR}(M)$ is a graded $\realnos$-algebra. (Recall that $H^m_{dR}(M)=0$ if $m>\text{dim}_\realnos M$). 

Recall if $\omega \in H^{i}_{dR}(M)$, $\eta \in H^{j}_{dR}(M)$, then $\omega \eta = (-1)^{ij} \eta \omega$.

\subsubsection{Pioncare duality}

\subsubsection{Cohomology of $\complexnos P^n$}

\section{Chern classes}
\finish{Introduce. There are a few different notions of characteristic classes....[GIVE SMALL INTRO TO CHAR CLASSES AND WHERE CHERN CLASSES FIT IN] and what they are geometrically}

\understandBetter{Note that by de Rham theorem $H^*(M, \realnos)\simeq H^*_{dR}(M)$} and we are essentially \checkCorrect{interested in $H^*(M, \mathbb{Z})\simeq H^*(M, \realnos)$ -- TRUE??}
Let $E\to M$ be a (complex) vector bundle of rank $k$ (over $\complexnos$). The $i$'th chern class is an element of de Rham cohomology
$c^i(E)\in H^{2i}_{dR}(M)$

\subsubsection{Invarient polynomials}

A $k$-multi-linear form $$\tilde{\phi}:\underbrace{M_r(\complexnos)\times \cdots \times M_r(\complexnos)}_{k \text{ times}}\to \complexnos$$
is \defn{invarient} if for any $g\in GL_r(\complexnos)$
$$\tilde{\phi}(A_1, \ldots, A_k) = \tilde{\phi}(gA_1g^{-1}, \ldots, gA_kg^{-1})$$
We write $\tilde{I}_k(M_r(\complexnos))$ for the $\complexnos$-vector space of all \defn{invarient $k$-multilinear forms} on $M_r(\complexnos)$. 

Recall a polynomial is said to be \defn{homogeneous} if all its monomials are of the same degree (for example $x^5 + 7xy^4 + x^3y^2$ is homogeneous of degree 5). Now $\tilde{\phi}\in \tilde{I}_k(M_r(\complexnos))$ induces a homogeneous polynomial $\phi$ of degree $k$ (with entries in $A$) given by $\phi:M_r(\complexnos)\to \complexnos$ such that
$$\phi(A):=\tilde{\phi}(A, A, \ldots, A)$$
and we note then that $\phi(gAg^{-1})=\phi(A)$ (for $g\in GL_r(\complexnos)$), hence we call $\phi$ an \defn{invarient polynomial}. Note that $\phi$ is called the \defn{polarized} form of $\tilde{\phi}$. We write $I_k(M_r(\complexnos))$ for the \defn{space of invarient polynomials of degree $k$}. 

Converesly, given an invarient polynomial $\phi \in I_k(M_r(\complexnos))$, we can get an invarient k-multilinear form $\tilde{\phi} \in \tilde{I}_k(M_r(\complexnos))$, by defining
$$\tilde{\phi}(A, \ldots, A):=\phi(A)$$
\understandBetter{(which is a restriction to the diagonal). [CHECK CORRECT AND REWRITE COMMENT]}

An example of a (degree $r$) invarient polynomial is the determinant function
$$\text{det}:M_r(\complexnos)\to \complexnos \in I_r(M_r(\complexnos))$$
and furthermore, the map
$$\text{det}(I+A)=\sum_{k=0}^r \gamma_k(A) $$
give invarient polynomials $\gamma_k\in I_k(M_r(\complexnos))$, where $I$ is the identity, $A \in M_r(\complexnos)$.

\subsubsection{Chern-Weil theorem}

\understandBetter{UNDERSTAND BETTER BELOW
DEF OF INVARIENT POLY EXTENDED TO $\Omega^p(End(E), M)$.}
Note that the Lie algebra $\mathfrak{gl}_r(\complexnos)\simeq M_r(\complexnos)$. Recall that a $k$-multilinear symmetric map $\tilde{P}:V\times \cdots \times V\to \complexnos$ over a vector space $V$ corresponds to an element $P\in S^k(V)^*$. Now such a symmetric $k$-multilinear map $\tilde{P}: \mathfrak{gl}_r(\complexnos)\times \cdots \times \mathfrak{gl}_r(\complexnos)\to \complexnos$ is invarient if and only if 
$$\sum_j P(A_1, \ldots, A_j, [A, A_j], A_{j+1}, \ldots, A_k)=0$$
(where $[A, A_j]$ is the Lie bracket in $\mathfrak{gl}_r(\complexnos)$). 

Now given an invarient symmetric multilinear map $\tilde{P}$, it induces a $k$-linear map
\begin{align*}
    \tilde{P} : & \left( \bigwedge {}^{i_1}M\otimes \End(E) \right) \times \ldots \times \left( \bigwedge {}^{i_k}M\otimes \End(E) \right)  \to \bigwedge {}^m_\complexnos M \\
    & (\alpha_1\otimes \gamma_1, \ldots, \alpha_k\otimes \gamma_k) \mapsto (\alpha_1\wedge \cdots \wedge \alpha_k)P(\gamma_1. \ldots, \gamma_k)
\end{align*}
where $i_1+\ldots +i_k=m$ is an arbitrary partition and $E$ any vector bundle. On the level of global sections this becomes a $k$-multilinear map 
$$\tilde{P}:\Omega^{i_1}(M, \End(E))\times \ldots \times \Omega^{i_k}(M, \End(E))\to \Omega^m_\complexnos(M)$$
If we restrict $\tilde{P}$ to only even froms, it stays a $k$-multilinear symmetric map, in particular if we restrict $\tilde{P}$ to $\Omega^{2}(M, \End(E))\times \ldots \times \Omega^{2}(M, \End(E))$
then it can be recovered from the polarized form $P(\alpha\otimes \gamma)=\tilde{P}(\alpha\otimes\gamma, \ldots, \alpha\otimes\gamma)$.

If $\nabla$ is a connection on $E$, and $\tilde{\nabla}$ the induced connection on $\End(E)$, then 
$$dP(\gamma_1, \ldots, \gamma_k)=\sum_{j=1}^k (-1)^{2(j-1)}P(\gamma_1, \ldots, \checkCorrect{\tilde{\nabla}}(\gamma_j), \ldots, \gamma_k)$$
for forms $\gamma_i\in \Omega^2(M, \End(E))$. \checkCorrect{DO I NEED THIS IN FULL GENERALITY?--IN HUYBRECTHS, OR IS DEG 2 FORMS ENOUGH?}
\understandBetter{[CHECK AND UNDERSTAND PREVIOUS PARAS BETTER]}


Let $P$ be an invarient k-multilinear symmetric polynomial on $\mathfrak{gl}(r, \complexnos)$. Then one can show the induced $2k$-form 
$$\tilde{P}(F_\nabla)\in \Omega^{2k}_\complexnos(M)$$
is closed. Hence we can define cohomology classes
$$[\tilde{P}(F_\nabla)]\in H^{2k}(M, \complexnos)$$
which one can show is independent of the chosen connection $\nabla$, i.e.
$$[\tilde{P}(F_\nabla)]=[\tilde{P}(F_{\tilde{\nabla}} )]$$
for any connections $\nabla, \tilde{\nabla}$. 

\subsubsection{Chern-Weil homomorphism}
\finish{[find ref. wells?/huybrechts?/tu]}


\subsubsection{Definition of Chern class}
\finish{see your notes}
We now consider specific invarient polynomials, to define Chern classes and other useful topological invarients. 

We define $\{\tilde{P}_k\}$ to be the homogenous polynomial of $\text{deg}(\tilde{P}_k)=k$ defined by 
$$det(I+B)=1+\tilde{P}_k(B)+\ldots +\tilde{P}_r(B)$$
where $I$ is the identity. \question{What is B? What is r?}

We then define the \defn{Chern forms} of a \checkCorrect{rank $r$} vector bundle $E$ endowed with a connection $\nabla$ as
$$c_k(E, \nabla):=\tilde{P}_k \left( \frac{i}{2\pi} F_\nabla \right) \in \Omega^{2k}_\complexnos (M)$$

The $k$'th \defn{Chern class} of $E$ is then the induced cohomology class of the respective Chern form, i.e. 
$$c_k(E):= [c_k(E, \nabla)] \in H^{2k}(M, \complexnos)$$

One can note that $c_0(E)=1$ and $c_k(E)=0$ for all $k>\text{rank}(E)$. 

We define the \defn{total Chern class} as 
\begin{align*}
c(E)& :=c_0(E)+c_1(E)+\ldots +c_r(E) \\
& = \text{det} \left( I + \frac{i}{2\pi} F_\nabla \right)
\end{align*}
noting that $c_i(E)\in H^{2i}(M, \complexnos)$.

\finish{WE CAN PROVE VARIOUS PROPERTIES OF CHERN CLASSES}

\subsubsection{Computing Chern classes}
To make it less cumbersome to expand the determinant and compute Chern classes, we find formulas in terms of the traces for the $k$-th Chern class. 

We start by diagonalising the curvature form $F_\nabla$ by finding a matrix $g\in GL_r(\complexnos)$ such that
$$\Theta := g^{-1}\left( \frac{i}{2\pi} F_\nabla \right) g = \text{diag}(\omega_1, \ldots, \omega_r)$$
for some $2$-forms $\omega_i$. 

\understandBetter{For example, if we choose an anti-Hermitian generator $g\in SU(r)\subset GL_r(\complexnos)$,} then we have
\begin{align*}
    \det(I+\Theta) &= \det (\text{diag}(1+\omega_1, \ldots, 1+\omega_r)) \\
    & = \prod_{i=1}^r (1+\omega_i) \\
    & = 1 + (\omega_1 + \ldots + \omega_r) + (\omega_1\omega_2 +\ldots + \omega_{r-1}\omega_r) + \ldots (\omega_1\omega_2\cdots \omega_r) \\
    & = 1 + \tr(\Theta) + \frac{1}{2}(\tr(\Theta)^2 - \tr(\Theta^2)) + \ldots + \det (\Theta) 
\end{align*}
and note that each summand is an elementery symmetric function of the $\omega_i$.
Now since $\det(I+\Theta)$ is an invarient polynomial we have $\det(I+\Theta)=\det(I+g\Theta g^{-1})=\det(I+g (g^{-1} (\frac{i}{2\pi} F_\nabla) g) g^{-1}) = \det (I+ \frac{i}{2\pi} F_\nabla)$. 
Or in other words for the general curvature $F_\nabla$, we have 
$$\det(I+F_\nabla)=\det(I+\frac{2\pi}{i}\Theta)$$
$$\tr(F_\nabla)= \tr( \frac{2\pi}{i} \Theta)= \checkCorrect{\frac{2\pi}{i} \tr(\Theta)}$$
This then gives us the Chern classes with respect to the curvature form $F_\nabla$ as
\begin{align*}
    c_0 & = 1 \\
    c_1 & = \tr(\Theta) = \frac{i}{2\pi} \tr(F_\nabla)
    \\
     c_2 & = \frac{1}{2}(\tr(\Theta)^2 - \tr(\Theta^2))  = \frac{1}{2} \understandBetter{\left( \frac{i}{2\pi} \right)^2 (\tr(F_\nabla)\wedge\tr(F_\nabla) - \tr(F_\nabla\wedge F_\nabla))} \\
    \vdots & \\
     c_r & = \det(\Theta)=\left( \frac{i}{2\pi} \right)^r \det(F_\nabla) 
\end{align*}


\subsubsection{Properties of Chern forms}

\finish{SEE WELLS. COMBINE WITH NEXT SECTION}

One can show that in fact a Chern form $c(E, \nabla)$ is a real differentiable form, from which it follows the Chern class $c(E)\in H^*(X, \realnos)\subseteq H^*(X, \complexnos)$.  Now note that the de Rham group $H^*(X, \realnos)$ on a smooth manifold has a ring structure given by 
$$c\cdot c' = [\omega \wedge \omega'] $$
for $c=[\omega], c'=[\omega '] \in H^*(X, \realnos)$.

\question{IS this the same as the product in de rham cohomology?}

\subsubsection{Axiomatic definition of Chern classes}
Denote the set of (complex) vector bundles $E\to M$ of $\complexnos$-rank $k$, over a fixed smooth manifold $M$ as $\mathfrak{E}$. Define
\begin{align*}
    c: \mathfrak{E} & \to H^*_{dR}(M) \\
    E & \mapsto c(E) = c_0(E)+c_1(E)+\ldots +c_k(E)
\end{align*}
where $c_i(E)\in H^{2i}_{dR}(M)$, with the following properties
\begin{enumerate}
    \item Naturality:
    
    Given two isomorphic bundles $E\simeq F$ over $M$, then $c(E)=c(F)$ (i.e. $c_i(E)=c_i(F) \forall i$)

    \item Triviality:

    If $E=M\times \complexnos^k$ is the trivial bundle then $c(E)=1$ (i.e. $c_0(E)=1\in H^0_{dR}(M)\simeq \realnos$ (if $M$ connected) and $c_1(E)=c_2(E)=\ldots =c_k(E)=0$). 

    \item Whitney Sum:

    If $E, F$ are two bundles over $M$ then
    $c(E\oplus F) = c(E) \cup c(F)$
    
    Note $\cup$ is multiplication in $H^*_{dR}(M)$, \checkCorrect{defined as}
    \begin{align*}
        c(E) \cup c(F) & = (c_0(E)+\ldots + c_{k}(E)) + (c_0(E)+\ldots + c_{s}(F)) \\
        & = (c_0(E) c_0(F)) + (c_0(E)\cup c_1(F) + c_1(E)\cup c_0(F)) + \ldots
    \end{align*}
    where in the last equality the first summand is degree 0, the second is degree 2, etc. 
    \understandBetter{Is this a recursive formula?}
    
    \item Functoriality

    Let $F:M\to N$ be a smooth map of manifolds, and $E\to N$ a vector bundle over $N$. Recall the pull back bundle $F^*E$ is a bundle over $M$, with fibres $(F^*E)_p=E_{F(p)}$ for $p\in M$. 
    
    Then functoriality says that 
    $c(F^* E) = F^*(c(E)) \in H^*_{dR}(M)$, (i.e. $c_i(F^* E) = F^*(c_i(E)) \in H^{2i}_{dR}(M)$ for all $i$).
    
    \item Normalisation (Euler)
    \finish{WRITE UP}

    \item 
    \checkCorrect{More properties??- check}

    \item \checkCorrect{Is THIS AN axiom or just a property?} Duality:

    If $E^*$ is the dual bundle of $E$ then $c_i(E^*)=(-1)^ic_i(E)$

\end{enumerate}

As an application of the naturality and triviality axioms, we observe that if $\forall i\geq 1, c^i(E)\neq 0$ (and so $c(E)=\neq 1$), then the bundle $E$ is not trivial. 

\subsubsection{Chern classes on Induced bundles}
[WRITE UP from huybrechts]

\subsubsection{Chern characters}
We now use a different invarient homogeneous polynomial to define Chern characters. 

Define the invarient polynomials $\{\tilde{Q}_k\}$ given by 
$$tr(e^B)=\tilde{Q}_0(B)+\tilde{Q}_1(B)+\tilde{Q}_1(B)+\ldots $$
Then we define the $2k$-forms 
$$ch_k(E,\nabla):=\tilde{Q}_k \left( \frac{i}{2\pi}F_\nabla \right) \in \Omega^{2k}_\complexnos (M)$$.
The $k$'th \defn{Chern character} is then defined as the cohomology classes of these forms, namely
$$ch_k(E):= [ch_k(E, \nabla)]\in H^{2k}(M, \complexnos)$$

We note that $ch_0(E)=\text{rank}(E)$.

Finally we define the \defn{total Chern character} as 
$$ch(E):=ch_0(E)+ch_1(E)+ch_2(E)+\ldots$$

\subsubsection{Example: Chern class of tangent bundle to $\complexnos P^1$}
\checkCorrect{UNDERSTAND THIS EXAMPLE BETTER AND REWRITE. WELLS EX 3.8, PG 96}

\understandBetter{Note that the tangent bundle $T(\complexnos P^1)$ is $\realnos$-linear isomophic to the real tangent bundle $T(S^2)$.} Given coordinates $z$ on $\complexnos P^1$, define a natural metric on $T(\complexnos P^1)$ as
$$h(z):= h(\partial_z, \partial_z) := \frac{1}{(1+|z|^2)^2}$$
Note that if we view this in $T(S^2)$ with coordinates $w=\frac{1}{z}$ \checkCorrect{(using stereographic projection from infinity)} then we get the same form 
$$\checkCorrect{h(\partial_w, \partial_w)=\frac{1}{(1+|w|^2)^2}}$$

We now have that locally the connection matrix is
\begin{align*}
A(z) & =h(z)^{-1}\partial h(z) \\
& = (1+|z|^2)^2 \partial \left( \frac{1}{(1+|z|^2)^2} \right) \\
& = (1+|z|^2)^2 \frac{\partial}{\partial z} \left( \frac{1}{(1+z\bar{z})^2} \right)dz \\
& = (1+|z|^2)^2 \left( -2 (1+z\bar{z})^{-3} \bar{z} \right)dz \\
& = -\frac{2\bar{z}}{1+|z|^2} dz
\end{align*}
\understandBetter{CHECK IF COMPUTATION IS CORRECT!}

The associated curvature form is then locally
$$\theta = \bar{\partial} A = \frac{2}{(1+|z|^2)^2} dz \wedge d\bar{z}$$
\understandBetter{WHY?? understand and fill details}

Hence the first Chern form is given by 
$$c_1(E, \nabla) = \frac{i}{\pi(1+|z|^2)^2} dz \wedge d\bar{z} = \frac{2dx\wedge dy}{\pi(1+|z|^2)^2}$$
\understandBetter{[WHY IS THIS THE FIRST CHERN FORM?]}

\understandBetter{WHY IS THIS THE CHERN CLASS?}
\begin{align*}
    \checkCorrect{c_1(E)} & =\int_{\complexnos P^1} c_1(E, \nabla) \\
    & = \checkCorrect{ \frac{2}{\pi} \int_{-\infty}^\infty \int_{-\infty}^\infty \frac{dx dy}{(1+||(x,y)||^2)^2}} \\
    & = \frac{2}{\pi} \int_{0}^\infty \int_{0}^{2\pi} \frac{\rho}{(1+\rho^2)^2} d\rho d\theta \\
    & = 4 \int_{0}^\infty \frac{\rho}{(1+\rho^2)^2} d\rho  \\
    & = 2 \int_{1}^\infty \frac{du}{u^2}  \\
    & = 2
\end{align*}
[CHECK COMPUTATION. WHY IS THIS FIRST CHERN CLASS IF IT IS? ]
WHAT ABOUT OTHER CHERN CLASSES? -- SEE WELLS FOR REMAINING DISCSUSSION

\checkCorrect{Using the formula for the first Chern class in terms of the trace, we find that
$$c_1(E) = \frac{i}{2\pi} \tr (\theta) = \frac{i}{2\pi} \frac{4}{1+|z|^2}$$
[I AM CERTAIN THIS IS WRONG!] I THINK IT SHOULD BE 2.}

\subsubsection{Example: Chern class of $\complexnos P^n$}
\finish{[LOOK UP and write up!!!]}

\section{Conclusion}
\finish{WRITE UP}

\section*{References}
\begin{enumerate}
\item Wells
\item Huybrechts
\item Nakahara
\item Tu
\item J.M.Lee; \textit{Introduction to Smooth Manifolds, 2nd ed.} Springer (2012)
\item I.Madsen, J.Tornehave; \textit{From Calculus to Cohomology, De Rham Cohomology and characteristic classes.} Cambridge University Press (1997)
\item P.B.Gilkey, R.Ivanova, S.Nik{\u c}evi\'c; \textit{Characteristic classes.} Elsevier Ltd (2006)
\end{enumerate}
\end{document}
