\documentclass[a4paper]{article}

\usepackage{amsthm, amsmath, latexsym, amssymb}
\usepackage{mathtools}
\usepackage{bbm}
\usepackage{stmaryrd}
\usepackage{verbatim}
\usepackage[dvipsnames]{xcolor}

\setcounter{secnumdepth}{1}

\theoremstyle{definition} \newtheorem*{definition}{Definition}
\theoremstyle{definition} \newtheorem*{definitions}{Definitions}
\theoremstyle{plain} \newtheorem{theorem}{Theorem}[section]
\theoremstyle{plain} \newtheorem{proposition}[theorem]{Proposition}
\theoremstyle{plain} \newtheorem{corollary}[theorem]{Corollary}
\theoremstyle{plain} \newtheorem{lemma}[theorem]{Lemma}
\theoremstyle{plain} \newtheorem{example}[theorem]{Example}

\newcommand{\checkCorrect}[1]{\textcolor{red}{#1}}
\newcommand{\understandBetter}[1]{\textcolor{orange}{#1}}
\newcommand{\question}[1]{\textcolor{orange}{#1}}
\newcommand{\explainFurther}[1]{\textcolor{blue}{#1}}
\newcommand{\finish}[1]{\textcolor{green}{#1}}

\newcommand{\defn}[1]{\textbf{#1}}
\newcommand{\realnos}{\mathbb{R}}
\newcommand{\complexnos}{\mathbb{C}}
\newcommand{\canonicaliso}{\cong}
\newcommand{\id}{\mathtt{id}}
\newcommand{\smoothCmaps}{C^\infty_\complexnos (U)}
\newcommand{\Hom}{\text{Hom}}
\newcommand{\End}{\text{End}}
\newcommand{\tr}{\text{tr}}
\newcommand{\smooth}{C^\infty}
\newcommand{\projCspace}{\complexnos P^n}


\begin{document}
\title{Chern Classes}
\author{Vatsal Kanoria}
\date{2023}
\maketitle

\begin{abstract}
A literature review on the theory of Chern classes from the perspective of differential geometry. We study in particular the example of complex projective space in detail. 
\end{abstract}

\tableofcontents

\section{Introduction}

We are interested in learning about topological invariants of vector bundles on manifolds. Namely those properties that are the `same' for isomorphic vector bundles. A particular type of global invariant on vector bundles is known as Characteristic Classes. Loosely speaking, characteristic classes measure the difference between a global product structure and a local product structure on a manifold, and are the primary obstruction to being able to admit global independent sections on a bundle. 

There are four different types of characteristic classes: Stiefel-Whitney classes, Pontryagin classes, Euler classes and Chern classes. Stiefel-Whitney and Pontryagin classes are characteristic classes of real vector bundles. Euler classes  characterise oriented real vector bundles.
Chern classes characterise complex vector bundles. 
Note that one can consider Chern classes of complex vector bundles over smooth (real) manifolds, and also of K{\"a}hler manifolds (the latter of which we do not discuss in detail). Note further that the theory of Chern classes can be applied equally if we replace vector bundles with principle $G$-bundles (with fibres the Lie group $G$), since these structures are `equivalent'. 

The theory of characteristic classes is a cohomology theory, and thus requires techniques from algebraic topology to compute these in general. However in the case of Chern classes of complex vector bundles, one can use techniques from complex differential geometry to compute these. This makes computing Chern classes comparatively more explicit than computing the other characteristic classes. In particular the Chern classes (which are elements in cohomology classes) can be defined in terms of an invariant polynomial of its curvature form. The curvature of a manifold is associated to a connection on a vector bundle. We will find that in fact the choice of connection does not impact the Chern class, so it is indeed an invariant.

Hence we spend a large portion of the thesis  reproducing important facts and examples from the theory of manifolds, vector bundles, connections on manifolds and the associated curvature. We then go on to define Chern classes and their properties. We compute all of the theory with a focus on complex projective space, whilst alluding to various other simple examples throughout the exposition. We also do not hesitate to recall theory relating to smooth and Riemannian manifolds, due to their familiarity and similarity in many respects to the respective theory on complex manifolds.

Complex projective space happens to be a particularly useful example to study in the theory of Chern classes, since it has many interesting complex manifolds that can be embedded into projective space and it also has the advantage of being able to apply linear algebraic methods on this space. 

The theory of characteristic classes have applications for example in topological quantum computing. Chern classes are an example of `primary' characteristic class. One may also define `secondary' characteristic classes (which arise when the first Chern class vanishes) leading to what is known as Chern-Simons theory. Chern-Simons theory is a topological quantum field theory. In the simplest case, $SU(2)$-Chern-Simons theory (which is known to physicists as the Yang-Lee-Fibonnaci model) gives rise to non-abelian anyons describing a model for topological quantum computing. The main theoretical advantage of topological quantum computing over other quantum computing models is in physical error correction. A formal treatment of Chern-Simons theory is out of the scope of this exposition, however it is defined in terms of the Chern-Weil homomorphism which we do discuss.

\section{Manifolds}

\subsubsection{Smooth manifolds}

A (smooth) $n$-dimensional \defn{manifold} $M$ is a second countable, Hausdorff space with a smooth maximal atlas. An \defn{atlas} consists of a collection of charts, that is, an open cover $U_\alpha$ of $M$ ($\alpha\in I$, $I$ index set)  and homeomorphims $\phi_\alpha:U_\alpha \to \phi_\alpha(U_\alpha) \subseteq \realnos^n$, such that the \defn{transition maps} $\phi_\alpha \circ \phi_\beta^{-1}$ are smooth (on $\phi_\beta(U_\alpha \cap U_\beta)\subseteq k^n$) for all $\alpha, \beta\in I$. 

Note that every atlas uniquely gives rise to a maximal atlas, so it will generally be sufficient to define any convenient atlas on a given manifold. Recall also the non-trivial fact that the dimension of a manifold is a topological invariant (in other words given two charts $\phi$, $\tilde{\phi}$ of $M$ with $\mathtt{im}(\phi)\subseteq \realnos^m$, $\mathtt{im}(\tilde{\phi})\subseteq \realnos^n$, we have $m=n$). It is also easy to show that the coordinate maps $\phi_\alpha$ are not only homeomorphisms but also diffeomorphisms. 

A manifold has a \defn{boundary} if some of its charts are \defn{boundary charts}. That is, a boundary chart $\phi:U\to \phi(U)\subseteq \mathbb{H}^n$ takes values in the closed upper half space $\mathbb{H}^n := \{(x_1, \ldots, x_n): x_i\geq 0, \forall i\}$ (so that $\partial \mathbb{H}^n \cap \phi(U)$ is non-empty, where $\partial \mathbb{H}^n := \{(x_1, \ldots, x_n): x_n=0\}$). Note that $\mathbb{H}^0 = \{0\}$ and $\partial \mathbb{H}^0 = \{\}$ by definition. The remaining charts $\phi:U\to \phi(U)\subseteq \realnos^n$ are called \defn{interior charts}. The \defn{interior} of a manifold $M$ are the points that come from an interior chart, $\texttt{Int} M := \{p\in M: \exists \textrm{interior chart } \phi:U\to \phi(U)\subseteq \realnos^n, p\in U \}$. The \defn{boundary} of $M$ are the points that come from a boundary chart, $\partial M:=\{p\in M: \exists \textrm{boundary chart } \phi:U\to \phi(U)\subseteq \mathbb{H}^n, p\in U \textrm{ and } \phi(p)\in \partial \mathbb{H}^n \}$. Confusingly, the boundary of a manifold is not the same as its topological boundary in general. It is non-trivial to show that $M$ can be decomposed into a disjoint union of its interior and boundary, $M = \mathtt{Int} \ M \mathbin{\dot{\cup}} \partial M$, hence every point of $M$ is either an interior point or exclusively, a boundary point.

\subsubsection{Example: An atlas on $S^1$}

On the circle $S^1:=\{(x, y): x^2+y^2 = 1\}\subseteq \realnos^2$, we can define an atlas $\mathfrak{A}_1 :=\cup_{i=1}^4 U_i$ given by 
\begin{align*}
&& U_1 = S^1 \cap \{(x, y): x>0 \} \\
\phi_1 & : U_1\to \realnos &
\phi_1^{-1} & :(-1, 1)\to U_1 \\
& (x,y) \xmapsto{\phi_1} x 
& & x \xmapsto{\phi_1^{-1}} (x, \sqrt{1-x^2})
\end{align*}
\begin{align*}
&& U_2 = S^1 \cap \{(x, y): x<0 \} \\
\phi_2 & : U_2\to \realnos &
\phi_2^{-1} & :(-1, 1)\to U_2 \\
& (x,y) \xmapsto{\phi_2} x 
& & x \xmapsto{\phi_2^{-1}} (x, - \sqrt{1-x^2})
\end{align*}
\begin{align*}
&& U_3 = S^1 \cap \{(x, y): y>0 \} \\
\phi_3 & : U_3\to \realnos &
\phi_3^{-1} & :(-1, 1)\to U_3 \\
& (x,y) \xmapsto{\phi_3} y
& & y \xmapsto{\phi_3^{-1}} (\sqrt{1-y^2}, y)
\end{align*}
\begin{align*}
&& U_4 = S^1 \cap \{(x, y): y<0 \} \\
\phi_4 & : U_4\to \realnos &
\phi_4^{-1} & :(-1, 1)\to U_4 \\
& (x,y) \xmapsto{\phi_4} y
& & y \xmapsto{\phi_4^{-1}} (- \sqrt{1-y^2}, y)
\end{align*}
and with transition functions given by (noting that $\phi_i(U_i\cap U_j) = (0, 1) \text{ or } (-1, 0)$), 
\begin{align*}
\phi_3 \circ \phi_1^{-1} & :\phi_1(U_1\cap U_3)\to \phi_3(U_1 \cap U_3) \\
\phi_3 \circ \phi_1^{-1} & : (0,1) \to (0,1) \\
& x \xmapsto{\phi_1^{-1}} (x, \sqrt{1-x^2}) \xmapsto{\phi_3} \sqrt{1-x^2}
\end{align*}
\begin{align*}
\phi_1 \circ \phi_3^{-1} & : (0,1) \to (0,1) \\
& y \xmapsto{\phi_1 \circ \phi_3^{-1}}  \sqrt{1-y^2}
\end{align*}
\begin{align*}
\phi_1 \circ \phi_4^{-1} & : (-1,0) \to (0,1) \\
& y \mapsto -\sqrt{1-y^2}
\end{align*}
\begin{align*}
\phi_4 \circ \phi_1^{-1} & : (0,1) \to (-1, 0) \\
& x \mapsto \sqrt{1-x^2}
\end{align*}
\begin{align*}
\phi_2 \circ \phi_3^{-1} & : (0,1) \to (-1, 0) \\
& y \mapsto \sqrt{1-y^2}
\end{align*}
\begin{align*}
\phi_3 \circ \phi_2^{-1} & : (-1,0) \to (0,1)\\
& x \mapsto -\sqrt{1-x^2}
\end{align*}
\begin{align*}
\phi_2 \circ \phi_4^{-1} & : (-1,0) \to (-1,0)\\
& y \mapsto -\sqrt{1-y^2}
\end{align*}
\begin{align*}
\phi_4 \circ \phi_2^{-1} & : (-1,0) \to (-1,0)\\
& x \mapsto -\sqrt{1-x^2}
\end{align*}
which are all smooth maps, hence $S^1$ is a smooth $2$-dimensional manifold. This atlas can easily be generalised to the $n$-sphere $S^n$.

\subsubsection{Example: Real projective space}

\defn{Real projective space} $\realnos P^n := \frac{\realnos^{n+1}\setminus \{0\}}{\sim}$ is the quotient space of $\realnos^{n+1}\setminus \{0\}$ under the relation $x\sim y \iff \exists \lambda \in \realnos \setminus \{0\} : \lambda x=y$. We write the equivalence class of $(x_1, \ldots, x_{n+1})\in \realnos^{n+1}$ as $[x_1, \ldots, x_{n+1}]\in \realnos P^n$. 
In other words, under the quotient map $q:\realnos^{n+1}\setminus \{0\}\to \realnos P^n$, $[x_1, \ldots, x_{n+1}]:=q(x_1, \ldots, x_{n+1})$ and recall $U \subseteq \realnos P^n \textrm{ open} \iff q^{-1}(U) \subseteq \realnos^{n+1}\setminus \{0\} \textrm{ open}$, where $\realnos^{n+1}\setminus \{0\}$ is endowed with subspace topology of $\realnos^{n+1}$.) 
This is an $n$-dimensional smooth manifold with atlas $\{(U_i, \phi_i)\}_{i=1, \ldots, n+1}$ given by $U_i=\{[x_1, \ldots, x_{n+1}]: x_i\neq 0\}$ and $\phi_i([x_1, \ldots, x_{n+1}])=(\frac{x_1}{x_i}, \ldots , \frac{x_{i-1}}{x_i}, \frac{x_{i+1}}{x_i}, \ldots, \frac{x_{n+1}}{x_i})\in \realnos^n$. Note that $q^{-1}(U_i)^C=\realnos^n$ is a closed hyperplane in $\realnos^{n+1}\setminus \{0\}$, hence the complement $q^{-1}(U_i)$ is the union of two hyperspaces $(\realnos^{n+1})^+$,$(\realnos^{n+1})^-$, which is open in $\realnos^{n+1}\setminus \{0\}$, hence $U_i$ is open in $\realnos P^n$. One can also show $\phi_i$ are indeed homeomorphisms.  Note it is easy to see that the inverse of $\phi_i$ is given by $\phi_i^{-1}(a_1, \ldots, a_n)=[a_1, \ldots, a_{i-1}, 1, a_{i}, \ldots, a_n]$ (since $[a_1, \ldots, a_{i-1}, 1, a_{i}, \ldots, a_n]=[\lambda a_1, \ldots, \lambda a_{i-1}, \lambda, \lambda a_{i}, \ldots, \lambda a_n]$, and we can set $\lambda\in \realnos$ accordingly). It is also easy to show that the transition maps $\phi_i\circ \phi_j^{-1}$ are smooth. 

\subsubsection{Riemannian manifolds}
A \defn{Riemannian manifold} is a smooth manifold $M$ with a smooth map $g:p\mapsto g_p$, $p\in M$, called a \defn{Riemannian metric}, where $g_p:T_pM\times T_pM\to \realnos$ is an inner product (i.e.\ a symmetric, positive-definite, bilinear form).
Note that it is always possible to find a Riemannian metric on any smooth manifold.

Given a chart $(U, x^1, \ldots, x^n)$ of $M$, $g$ is smooth if and only if $g_{ij}:=g(\frac{\partial}{\partial x^i}, \frac{\partial}{\partial x^j}):U\to \realnos$ is smooth for all $i,j=1,\ldots, n$. Note $g_{ij}(p):=g_p(\frac{\partial}{\partial x^i}\vert_p, \frac{\partial}{\partial x^j}\vert_p) \in \realnos$ and we will make use of the shorthand notation $\partial_i:=\frac{\partial}{\partial x^i}$. We can now view $(g_{ij})$ as an $n^2$ matrix in $M_n(C^\infty (U))$, and since at each point $p$, $g_p$ is an inner product (so symmetric and positive definite) this also implies the matrix $(g_{ij})$ is symmetric and positive, and also in particular is non-degenerate, thus invertible.

Since $(g_{ij}(p))$ is an invertible matrix over $\realnos$, we write its inverse as $(g^{kl}(p))$, and recall matrix multiplication of a matrix with its inverse gives us the following \defn{`contraction'},
$\sum_j g_{ij}(p)g^{jk}(p)=\delta_i^k(p)$ (where $\delta_i^k\in C^\infty(U)$ such that $\delta_i^k(p)=1$ if $i=k$ or $0$ otherwise). In other words, as smooth functions over $U$ we have $\sum_j g_{ij}g^{jk}=\delta_i^k$. 

Recall also that the vector space of bilinear forms on $V$ can be canonically identified with $V^\ast \otimes V^\ast$, by the map $\eta\otimes \psi \mapsto ((v,w)\mapsto \eta(v)\psi(w))$. Since $dx^i\vert_p\otimes dx^j\vert_p$, $i,j=1,\ldots, n$, is a basis of $T_pM^\ast \otimes T_pM^\ast$, we can write the bilinear form $g_p=\sum_{i,j} g_{ij}(p)dx^i\vert_p \otimes dx^j\vert_p$, for some $g_{ij}(p)\in \realnos$ as an element of $T_pM^\ast \otimes T_pM^\ast$ under the identification. Furthermore it is easy to show, that the coefficients $g_{ij}(p)\in \realnos$ in this expression agrees with the previous definition $g_{ij}(p)=g_p(\frac{\partial}{\partial x^i}\vert_p, \frac{\partial}{\partial x^j}\vert_p)$ under this identification. Further, this all makes sense if we forget the point dependency, and may write (within a local chart), $g=\sum_{i,j}g_{ij} dx^i\otimes dx^j$. 

\subsubsection{Complex manifolds}

A \defn{complex manifold} follows the definition of a smooth manifold, with two key differences. Namely, the charts $(\phi_\alpha, U_\alpha)$ take values in complex tuples $\phi_\alpha(U_\alpha)\subseteq \complexnos^n$, and the transition maps $\phi_\alpha\circ \phi_\beta^{-1}$ are required to be holomorphic $\forall \alpha, \beta$. Note if the complex dimension is $n$, then forgetting the holomorphic structure gives a smooth manifold with real dimension $2n$.

On an open subset $U$ of a complex manifold $M$, we can define local coordinates $z^1,\ldots, z^n\in \smooth_\complexnos (U)$, where the transition functions between different `patches' $U, V\subseteq M$ of the manifold are holomorphic functions of these variables. Then we can write $z^k=x^k+iy^k, k=1,\ldots, n$, where $n=\text{dim}_\complexnos M$, and $x^k, y^k$ are smooth real functions on $U$. 

Note that a holomorphic function $f:U\to \complexnos$, $U\subseteq \complexnos$ can be written as $f(z)=f(x, y)=u(x,y)+iv(x,y)$ as a function of two real variables $x, y$, and $u$ and $v$ are the real and imaginary parts of $f$ respectively. 
Note that if we write $z=x+iy$, for real variables $x, y$, we can define the following operators
$$\frac{\partial}{\partial z}:=\frac{1}{2} \left(\frac{\partial}{\partial x} - \frac{\partial}{\partial y}\right)$$
$$\frac{\partial}{\partial \bar{z}}:=\frac{1}{2} \left(\frac{\partial}{\partial x} + \frac{\partial}{\partial y}\right)$$
Then we can define an extended differential map $df_\complexnos(z):T_z \realnos^2 \otimes \complexnos \to T_{f(z)} \realnos^2\otimes \complexnos$ which has corresponding jacobian
$$J(f)=\begin{pmatrix}
    \frac{\partial f}{\partial z} & 0 \\
    0 &   \frac{\partial \bar{f}}{\partial \bar{z}}
\end{pmatrix}$$
with respect to the basis $\{ \frac{\partial }{\partial z},   \frac{\partial }{\partial \bar{z}} \}$. This theory is easily generalised to holomorphic functions on open subsets of $\complexnos^k$ by taking components. 

The following statement highlights a key difference between complex manifolds and smooth real manifolds. Any global holomorphic function on a (compact, connected) complex manifold is constant!

\subsubsection{Example: Complex projective space}

Note that we can define more projective spaces in a similar fashion to real projective space, if we replace $\realnos$ in the definition by any of the division algebras $\complexnos, \mathbb{H}, \mathbb{O}$ (complex numbers, quaternions, octonians). As vector spaces, $\complexnos \simeq \realnos^2$, $\mathbb{H} \simeq \realnos^4$, $\mathbb{O} \simeq \realnos^8$, hence as smooth manifolds $\complexnos P^n$, $\mathbb{H}P^n$, $\mathbb{O}P^n$, have the corresponding real dimension $2n, 4n, 8n$ respectively.

In particular complex projective space $\complexnos P^n$ can be considered as a complex manifold of complex dimension $n=\text{dim}_\complexnos \complexnos P^n$. 

Let us write the atlas on complex projective space for completeness. Define an open covering $U_i:=\{(z_0:\ldots :z_n)| z_i\neq 0\}\subseteq \projCspace$ (for $i=0,\ldots, n$, and $(z_0:\ldots :z_n)$ standard homogeneous coordinates). Then we have charts
\begin{align*}
\phi_i & :U_i \to \complexnos^n \\
& \ (z_0:\ldots : z_n) \xmapsto{\phi_i} \left( \frac{z_0}{z_i}, \ldots, \frac{z_{i-1}}{z_i}, \frac{z_{i+1}}{z_i}, \ldots, \frac{z_n}{z_i} \right)
\end{align*}
From this we obtain transition functions $\psi_{ij}:=\psi_i\circ \psi_j^{-1}:\psi_j(U_i\cap U_j)\to \psi_i(U_i\cap U_j)$ given by
$$\psi_{ij}(w_1, \ldots, w_n) = \left( \frac{w_1}{w_i}, \ldots, \frac{w_{i-1}}{w_i}, \frac{w_{i+1}}{w_i}, \ldots, \frac{w_{j-1}}{w_i}, \frac{1}{w_i}, \frac{w_j}{w_i}, \ldots, \frac{w_n}{w_i} \right)$$
which are clearly bijective and holomorphic.

\section{Bundles and Sections}

\subsubsection{Fibre Bundles}

A \defn{fibre bundle} consists of three topological spaces $E, B, F$, and a continuous surjection $\pi:E\rightarrow B$ (called the \defn{projection}), such that \defn{`local trivialisations'} exist (i.e. for every $p\in B$, there exists a neighbourhood $U\subseteq B$ of $p$ and a homeomorphism $\psi: \pi^{-1}(U) \rightarrow U \times F$, such that $\pi \vert_{\pi^{-1}(U)} = \pi_1 \circ \psi$ where $\pi_1$ is the projection on to the first component). $E$ is called the \defn{total space}, $B$ the \defn{base space}, $F$ the \defn{typical fibre}, and $E_p:=\pi^{-1}(p)$ the \defn{fibre} over $p$. As a shorthand we may denote a fibre bundle as $F\hookrightarrow E\xrightarrow{\pi} B$.

A continuous map $U:E\rightarrow E'$ is a \defn{(fibre) bundle morphism} between two fibre bundles $E\xrightarrow{\pi} B$ and $E'\xrightarrow{\pi'} B'$ with the same typical fibre $F$, if there exists a map $u:B\rightarrow B'$ such that $\pi' \circ U = u \circ \pi$

In the theory of Chern classes, there are two important types of fibre bundles to consider: vector bundles and principle $G$-bundles. Roughly speaking, in the case of vector bundles, each fibre is isomorphic to $\realnos^k$, whereas for $G$-bundles, the fibres are isomorphic to the Lie group $G$. The notion of a bundle morphism also generalises to both these structures, with some extra conditions to preserve the additional structure. 

\subsubsection{Example: the Torus}
We can view the torus $T^2=S^1\times S^1$ as a fibre bundle over the base manifold $S^1$. We define the projection $T^2 \xrightarrow{\pi} S^1$ as $\pi(x, y)=x$. Hence a fibre at $p\in S^1$ is $E_p := T^2_p = \pi^{-1}(p) = \{(p, y): y\in S^1\} = \{p\}\times S^1 \cong S^1$. Given a point $p\in S^1$, $S^1$ is open in $S^1$ (which is endowed with the subspace topology of $\realnos^2$), and hence $S^1$ is an open neighbourhood of $p$. So we can define a global trivialisation $\psi:T^2=\pi^{-1}(\{p\}\times S^1)\to S^1 \times S^1$ given by $\psi(x, y)=(x, y)$. Hence $T^2 \xrightarrow{\pi} S^1$ is a trivial bundle $T^2\cong S^1\times S^1$. 

\subsubsection{Example: Hopf fibrations.}

Let $\mathbb{F}\in \{\realnos, \complexnos, \mathbb{H}, \mathbb{O}\}$, which have respective real dimensions $m: = \text{dim}_\realnos (\mathbb{F})\in \{1,2,4,8\}$. Write $\mathbb{F}^\times := \mathbb{F}\setminus \{0\}$. Recall the projective space $\mathbb{F}P^1$ is by definition the linear $\mathbb{F}$-lines in $\mathbb{F}^2$. Alternatively we can define  $\mathbb{F}P^1:=\frac{\mathbb{F}^2 \setminus \{0\}}{\mathbb{F}^\times}$ with action $\lambda (v, w)=(\lambda v, \lambda w)$ for all $\lambda \in \mathbb{F}^\times$ and $(v,w)\in \mathbb{F}^2 \setminus \{0\}$.  

Now note the sphere $S^{2m-1}\subseteq \mathbb{F}^2\setminus \{0\}$ (so in particular, $S^1 \subseteq \realnos^2$, $S^3 \subseteq \complexnos^2 \simeq \realnos^4$, $S^7\subseteq \mathbb{H}^2 \simeq \realnos^8$, $S^{15}\subseteq \mathbb{O}^2\simeq \realnos^{16}$). Hence the canonical projection $\tilde{\pi}:\mathbb{F}^2 \setminus \{0\} \to \mathbb{F}P^1=\frac{\mathbb{F}^2 \setminus \{0\}}{\mathbb{F}^\times}$ induces a projection $\pi:S^{2m-1} \to \mathbb{F}P^1 = \frac{\mathbb{F}^2 \setminus \{0\}}{\mathbb{F}^\times}$ (by restricting $\tilde{\pi}$ to the sphere). 
Now taking a line $l\in \mathbb{F}P^1$, we see that $\pi^{-1}(l) = l \cap S^{2m-1} \simeq S^{m-1}$ is the unit sphere in $l$. 

Thus we have that $\pi$ is a fibre bundle with total space $S^{2m-1}$, base space $S^m$, and typical fibre $S^{m-1}$. 

\subsubsection{Real vector bundles}

A $k$-dimensional smooth \defn{real vector bundle} is a fibre bundle with typical fibre $\realnos^k$, such that the projection map $E\xrightarrow{\pi} B$ is a smooth surjection of smooth manifolds, and the local trivialisations $\psi:\pi^{-1}(U_\alpha) \rightarrow U_\alpha \times \realnos^k$ are diffeomorphisms (for some open cover $\{U_\alpha\}$ of $B$). Moreover, the fibres $E_p:=\pi^{-1}(p)$ are required to be $k$-dimensional vector spaces and restricting any local trivialisation $\psi_p:=\psi \vert_{E_p}$ should give a linear isomorphism $\psi_p:\pi^{-1}(p)\rightarrow \{p\} \times \realnos^k \canonicaliso \realnos^k$ at each point $p\in B$.

Suppose $\psi$, $\tilde{\psi}$ are two local trivialisations, defined on $\pi^{-1}(U)$, $\pi^{-1}(\tilde{U})$ respectively,
for a vector bundle $E\xrightarrow{\pi} B$. Assume $U \cap \tilde{U}\neq \phi$ and given a 
point $p\in U \cap \tilde{U}$, we have the map $(\psi \circ \tilde{\psi}^{-1})_p:=(\psi)_p \circ (\tilde{\psi}^{-1}\vert_{\{p\}\times \realnos^k})$ 
is a linear isomorphism $(\psi \circ \tilde{\psi}^{-1})_p : \realnos^k \rightarrow \realnos^k$. Hence we can represent $(\psi \circ \tilde{\psi}^{-1})_p$ by an invertible matrix $(g_{mn})_p$ which depends on the point $p$. Hence over $U\cap \tilde{U}$, the entries $g_{mn}\in C^\infty(U\cap \tilde{U})$ are smooth maps. Whence we obtain a map, called a \defn{transition function}
$g:U\cap \tilde{U}\rightarrow \mathrm{GL}_k(\realnos)$ such that $g:p\mapsto (g_{mn}(p))$. 

\subsubsection{Oriented vector bundles}

An \defn{orientation} of an $n$-dimensional real vector space $V$ is an equivalence class of ordered bases $[e_1, \ldots, e_n]$ under the relation $(e_1, \ldots, e_n)\sim (e_1', \ldots, e_n')$ if and only if the change of basis matrix $(a_{ij})$, defined by $e_j=\sum_i a_{ji}e_i'$, has positive determinant $\mathrm{det}(a_{ij})>0$. So to any non-zero vector space we can have exactly two possible orientations $\pm 1$ (corresponding to the two equivalence classes, which in turn corresponds to the strictly positive and negative determinants). Note that given two orderings of a basis $\{e_i\}$, with $e_{\sigma(j)}=e_i$ for a permutation $\sigma\in S^n$, have the same orientation if and only if $\mathrm{sign}(\sigma)=1$. Furthermore an orientation of $+1$ is associated to the standard ordered basis of $\realnos^k$.  

A real vector bundle $E\xrightarrow{\pi} M$ with local trivialisations $\psi_U:\pi^{-1}(U)\to U\times \realnos^k$ is \defn{oriented} if the fibres $\pi^{-1}(p)$ have the same orientation as $\realnos^k$ for every $p \in U$ 
(in other words the local trivialisations are orientation preserving). 
We could equivalently say that a vector bundle is oriented if at each point $p\in U$ of any local frame $s_1, \ldots, s_k\in \Gamma(U, E)$, the basis $s_1(p), \ldots, s_k(p)$ has the same orientation. 
Note that the structure group of an oriented vector bundle can be restricted to the subgroup $\mathrm{GL}_k(\realnos)^+=\{A\in \mathrm{GL}_k(\realnos): \mathrm{det}(A)>0\}$.

\subsubsection{Complex vector bundles}

A \defn{complex vector bundle} follows the definition of a real vector bundle, however has typical fibre $\complexnos^k$. Hence the fibres $E_p$ are $k$-dimensional vector spaces over $\complexnos$. 

Note that complex vector bundles are canonically oriented vector bundles. Note that the base space of a complex vector bundles is not neccessarily a complex manifold. For example we can have a complex vector bundle over a smooth real manifold - (say for example attaching as fibres copies of $\complexnos$ at each point on the circle $S^1$ in a smooth way. 

For most of this exposition - since we are interested in discussing Chern classes - we will generally consider \emph{vector bundles} to mean \emph{complex vector bundles} unless otherwise specified, or unless it is clearly not a complex vector bundle from the context. We will be a little loose with notation, for example sometimes writing $C^\infty (M)$ to mean both smooth real functions or smooth complex functions, which should be obvious from the context, and similarly for example with  sections $\Gamma(M)$, and forms $\Omega(M)$ to denote either the real or complex case. 

\subsubsection{Holomorphic vector bundles}
\defn{A holomorphic vector bundle} (of rank $r$) is a fibre bundle where the base manifold $M$ and total space $E$ are complex manifolds. Furthermore the projection map $\pi:E\to M$ is required to be holomorphic, and the fibres are complex vector spaces. Finally the trivialisations $\phi_i:E|_U \to U\times \complexnos^r$ are required to be biholomorphic, and are $\complexnos$-linear when restricted to a point $p\in U$. By \defn{biholomorphic} we mean that $\phi_i$ is injective, holomorphic and has a holomorphic inverse, and further the image of $\phi_i$ is open in $U\times \complexnos^r$. 

Note that every holomorphic vector bundle is also a smooth manifold. A holomorphic vector bundle is not to be confused with a complex vector bundle. A complex vector bundle does not always admit a holomorphic structure. 

A \defn{holomorphic line bundle} is a holomorphic vector bundle of rank $1$.

\subsubsection{Example: Tautological line bundle $\mathcal{O}(-1)$ on $\complexnos P^n$ }
Note that we can define $\projCspace$ as the lines through $0$ in $\complexnos^{n+1}$, with a surjective map $p:\complexnos^{n+1}\setminus \{0\}\to \projCspace$ such that $v\xmapsto{p} \complexnos v$. 

Note we can do this due to the following correspondence 
\begin{align*}
    \{\text{lines through $0$ in $\complexnos^{n+1}$}\} & \rightleftarrows \frac{\complexnos^{n+1}\setminus \{0\}}{\complexnos^\times} \\
    l & \mapsto [v], \text{ for any } v\in l\setminus \{0\} \\
    \complexnos v & \mapsfrom [v]
\end{align*}

Now consider the trivial vector bundle $\projCspace \times \complexnos^{n+1}$ of rank $n+1$ over $\projCspace$. We define
$$\mathcal{O}(-1) := \{(l, w)\in \projCspace \times \complexnos^{n+1} : w\in l\}$$
(Note in particular that $(l, 0)\in \mathcal{O}(-1)$).

Consider the composition $\mathcal{O}(-1) \xhookrightarrow{i} \projCspace \times \complexnos^{n+1} \xrightarrow{p_1} \projCspace$ where $p_1$ is projection onto the first factor and $\xhookrightarrow{i}$ denotes the inclusion map. Then this composition gives a holomorphic vector bundle $\pi := p_1 \circ i: \mathcal{O}(-1) \to \projCspace$ called the \defn{tautological line bundle on $\complexnos P^n$}. 

Now we define `coordinates' for this bundle. Consider $v_p\in \mathcal{O}(-1)_p$. Now a point $p\in \complexnos P^n$ corresponds to a line $l\in \complexnos^{n+1}$. So for $v_p = (p, v)\in l$, we have $v_p = (v^0_p, \ldots, v^n_p) \in \complexnos^{n+1}$. Hence we obtain coordinates defined by $z_i(v_p)=v_p^i \in \complexnos$, and furthermore $z_i\in \Gamma(\mathcal{O}(-1)^*) = \Gamma(\mathcal{O}(1))$, where $\mathcal{O}(1)$ is the dual holomorphic line bundle of $\mathcal{O}(-1)$. We call $z_0, \ldots, z_n$ the standard homogeneous linear coordinates on $\projCspace$ as sections of $\Gamma(\mathcal{O}(1))$.

\subsubsection{Tangent bundles}

The \defn{tangent space} $T_p M$ at a point $p\in M$ is the algebra of point derivations of $C^\infty (M)$. So by definition any \defn{tangent vector} $(X_p:C^\infty (M)\to \realnos) \in T_p M$ is a linear map that satisfies the following Liebniz rule $X_p(fg)=f(p)X_p(g) + X_p(f)g(p)$, for $f,g\in C^\infty (M)$. Equivalently we can define a tangent vector at $p\in M$ as the derivative to a curve $\gamma: (-\epsilon, \epsilon)\to M$ in the manifold with $p\in \gamma (-\epsilon, \epsilon)$. 

Given a chart $(U, \phi)$ of $M$, we can write $\phi = (x^1, \ldots, x^n)$, where $x^i := \pi^i\circ \phi$, and $\pi^i:\realnos^n \to \realnos$ is the canonical projection. We obtain a basis $\{\frac{\partial}{\partial{x^i}} \vert_p\}_{i=1,\ldots, n}$ 
of $T_p M$, where we define $\frac{\partial}{\partial{x^i}} \vert_p f := \frac{\partial}{\partial{\pi^i}}\vert_{\phi (p)} f\circ \phi^{-1}$. Note in particular that 
$\dim (T_p M) = n$ where 
$n$ is the dimension of $M$.

The \defn{tangent bundle} over an $n$-dimensional manifold $M$ is the $n$-dimensional vector bundle $TM:=\cup_p T_p M\xrightarrow{\pi} M$ with $\pi(X_p)=p$ for $X_p\in T_p M$. Note that given a chart $(U, \phi)$ of $M$ with coordinates $\phi =(x_1, \ldots, x_n)$, we obtain a local frame $\{\frac{\partial}{\partial{x^i}}\}_i$ for the tangent bundle, which corresponds to a local trivialisation. 

\subsubsection{Holomorphic tangent bundle}
Let $(\phi, U)$ be a chart of a complex manifold $M$, with coordinates $(z^1, \ldots, z^n)$ over $U$. Then we can define the \defn{holomorphic tangent bundle} locally over $U$ as the $\smooth(U)$-linear span of $\{ \frac{\partial}{\partial z^1}, \ldots,  \frac{\partial}{\partial z^n}\}$. 

This local definition glues together nicely to give a global definition of the holomorphic tangent bundle, by considering the transition functions $\phi_{ij}:=\phi_i\circ \phi_j^{-1}$ of the charts $(\phi_i, U_i), (\phi_j, U_j)$, and applying the transformation given by the Jacobian
$$J(\phi_{ij})(\phi_j(z)):=\left( \frac{\phi_{ij}^k}{\partial z^l} (\phi_j(z)) \right)_{k, l}$$

Note we can also define the \defn{holomorphic cotangent bundle} as the dual of the holomorphic tangent bundle. (Hence we can define $p$-forms $\Omega^p(M)$ etc. for complex manifolds $M$).

\subsubsection{Sections and vector fields}
A \defn{section} of a vector bundle $E\xrightarrow{\pi} B$ is a smooth map $\sigma:B\rightarrow E$, such that $\pi \circ \sigma=\id_B$ (in other words, $\sigma(p)\in E_p, \forall p\in B$). The set of all sections of a vector bundle $E$ is denoted $\Gamma (E)$ and is a module over smooth functions $C^\infty(M)$.

A \defn{vector field} is a section of the tangent bundle $TM$. The $C^\infty (M)$ module of vector fields is denoted $\mathfrak{X}(M):=\Gamma(TM)$. Note that if there exists a global trivialisation from the tangent bundle, then the base manifold is said to be \defn{parallelizable}. 

\subsubsection{Frames and trivialisations}

Let $E\to M$ be a complex vector bundle of rank $r$. (Note the following ideas hold for any vector bundle, but we write them in the special case of a complex vector bundle for simplicity.) 

Given an open subset $U\subseteq M$, we denote the $C^\infty (U)$-module of sections on $U$ rather than on the whole base manifold, as $\Gamma (U, E)$. A \defn{local frame} for a vector bundle $E$ is a collection of local sections $e_1, e_2, \ldots, e_r\in \Gamma (U, E)$, such that for every $p\in U$, $e_1(p),\ldots, e_r(p)$ is a basis of the fibre $E_p$. 

We can identify local trivialisations with local frames. Given a local frame $e_1, \ldots, e_r\in \Gamma(U, E)$, define a local trivialisation $\psi^{-1}:U\times \complexnos^r \to \pi^{-1}(U)$ by
$\psi^{-1}(p, a_1, \ldots, a_r) = \sum_{i=1}^r a_ie_i(p)$. Conversely, given a local trivialisation $\psi:\pi^{-1}(U) \to U\times \complexnos^r$, define a local frame $e_1,\ldots , e_r\in \Gamma(U, E)$ by $e_i(p)=\psi^{-1} (p, \delta_i)$, where $\delta_i = (0, \ldots, 0, 1, 0, \ldots, 0)$ with the $1$ in the $i$'th position. 

Furthermore a \defn{global frame} (i.e. a set of sections $e_1, \ldots e_r\in \Gamma(E)$ which form a basis at each point $p\in B$) corresponds to a \defn{global trivialisation} $\psi:E \to M\times \complexnos^k$. If such a global trivialisation exists, we write $E\cong M\times \complexnos^r$ (noting that global trivialisations are diffeomorphisms) and say that $E$ is the \defn{trivial vector bundle} of dimension $r$. 
Note that if in particular, we cannot find $r$ non-vanishing sections in a vector bundle (in which case we cannot find $r$ linearly independent vectors in $E_p$ at every point $p\in B$), then a global frame does not exist, hence the vector bundle is non-trivial.  

Let $f=(e_1,\ldots, e_r)$ be a local frame of the bundle over some open neighbourhood $U\subseteq M$. A smooth map $g:U\to GL_r(\complexnos)$ is called a \defn{change of frame mapping}, since $fg$ is another local frame on $U$ given by 
$$fg=(\sum_{k=1}^r g_{k1}e_k,\ldots, \sum_{k=1}^r g_{kr}e_k)$$
noting that $g_{kj}\in \smoothCmaps$. We may also express this in terms of matrix multiplication as $(fg)(p):=f(x)g(x)$. Note if we wrote $f$ as a column vector instead of a row vector then we would write $gf$ instead of $fg$. One can show that given any two local frames $f, f'$ of $E$ over $U$, there always exists a change of frame mapping $g:U\to GL_r(\complexnos)$ such that $f'=fg$. 

\subsubsection{Local representation of sections}
Given a local frame $f=(e_1,\ldots, e_r)$ we may express a section $s\in \Gamma(U, E)$ locally as
$$s=\sum_{i=1}^r s^i(f)e_i$$
for some unique $s^i:=s^i(f)\in \smoothCmaps$ ($i=1,\ldots, r$), and we define 
$$s(f):= \begin{pmatrix}s^1(f) \\ \vdots \\ s^r(f)\end{pmatrix} \in \smoothCmaps^r$$
Hence locally we have shown that $\Gamma(E, U) \simeq \smoothCmaps^r$.

We can view this same fact in terms of local trivialisations.  Let $\psi: E\vert_U \to U\times \complexnos^r$ be a local trivialisation (which we can assume is associated with the frame $f$,  but the following holds in more generality also).  Then we may write
\begin{align*}
\Gamma(E, U) & \simeq \smoothCmaps^r \\
s & \mapsto \psi\circ s = (s^1, \ldots,  s^r)
\end{align*}
for $s^i\in \smoothCmaps$.  
The last equality holds since $\psi\circ s$ is a section of $U\times \complexnos^r$,
and $\Gamma(U\times \complexnos^r)$ are essentially smooth maps $U\to \complexnos^r$, which by taking components is the same as $r$ smooth maps,  i.e.  $\smoothCmaps^r$.  

Suppose $g:U\to GL_r(\complexnos)$ is a change of frame mapping, so that $f'=fg$ is another local frame. Then we obtain the following transformation law $s(f')=s(fg)=g^{-1}s(f)$, or in other words
$$gs(f')=s(f)$$
which follows from direct computation since $s^i(fg)=\sum_{j=1}^r g_{ij}^{-1} s^j(f)$.

\subsubsection{Hermitian vector bundles}
On a complex vector space $V$, a \defn{Hermitian inner product} is a map $(\cdot, \cdot):V\times V\to \complexnos$ such that for all $u,v\in V$
\begin{align*}
& (u,v)=\overline{(v, u)} \\
& (\lambda u + v, w) = \lambda (u, w) + (v, w) \\
& (u, v)\geq 0 \\
& (u,v)=0, \forall v \implies u=0
\end{align*}
Or in other words $(\cdot, \cdot)$ is a conjugate-symmetric, sesquilinear, positive, non-degenerate map. 

For example on $V=\complexnos^n$, $(x, y)=\sum_i x_i\overline{y_i}$, $\forall x,y\in \complexnos^n$, is a Hermitian inner product. Note in this case we have that for a matrix $A\in M_n(\complexnos)$ we have $(Au, v)=(u, \overline{A}^Tv)$.

A \defn{Hermitian vector bundle} assigns a Hermitian inner product to each fibre $E_p$ of the vector bundle. More formally, for every open neighbourhood $U\subseteq M$, and for every pair of sections $s, \gamma \in \Gamma(U, E)$, \defn{a Hermitian metric} $h$ on $E$ is a map $h(s, \gamma):=\langle s, \gamma \rangle:U\to \complexnos$ such that $\langle s, \gamma \rangle(p)=\langle s(p), \gamma (p) \rangle$ is a Herimitian inner product on $E_p$, and $\langle s, \gamma \rangle$ is smooth. 

One may show that it is possible to construct a Hermitian metric on any complex vector bundle (the construction of which involves a partitions of unity). 

\subsubsection{Hermitian metric local representation}
Given a local frame $f=(e_1, \ldots, e_r)$ over $U$, define 
$$h(f)_{ij}=\langle e_i, e_j \rangle \ (:U\to \complexnos)$$
Whence $h(f)\in M_r(\smoothCmaps)$ is a positive definite, Hermitian, symmetric matrix, that represents $h$ locally with respect to the frame $f$. 

In terms of a local trivialisation $\psi:E|_U\to U\times \complexnos^r$ a Hermitian metric locally on $E|_U$ can be expressed as requiring
$$\langle \cdot, \cdot \rangle_p := h_p(\psi^{-1}_p(\cdot), \psi^{-1}_p(\cdot)):\complexnos^r \times \complexnos^r \to \complexnos$$
to be a Hermitian inner product at each $p\in U$. 

Recall given $s, \gamma\in \Gamma(U, E)$, we can write $s(f)=(s^1(f), \ldots, s^r(f))^T$, $\gamma(f)=(\gamma^1(f), \ldots, \gamma^r(f))^T$ locally with respect to the frame $f$. Then one may show that 
$$\langle s, \gamma \rangle = \overline{\gamma(f)^T}h(f)s(f)$$
(where the product here is matrix multipilicaiton).

Furthermore, if $g:U\to GL_r(\complexnos)$ is a change of frame mapping with $f'=fg$. Then we get the transformation law
$$h(f')=\overline{g^T}h(f)g$$
for the local representation of $h$ with respect to the frames $f, f'$.

\subsubsection{Example: Constant hermitian structure on the trivial bundle}
Consider the trivial bundle $E=M\times \complexnos^r$. Recall the trivial bundle gives a global trivialisation hence a global frame, thus any sections $s, \sigma \in \Gamma(E)$ can be written as  $s=(s^1, \ldots, s^r)$, $\sigma=(\sigma^1, \ldots, \sigma^r)$ over the frame, where $s^i, \sigma^i \in \smooth_\complexnos (M)$. Now we can consider the \defn{constant hermitian structure} defined on the trivial bundle as
$$h(s, \sigma)=s^1\bar{\sigma^1} + \ldots s^r\bar{\sigma^r}$$
(This is `constant' in the sense that if we take the hermitian inner product of two vectors at a point, and `the same' two vectors at another point, it will give you some constant complex number). 

\subsubsection{Example: Hermitian metric on a holomorphic line bundle}
Let $E$ be a holomorphic line bundle, with global holomorphic sections $s^1, \ldots, s^r$ of $E$, so that at every point at least one of the sections is non-zero. Let $\psi:E|_U\to U\times \complexnos^r$ be a local trivialisation over $U$ (so that for $p\in U$, $\psi_p:E_p\to \{p\}\times \complexnos^r \simeq \complexnos^r$). Then $\forall p\in U$ define
$$h(v_p):=\frac{|\psi_p(v_p)|^2}{\sum_{i=1}^r |\psi_p (s^i_p)|^2}$$
for $v_p\in E_p$. This is a Hermitian inner product at $p$, and hence gives a hermitian metric on $E$. Note by an abuse of language, this can be written in shorthand as $h(v_p)=(\sum_i |s^i|^2)^{-1}$.

In particular, $\mathcal{O}(1)$, the dual of the tautological line bundle on $\complexnos P^n$ (with the standard globally generating sections $s^1, \dots, s^r\in \Gamma(\mathcal{O}(1))$) has this Hermitian metric.

\subsubsection{Example: Fubini-Study metric on tangent bundle of $\complexnos P^n$}
Let $(U_j, \phi_j)$ be the standard atlas of $\complexnos P^n$, where we recall that $\phi_j:U_j\simeq \complexnos$ such that $\phi_j((z_0 : \ldots : z_n))=(\frac{z_0}{z_j}, \ldots, \frac{z_{j-1}}{z_j}, \frac{z_{j+1}}{z_j}, \ldots, \frac{z_n}{z_j})\sim (1, w^1, \ldots, w^n)$. This gives a local frame for the holomorphic tangent bundle of $\complexnos P^n$, namely $\{\partial_1, \ldots, \partial_n\}$, where $\partial_i=\frac{\partial}{\partial w^i}$ over $U_j$. 

We define a hermitian metric on the tangent bundle for $\complexnos P^n$ as follows. On $U_j$ let
$$h_{i\bar{j}}:=h(\partial_i,\bar{\partial}_j):=\frac{((1+|w|^2))\delta_{ij}-\bar{w}
_i w_j}{(1+|w|^2)^2}$$
where $\delta_{ij}$ is the Kronecker delta and $|w|^2 = |w^1|^2 + \ldots + |w^n|^2$. Note that $h_{i\bar{j}}$ is a positive definite hermitian matrix, and defines the \defn{Fubini-study metric}. 

We define the \defn{Fubini-study form} locally over $U_j$
\begin{align*}
    \omega_{FS} & := \frac{i}{2\pi} \sum_{i, j} h_{i\bar{j}} dw^i \wedge d\bar{w}^j \\
 & = \frac{i}{2\pi} \partial \bar{\partial} \text{log} \left( 1 + \sum_{i=1}^k |w_k|^2  \right) \in \Omega^{1,1}(U_j)
\end{align*}


Note one can show that in fact $\omega_{FS}$ can be defined as a global form in $\Omega^{1,1}(M)$ since it is compatible over all intersections $U_i\cap U_j$. 

One can show that $\omega_{FS}$ is a closed, real, positive definite $(1,1)$-form. Since $\omega_{FS}$ it is a top form over $1$-dimensional projective space, we can integrate and show that $\int_{\complexnos P^1} \omega_{FS}=1$. This is since
\begin{align*}
  \int_{\complexnos P^1} \omega_{FS} & = \int_\complexnos \frac{i}{2\pi}\frac{1}{(1+|z|^2)^2}dz\wedge d\bar{z}\\
  & = \frac{1}{\pi}\int_{\realnos^2}\frac{1}{(1+||(x,y)||^2)^2}dx\wedge dy \\
  & = 2 \int_0^\infty \frac{rdr}{(1+r^2)^2} \\
  & = 2 \left( \frac{1}{2} \int_1^\infty \frac{du}{u^2 }\right) \\
  & = \int_1^\infty \frac{du}{u^2}  = \left[ - \frac{1}{u} \right]_1^\infty = 1
\end{align*}

\subsubsection{New bundles from old}
Let $\pi_E:E\to M, \pi_F:F\to M$ be two (complex) vector bundles over $M$, ($p\in M$).

The \defn{dual bundle} of $E$ is $\pi_{E^\ast}:E^\ast \to M$ with fibres $(E^*)_p = (E_p)^*$.

The \defn{tensor product} of $E$ and $F$ is a bundle $\pi_{E\otimes F}:(E\otimes F) \to M$ with fibres $(E\otimes F)_p:=(E_p \otimes F_p)$. Furthermore given sections $s_E\in \Gamma(E), s_F\in \Gamma(F)$ then define $s_E\otimes s_F\in \Gamma(E\otimes F)$ as $(s_E\otimes s_F)(p)=s_E(p)\otimes s_F(p)$ for all $p\in M$. Lastly note that if $\{s^1, \ldots, s^k\}$ is a local frame for $E$, and $\{\sigma^1,\ldots , \sigma^t\}$ is a local frame for $F$, then $\{s^i\otimes s^j : i=1,\ldots, k, j=1,\ldots , t\}$ is a local frame for $E\otimes F$, (since $s^i(p)\otimes s^j(p)$ gives a basis of $E_p\otimes F_p$).  

We can also define a bundle \defn{$\Hom(E,F)$} over $M$, with fibres $\Hom(E,F)_p = \Hom(E_p, F_p)$. Then this gives us another bundle namely $\defn{\End}(E):=\Hom(E,E)$. 

Suppose $f:M\to N$ is a function of manifolds, and $E$ is a vector bundle on $N$. Then we obtain a vector bundle $f^* E$ on $M$ called the \defn{pull-back bundle} with fibres given by $(f^*E)_p=E_{f(p)}$.

Note that given a real vector space $V$ we can construct another real vector space $V^\complexnos := V\otimes_\realnos \complexnos$, where $\complexnos$ is considered as a real vector space. $V^\complexnos$ can be made into a complex vector space by defining complex multiplication $\beta(v\otimes \alpha) = v\otimes (\beta \alpha)$ for $v\in V$, $\alpha, \beta \in \complexnos$. The complex vector space $V^\complexnos$ is called the \defn{complexification} of $V$. Similarly we can obtain a complex vector bundle $E^\complexnos := E\otimes \complexnos$ from a real vector bundle $E$, by complexifying each fibre. To be explicit, the fibres of $E^\complexnos$ are given by $(E\otimes \complexnos)_p := E_p\otimes \complexnos$.
Conversely one may also obtain a a real vector space from a complex vector space by restricting scalars, and at a global level we can obtain a real vector bundle from a complex vector bundle. 

\subsubsection{Induced hermitian metrics}
Suppose the bundles $E, F$ are endowed with hermitian structures. Then one can naturally define hermitian structures on $E^*$, $E\oplus F$, $E\otimes F$, $\Hom(E, F)$, and moreover if $(M, g)$ is a Hermitian manifold then $TM$, $TM^*$, and more generally $\bigwedge^{p,q} M$ all have Hermitian structures.

Suppose $f:M\to N$ smooth map between manifolds, and $E$ is a bundle on $N$, and $N$ has a hermitian structure $h$. Then the pull back bundle $f^*E$ has a hermitian structure given by the hermitian inner product
$$(f^*h)_p:=h_{f(p)}$$
on the fibres $(f^*E)_p=E_{f(p)}$.

\subsubsection{Multilinear algebra}
We state here some vector space isomorphisms without justification, to review concepts from multi-linear algebra, which will be useful to us. Let $U, V, V_i, W$ be finite dimensional vector spaces over a field, say the real numbers for simplicity. We use the following notation.
$$T^{(r,s)}:=\underbrace{V^*\otimes \cdots \otimes V^*}_{s \text{ times}} \otimes \underbrace{V \otimes \cdots \otimes V}_{r \text{ times}} $$
$$\Hom(V_1, \ldots V_n; \realnos):= \{\text{Multilinear maps } V_1\times \cdots \times V_n \to \realnos\}$$
$$\Hom(V_1, \ldots V_n; W):= \{\text{Multilinear maps } V_1\times \cdots \times V_n \to W\}$$

Now we obtain the following vector space isomorphisms.
\begin{align*}
V^*\otimes W & \simeq \Hom_\realnos(V, W) \\
V^*\otimes V & \simeq \Hom_\realnos(V, V) = \text{End}_\realnos (V) 
\end{align*}
Note in particular that the dimension of $V^*\otimes W$ is $\text{dim}(V)\text{dim}(W)=\text{dim}(\Hom(V, W))$. We can also generalise this isomorphism to the multilinear case as follows.
\begin{align*}
V_1^* \otimes V_2^* \otimes \cdots \otimes V_n^* & \simeq (V_1\otimes \cdots \otimes V_n)^* \\
& \simeq \Hom_\realnos (V_1, \ldots, V_n; \realnos) 
\end{align*}
\begin{align*}
V_1^* \otimes V_2^* \otimes \cdots \otimes V_n^* \otimes W & \simeq \Hom_\realnos((V_1\otimes \cdots \otimes V_n), W) \\
& \simeq \Hom_\realnos (V_1, \ldots, V_n; W) 
\end{align*}

Note that linear maps $V_1\otimes \cdots \otimes V_n\to \realnos$ are the same as multilinear maps $V_1\times \cdots \times V_n \to \realnos$.

Finally, note the following identity.
$$V\otimes (W\oplus U)=(V\otimes W)\oplus (V\otimes U)$$

\subsubsection{Tensoriality}
Suppose $(\pi_E, E, M)$, $(\pi_F, F, M)$ are two (complex) vector bundles over the same (smooth) base manifold $M$. A \defn{bundle morphism} is a smooth map $\phi:E\to F$ that satisfies
\begin{enumerate}
    \item $\pi_F\circ \phi = \pi_E$ 
    \item $\phi \vert_{E_p}=:\phi_p:E_p\to F_p$ is linear, $\forall p\in M$
\end{enumerate}
Note that the first property in particular implies that $\phi(E_p)\subseteq F_p$ at every $p\in M$ (and so the codomain in the second property makes sense).  

We denote the set of all bundle morphisms from $E$ to $F$ as \defn{Hom($\mathbf{E, F}$)}. One can also multiply a bundle morphism $\phi \in \text{Hom}(E, F)$ by a smooth function $f\in C^\infty (M)$ to produce another bundle morphism $f \phi :E\to F$,
given by $v\mapsto f(p)\phi(v)$ for all $v\in E_p$.  
This makes $\text{Hom}(E, F)$ into a $C^\infty(M)$-module.

A useful and important fact is that if $\phi:E\to F$ is a bundle morphism, then 
\begin{align*}
\tilde{\phi}:\ & \Gamma(E)\to \Gamma(F) \\
& s \mapsto \phi \circ s
\end{align*}
is a homomorphism of $C^\infty(M)$-modules. (In other words, $\tilde{\phi}(fs)=f\tilde{\phi}(s)$ for all $f\in C^\infty (M), s\in \Gamma(E)$.)

Moreover, this induces a module isomorphism 
\begin{align*}
\text{Hom}(E, F) & \simeq \text{Hom}_{C^\infty (M)}(\Gamma(E), \Gamma(F)) \\
\phi & \mapsto \tilde{\phi} 
\end{align*}

Conversely if a map $L:\Gamma(E)\to \Gamma(F)$ is $C^\infty(M)$-linear (i.e. $L_p(fs)=fL_p(s)$ for all $f\in C^\infty (M), s\in \Gamma(E), p\in M$), then there exists a unique bundle morphism $\phi:E\to F$ such that $L=\tilde{\phi}$, in which case we say that $L$ is \defn{tensorial}. 

As an example, we show that 1-forms are tensorial. Consider $\omega\in \Omega(M) = \Gamma(TM^*)$. So $\omega:\Gamma(TM)\to \smooth (M)\simeq \Gamma(M\times \realnos)$. By definition, note that for $x\in M$, $X\in \Gamma(TM)$, we have $\omega(X)_x = \omega_x(X_x)\in \realnos$. Now for $f\in \smooth(M)$, we have $\omega(fX)_x = \omega_x((fX)_x)= \omega(f(x)X_x) = f(x)\omega_x(X_x)$, hence $\omega(fX)=f \omega(X)$. So $\omega$ is tensorial and comes from a bundle morphism $TM\to M\times \realnos$ over the base manifold $M$. In particular $\omega_p:T_p(M)\to \{p\} \times \realnos \simeq \realnos$ is linear.  

We can generalise this definition of tensoriality to say that a $\smooth(M)$-multilinear map $\phi:\Gamma(E_1)\times \Gamma(E_2)\times \ldots \times \Gamma(E_k)\to \Gamma(F)$ is tensorial, (where $E_i, F$ are bundles over $M$). Or in other words, $\phi$ is tensorial as a $\smooth(M)$-linear map $\Gamma(E_1)\otimes_{\smooth(M)} \cdots \otimes_{\smooth(M)} \Gamma(E_k)\to \Gamma(F)$. Note that $\Gamma(E_1)\otimes_{\smooth(M)} \cdots \otimes_{\smooth(M)} \Gamma(E_k)\simeq \Gamma(E_1\otimes \cdots \otimes E_k)$, hence if $\phi$ is tensorial, it comes from a bundle morphism $E_1\otimes \cdots \otimes E_k\to F$.

As an example of this consider tensor fields, namely 
\begin{align*}
T & \in \Gamma(T^*M\otimes \cdots \otimes T^*M \otimes TM) = \Gamma(T^{(1, k)}(TM)) \\
& \ \ \simeq \text{Hom}_{\smooth(M)}(\Gamma(TM), \Gamma(TM), \ldots, \Gamma(TM); \Gamma(TM)) \\
& \ \ = \text{Hom}_{\smooth(M)}(\mathfrak{X}(M), \mathfrak{X}(M), \ldots, \mathfrak{X}(M); \mathfrak{X}(M))
\end{align*}
which are $\smooth(M)$-multilinear, which is seen from the last isomorphism (which is just a generalisation of $\Gamma(E^*\otimes F)\simeq \Hom_{\smooth (M)}(E, F)$). Hence every tensor field  $T:\mathfrak{X}(M)\times \mathfrak{X}(M) \ldots \times \mathfrak{X}(M)\to \mathfrak{X}(M)$, comes from a bundle morphism $TM\otimes \cdots \otimes TM\to TM$.

As an example, we will see that the classical curvature form $R:\Gamma(TM)\times \Gamma(TM) \times \Gamma(E)\to \Gamma(E)$ is $\smooth(M)$-trilinear (i.e. is tensorial), and hence comes from a bundle morphism $TM\otimes TM\otimes E\to E$. Furthermore torsion $T:\Gamma(TM)\times \Gamma(TM)\to \Gamma(TM)$ is $\smooth (M)$-bilinear, and hence comes from a bundle morphism $TM\otimes TM\to TM$.

Note that if $L$ is not tensorial, then it does not in general give rise to a bundle morphism. We will see that some non-examples include the connection operator (which is of the form $L_p(fs)=fL_p(s)+X f\cdot s$, where we have added an extra term due to a Liebniz rule). However we will also see that any two connections differ by a tensorial map, i.e. that $\nabla^1-\nabla^2$ is tensorial for connections $\nabla^1, \nabla^2$, which will allow us to express connections locally in terms of this operator. 

Another non-example is the exterior derivative $d:\Gamma(M\times \realnos)=\smooth(M)\to \Omega^1(M)=\Gamma(T^*M)$. For $d$ to be tensorial we would require $d(fg)=f\cdot dg$ for $f\in \smooth(M)$, $g\in \Gamma(M\times \realnos)\simeq \smooth(M)$. However this is in general not the case, due to the Liebniz rule, i.e. $d(fg)=g\cdot df + f\cdot dg$ and $g\cdot df\neq 0$ in general. 


\subsubsection{More module isomorphisms}
The technicalities of viewing bundles in a variety of ways will become useful in later exposition. We also observe here that constructions relating to vector spaces (local constructions) can be generalised into global statements relating spaces of smooth sections with collections of bundle morphisms.

Consider the following useful $C^\infty(M)$-module isomorphism. Let $\pi_{E^\ast}:E^\ast \to M$ be the dual bundle of $\pi_E:E \to M$. Then 
\begin{align*}
\Gamma(E^\ast) & \simeq \text{Hom}(E, M\times \complexnos) \\
s & \xmapsto{\simeq} (v_p\mapsto (p, s_p(v_p))
\end{align*}
for all $v_p\in E_p$, $p\in M$. Note this makes sense as for $s\in \Gamma(E^*)$, $p\in M$, then $s_p:E_p\to \complexnos$ is a linear map (i.e. an element of $(E_p)^*$).
Note also that $M\times \complexnos$ is the trivial line bundle. 

As a corollary of the previous two isomorphisms we have 
$$\Gamma(E^\ast)\simeq \text{Hom}_{C^\infty (M)}(\Gamma(E), C^\infty (M))$$
since $\Gamma(E^\ast) \simeq \text{Hom}(E, M\times \complexnos) \simeq \text{Hom}_{C^\infty (M)}(\Gamma(E), \Gamma(M\times \complexnos))\simeq \text{Hom}_{C^\infty (M)}(\Gamma(E), C^\infty (M))$. The last equality holds since we have a one-dimensional global trivialisation $M\times \complexnos$, and we saw that we can write a section locally as smooth functions given a local trivialisation (i.e. $\Gamma(U\times \complexnos^r)\simeq C^\infty (M)^r$, with $r=1, U=M)$.

As an application, if we set $E=TM$ to be the tangent bundle on $M$ in this isomorphism, then $\Gamma(TM^*) = \Omega^1(M) \simeq \text{Hom}_{C^\infty (M)}(\Gamma(TM), C^\infty (M))$. So given any one-form $\omega\in \Omega^1(M)$ and vector field $X\in TM$, we have that $\omega(X)\in C^\infty(M)$ is a smooth function.

It is easy to see that we can generalise the previous two isomorphisms as follows
\begin{align*}
\Gamma(\oplus^r E^\ast) & \simeq \text{Hom}(E, M\times \complexnos^r) \\
 & \simeq \text{Hom}_{C^\infty (M)}(\Gamma(E), C^\infty (M)^r)
\end{align*} 

Let us investigate another bundle, namely $E^*\otimes F$ which has fibres 
$$(E^*\otimes F)_p=E_p^*\otimes F_p\simeq \text{Hom}_\complexnos (E_p, F_p)$$
where the last isomorphism is a fact from linear algebra. To be explicit, given vector spaces $V, W,$ recall the vector space isomorphism
\begin{align*}
V^*\otimes W & \simeq \text{Hom}(V, W) \\
\mu \otimes w & \xmapsto{\simeq} (v \mapsto \mu(v)\cdot w)
\end{align*}
which can be applied pointwise at each fibre of $E^*\otimes F$. This can be made into a global statement as follows 
$$\Gamma(E^*\otimes F)\simeq \text{Hom}(E,F)$$
Using that $\text{Hom}(E,F)\simeq \text{Hom}_{C^\infty (M)}(\Gamma(E), \Gamma(F))$ (as seen previously), we get
\begin{align*}
\Gamma(E^*\otimes F) & \simeq \text{Hom}_{C^\infty (M)}(\Gamma(E), \Gamma(F)) \\
\epsilon \otimes \eta & \xmapsto{\simeq} (\sigma \mapsto \epsilon(\sigma) \eta)
\end{align*}
where $\sigma\in \Gamma(E)$, and note that $\epsilon(\sigma)\in C^\infty (M)$. To understand why $\epsilon \otimes \eta\in \Gamma(E^*\otimes F)$ here, think of $(\epsilon \otimes \eta)_p=\epsilon_p \otimes \eta_p, \forall p\in M$ giving a section of $E^*\otimes F$; or more formally, note the canonical $C^\infty(M)$-module isomorphisms
\begin{align*}
\Gamma(E\otimes F) & \simeq \Gamma(E)\otimes_{C^\infty (M)} \Gamma(F) \\
\Gamma(E^*) & \simeq \Gamma(E)^* 
\end{align*}

\section{Differential Forms}

\subsubsection{Complex differential forms}
Complex differential forms on a manifold are essentially differential forms which are allowed to take complex coefficients. We can write any complex $k$-form as a $(p,q)$-form. Let us make this more precise.

Define $\bigwedge^p_\complexnos(T^\ast M):=\bigwedge^p (T^\ast M)\otimes \complexnos$, i.e.\ the complexification of $\bigwedge^p_\realnos(T^\ast M)$. Note that we can express the elements of $\Omega^p_\complexnos(M)=\Gamma(\bigwedge^p_\complexnos(T^\ast M))$ as $\omega+i\eta \in \Omega^p_\complexnos(M)$ for $\omega, \eta \in \Omega_\realnos ^p(M) = \Gamma(\bigwedge^p_\realnos(T^\ast M))$, and $i$ the imaginary unit.

Let $M$ be a complex manifold, with local coordinates $z^1, \ldots, z^n$ over $U$, and $z^k=x^k+iy^k$. We note the following two complex one-forms defined as
$$dz^k := dx^k + i dy^k \in \Omega^1_\complexnos (M)$$
$$d\bar{z}^k := dx^k - i dy^k \in \Omega^1_\complexnos (M)$$
Then any complex one-form $\omega\in \Omega^1_\complexnos (M)$ can be written uniquely for $f_k, g_k \in C_\complexnos ^\infty(M)$ as 
$$\omega = \sum_{k=1}^n (f_kdz^k + g_kd\bar{z}^k)$$

We define $\Omega^{1, 0}(M)$ to be the forms generated by $dz^k, k=1, \ldots, n$. In other words $\Omega^{1, 0}(M)$ are all possible $\smooth(M)$-linear combinations of the $dz^k$. Define $\Omega^{0, 1}(M)$ to be the forms generated by $d\bar{z}^k, k=1, \ldots, n$. 
Then we define 
$$\Omega^{p, q}_\complexnos (M) := \underbrace{\Omega^{1, 0} \wedge \Omega^{1, 0} \wedge \cdots \wedge \Omega^{1, 0}}_{p \text{ times}} \wedge \underbrace{\Omega^{0, 1} \wedge \Omega^{0, 1} \cdots \wedge \Omega^{0, 1}}_{q \text{ times}}$$
(Note that one can show these forms transform tensorially - under a change of holomorphic coordinates - hence give rise to complex vector bundles). An element $\alpha\in \Omega^{p,q}_\complexnos(U)$ can be written locally as 
$$\alpha := \sum_{|I|=p, |J|=q} f_{I,J} dz^I \wedge d\bar{z}^J \in \Omega^{p,q}_\complexnos(U)$$
where $f_{I, J}\in \smooth_\complexnos(U)$, and $I, J$ represent the appropriate multi-indices. 

Now one can show that
\begin{align*}
\Omega^k_\complexnos(M) & =\bigoplus_{k=p+q} \Omega^{p, q}_\complexnos (M) \\
& = \Omega^{k, 0}\oplus\Omega^{k-1, 1}\oplus \cdots \oplus \Omega^{1, k-1}\oplus \Omega^{0, k}
\end{align*}
This induces a vector bundle with projection 
$$\pi^{p,q}:\Omega^k_\complexnos (M)\to \Omega^{p, q}_\complexnos(M)$$

\subsubsection{Dolbeault operators}
The usual exterior derivative $d:\Omega^r \to \Omega^{r+1}$ gives $d(\Omega^{p,q}_\complexnos)\subseteq_{r+s=p+q+1} \Omega^{r, s}$. Hence we define the following operators
$$\partial := \pi^{p+1, q}\circ d:\Omega^{p,q}\to \Omega^{p+1, q}$$
$$\bar{\partial} := \pi^{p, q+1}\circ d:\Omega^{p,q}\to \Omega^{p, q+1}$$
called \defn{Dolbeault operators}, where $\pi^{p, q}:\Omega^k_\complexnos (M)\to \Omega^{p, q}_\complexnos(M)$ is the projection map defined previously. 

We can express the action of the Dolbeault operators locally, given $\alpha := \sum_{|I|=p, |J|=q} f_{I,J} dz^I \wedge d\bar{z}^J \in \Omega^{p,q}_\complexnos(U)$, we have
$$\partial \alpha = \sum_{|I|=p, |J|=q} \sum_l \frac{\partial f_{I,J}}{\partial z^l} dz^l\wedge dz^I \wedge d\bar{z}^J$$
$$\bar{\partial} \alpha = \sum_{|I|=p, |J|=q} \sum_l \frac{\partial f_{I,J}}{\partial \bar{z}^l} d\bar{z}^l\wedge dz^I \wedge d\bar{z}^J$$

One can also show the following properties of Dolbealt operators
$$\partial +\bar{\partial} = d$$
$$\partial^2 = \bar{\partial}^2 = \partial\bar{\partial} + \bar{\partial}\partial = 0 $$
Note the second property $\bar{\partial}^2=0$ gives rise to a cohomology theory called \defn{Dolbeault cohomology} defined as
$$H^{p,q}_\complexnos (M) := \frac{\text{ker}(\bar{\partial}:\Omega^{p,q}\to \Omega^{p, q+1})}{\text{im}(\bar{\partial}:\Omega^{p,q-1}\to \Omega^{p, q})}$$

\subsubsection{Differential forms with vector coefficients}
We define the differential $p$-forms with coefficients in (a complex vector bundle) $E$ as 
\begin{align*}
\Omega^p(M, E) & :=\Gamma(M, \bigwedge {}^p(T^\ast M) \otimes E) \\
& \simeq \Hom(\bigwedge {}^p (TM), E)
\end{align*}
In other words the elements $\omega\in \Omega^p(M, E)$ at a point $x\in M$ are skew-symmetric multilinear forms 
$$\omega_x:\underbrace{T_xM\times T_xM\times \ldots T_xM}_{p \text{ times}} \to E_x$$
assigning a vector in the fibre $E_x$ to the $p$-tuples of tangent vectors at $x$. 

Note in particular that 
$$\Omega^0(M, E) = \Gamma(E)$$
$$\Omega^1(M, E) = \Gamma(T^\ast M \otimes E) \simeq \Hom(TM, E)$$

Note that locally (over $U$),  $0$-forms with coefficients in $E$ are local sections $s\in \Gamma(E, U)$ and hence can be expressed as an $r$-tuple of smooth maps $s=s(f)\in \smoothCmaps^r$ once a frame $f$ is fixed (or alternatively a local trivialisation is chosen),  as seen previously.  

Moreover,  $1$-forms with coefficients in $E$ can be expressed locally (over $U$) as an $r$-tuple of one forms.

\subsubsection{Covarient derivative of differential forms with vector coefficients}
We define the \defn{covarient derivative} $D:\Omega^k(E, M)\to \Omega^{k+1}(E, M)$ such that for $\alpha\in \Omega^k(M)$, $s\in \Gamma(E, U)$,
$$D(\alpha \otimes s)=d\alpha \otimes s + (-1)^k \alpha \wedge Ds$$

We note that for any $f\in \smooth(U)$,
$$D(\alpha \otimes fs) = D (f\alpha \otimes s)$$

Furthermore we obtain a general Liebniz rule
$$D(\alpha \wedge \gamma)=d\alpha \wedge \gamma + (-1)^k \alpha \wedge D\gamma$$
for $\alpha \in \Omega^k(M)$, $\gamma\in \Omega^l(E, M)$. 

\section{Connections}

\subsubsection{Connection definitions}

There are a few different equivalent ways to view connections.
We can define a \defn{connection} to be a bilinear map 
\begin{align*}
\nabla: & \Gamma(TM)\times \Gamma(E)\to \Gamma(E) \\
& (X, s)\mapsto \nabla_X s
\end{align*}
such that for all $f\in \smooth (M)$ we have
\begin{align*}
& \nabla_{fX}s=f\nabla_X s \\
& \nabla_X(fs)=(Xf)s + f\nabla_X s 
\end{align*}
Recall that $Xf\in \smooth(M)$ where $(Xf)(p)=X_p(f)\in \realnos$ for $p\in M$. (Note that the order here matters, as $fX\in \Gamma(TM)$ (as $\Gamma(TM)$ module over $\smooth(M)$) and $(fX)(p) = f(p)X(p)$, $p\in M$.) Note that if we fix 
$X\in \mathfrak{X}(M)$, we can think of the connection as an operator $\nabla_X:\Gamma(E)\to \Gamma(E)$.

We have an \defn{affine (or linear) connection} if $E=TM$ in the above definition.

We can also define a \defn{connection} as a $\complexnos$-linear map $\nabla : \Gamma(E)\to \Gamma(T^\ast M\otimes E)$ such that 
$$\nabla(fs)=df \otimes s + f\nabla s$$ 
for all $f\in C^\infty(M), s\in \Gamma(E)$.   

Note that $\Gamma(T^*M\otimes E)\simeq \Hom(TM, E)\simeq \Hom_{\smooth(M)}(\Gamma(TM), \Gamma(E)) = \Hom_{\smooth(M)}(\mathfrak{X}(M), \Gamma(E))$. Hence due to the last isomorphism, we can write for $s\in \Gamma(E), X\in \mathfrak(M)$, that $\nabla s(X)\in \Gamma(E)$. Moreover $\nabla s$ is $\smooth(M)$-linear, i.e. $\nabla s(fX)=f(\nabla s (X))$ for $f\in \smooth (M)$.

To show that these two definitions of a connection are equivalent, let us write $\nabla^1$ for the definition $\nabla^1:\Gamma(E)\to \Gamma(T^*M\otimes E)$ and $\nabla^2$ for the definition $\nabla^2:\Gamma(TM)\times \Gamma(E)\to \Gamma(E)$. If we are given $\nabla^1$, we can define $\nabla^2_X s:=\nabla^1 s(X)$, and it is easy to check the axioms of $\nabla^2$ are satisfied, namely $\complexnos$-bilinearity, the Liebniz rule, and that it is $\smooth(M)$-linear in the first argument. Conversely given $\nabla^2$, we can define $\nabla^1 s(X):=\nabla^2_X s$ and show that the axioms for $\nabla^1$ to be a connection are satisfied. 

Notice that we could have equivalently expressed the connection as $\nabla:\Omega^0(M, E)\to \Omega^1(M, E)$ in the language of differential $p$-forms with coefficients in $E$ (since we showed previously that $\Omega^0(M, E)=\Gamma(E)$, $\Omega^1(M, E)=\Gamma(T^*M\otimes E)$ by definition). This view agrees with the generalised definition of the exterior derivative.

\subsubsection{Example: Trivial connection on trivial line bundle}
As an example consider the trivial vector bundle of rank 1 (trivial line bundle)
$\pi:E\to M$, with $E=M\times \realnos$.
Note in this case a section on $E$ is a smooth function $\Gamma(E)=\Gamma(M\times \realnos)\simeq \smooth(M)$, since $s(p)\in \{p\}\times \realnos \simeq \realnos$ for $s\in \Gamma(E)$, $\forall p\in M$.
We claim that $\nabla: \Gamma(TM)\times \smooth(M)\to \smooth(M)$ such that $\nabla_X f := Xf$ is a connection on this bundle (called the \defn{trivial connection}). For this we need to show, it is $\realnos$-bilinear, and satisfies the two properties of a connection. 

$\nabla$ is $\realnos$-linear in the first argument since $\nabla_{X+Y}f=(X+Y)f=Xf+Yf=\nabla_Xf + \nabla_Yf$. Note that it is enough to consider $\nabla_{X+Y}$ instead of $\nabla_{aX+bY}$, $a,b\in \realnos$, since real numbers can be treated as constant functions and set $a=b=1$, and use the property $\nabla_{fX}s=f\nabla_X s$ of connections (we are yet to show). Also note that the equality $(X+Y)f=Xf+Yf$ follows from the fact that at every $p\in M$,
$((X+Y)f)(p)=(X+Y)_pf=(X_p+Y_p)f=X_pf+Y_pf=(Xf)(p)+(Yf)(p)=(Xf+Yf)(p)$.

To show that $\nabla$ is $\realnos$-linear in the second argument, we have $\nabla_X(af_bg)=X(af+bg)=aXf+bXg=a\nabla_Xf + b\nabla_X g$, ($f, g \in \smooth (M), a,b\in \realnos$) where $X(af+bg)=aXf+bXg$ is due to the $\realnos$-linearity of the space of point derivations. 

Further $\nabla_{fX}g=(fX)g=f(Xg)=f\nabla_X g$, and to make explicit the middle equality, for $p\in M$, $((fX)g)(p)=(fX)_p(g)=(f(p)X_p)(g)=f(p)X_p(g)=f(p)(Xg)(p)$.

Finally we see that $\nabla_X(fs)=X(fs)=fX(s)+X(f)s=f\nabla_Xs + (Xf)s$ where we have used the Liebniz rule on the product of two functions (since $s\in \Gamma(E)\simeq \smooth(M)$.

Hence we have shown that $\nabla$ is indeed a connection on the trivial line bundle. 

Let us also view this connection in terms of the definition $\nabla:\Gamma(E)\to \Gamma(T^*M\otimes E)$. Recall since $E=M\times \realnos$ in this case, then $\Gamma(E)\simeq \smooth (M)$. Now $\Gamma(T^*M\otimes E)\simeq \Hom_{\smooth(M)}(\Gamma(TM), \Gamma(E)) = \Hom_{\smooth(M)}(\Gamma(TM), \smooth(M)) \simeq \Gamma(TM)^* \simeq \Gamma(T^*M)=\Omega^1(M)$. So we have shown $\nabla:\smooth(M)\to \Omega^1(M)$ (or in other words $\nabla:\Omega^0(M)\to \Omega^1(M)$) on the trivial line bundle. Defining $(\nabla f)X=Xf$ for $f\in \smooth(M)$ as above, we observe this is simply the definition of the exterior derivative $d:\Omega^0(M)\to \Omega^1(M)$ (since $df(X)=Xf$ by definition). So we have that the trivial connection on the trivial line bundle is the exterior derivative (acting on smooth functions $\nabla f=df \in \Omega^1(M)$). 

\subsubsection{Example: Trivial connection on trivial bundle}
We now consider the trivial rank $k$ vector bundle, $E=M\times \realnos^k$ on $M$. In this case we notice that $\Gamma(E)=\Gamma(M\times \realnos^k)\simeq \smooth (M)^k$ since for a section $s\in \Gamma(E)$ and for $p\in M$ we have $s(p)\in E_p=\{p\}\times \realnos^k\simeq \realnos^k$, hence we can write $s(p)=(s^1(p),\ldots, s^k(p))\in \realnos^k$, and $s^i\in \smooth(M)$. Note that $s^i$ is smooth since it is a composition of smooth maps $s^i=\pi_i\circ s$, where $\pi_i$ is the $i$'th projection. Note also that $fs=(fs^1, \ldots, fs^k)$ for $f\in \smooth(M)$. Now we claim that $\nabla:\Gamma(TM)\times \smooth(M)^k\to \smooth(M)^k$ such that $\nabla_X s = (Xs^1, Xs^2, \ldots, Xs^k)$, where $s=(s^1, \ldots, s^k)\in \smooth (M)^k$ and $X\in \Gamma(TM)$ is a connection on the trivial bundle. The proof follows similarly to that of the trivial connection on the trivial line bundle. We call this connection the \defn{standard trivial connection} on the trivial $k$-bundle. 

Let us also view this connection in the definition $\nabla:\Gamma(E)\to \Gamma(T^*M\otimes E)$. Now $\Gamma(E)\simeq \smooth(M)^k$ on the trivial bundle. So  $\Gamma(T^*M\otimes E)\simeq \Hom_{\smooth(M)}(\Gamma(TM), \smooth(M)^k)\simeq \Gamma(TM^*)^k \simeq \Omega^1(M)^k$. Note that we have used the fact here that if $M$ is a module over a commutative ring $R$, then 
$$\Hom(M, R^k)\simeq \underbrace{M^*\oplus \cdots \oplus M^*}_{k \text{ times}} \simeq (M^*)^{k}$$
So we have shown that $\nabla:\smooth(M)^k\to \Omega^1(M)^k$. As before, for $f=(f^1, \ldots, f^k)\in \smooth(M)^k, X\in \Gamma(TM)$, let us define $(\nabla f)(X)=(Xf^1, \ldots, Xf^k)=(df^1(X), \ldots, df^k(X))=(df^1, \ldots, df^k)(X)$, noting that $(df^1, \ldots, df^k)\in \Omega^1(M)^k$.

\subsubsection{Example: Affine connection on $\realnos^n$}
Consider $M=\realnos^n$, $T_pM=\realnos^n$ for all $p\in M$.
The atlas consisting of a single chart $\text{id}=(u^1, \ldots, u^n)$ on $\realnos^n$, gives us a global frame $\{\frac{\partial}{\partial u^i}\}_i$ on $\realnos^n$. Hence we can write any sections $X, Y\in \Gamma(TM)$ as $X=\sum_{i=1}^n x_i \frac{\partial}{\partial u^i}$, $Y=\sum_{i=1}^n y_i \frac{\partial}{\partial u^i}$, for $x_i, y_i \in \smooth(M)$. We define an affine connection $\nabla:\Gamma(TM)\times \Gamma(TM)\to \Gamma(TM)$ such that 
\begin{align*}
\nabla_X Y & := \sum_{i=1}^n X(y_i) \frac{\partial}{\partial u^i} \\
& = \sum_{i, j} x_j\frac{\partial}{\partial u^j}(y_i)\frac{\partial}{\partial u^i}
\end{align*}
called the \defn{standard connection} on the tangent bundle to $\realnos^n$. It is straight forward to show that the axioms for a connection hold. In fact this is an example of a Levi-Civita connection.

\subsubsection{Levi-Civita connection}
Let $(M, g)$ be a Riemannian manifold. There exists a unique affine connection $\nabla:\mathfrak{X}(M)\times \mathfrak{X}(M)\to \mathfrak{X}(M)$ with the properties, for $X,Y,Z\in \mathfrak{X}(M)$,
$Xg(Y,Z)=g(\nabla_XY, Z)+g(Y, \nabla_XZ) (\in C^\infty(U))$ (compatibility with the metric) and $[X,Y]=\nabla_XY-\nabla_YX$ (torsion-free), called the \defn{Levi-Civita connection}. The uniqueness of the connection can be seen from the second Christoffel formula (derived below). 

Note that given a chart $(U, x^1, \ldots, x^n)$ of $M$, any affine connection $\nabla:\mathfrak{X}(M)\times \mathfrak{X}(M)\to \mathfrak{X}(M)$ is completely determined by $n^3$ coefficients $\Gamma^k_{ij}\in C^\infty (U)$ where
$$\nabla_{\partial_i}\partial_j=\sum_{k=1}^n \Gamma^k_{ij} \partial_k \in \mathfrak{X}(U)$$
Note that this fully determines the affine connection since any vector field can be written as $X=\sum_{i=1}^n \alpha_i \partial_i\in \mathfrak{X}(U)$ where $\alpha_i\in C^\infty (U)$, and since we know how to apply a function to each component of the connection (either by using Liebniz rule or that connections are affine connections are $C^\infty(U)$-linear in the first argument). In the case that $\nabla$ is a Levi-Civita connection, the $\Gamma^k_{ij}$ are called \defn{Christoffel symbols}.

Let $\nabla$ be a Levi-Civita connection. Let us rewrite the compatiblity with a metric and torsion-free conditions in terms of local coordinates, with respect to the local frame $\{\partial_i\}_{i=1}^n$ of $TU$. Firstly, that $\nabla$ is compatible with the metric:
\begin{align*}
    \partial_i g(\partial_j, \partial_k) & = g(\nabla_{\partial_i}\partial_j, \partial_k) + g(\partial_j, \nabla_{\partial_i}\partial_k) \\
    & = g(\sum_s \Gamma^s_{ij}\partial_s, \partial_k)+g(\partial_j, \sum_s \Gamma^s_{ik}\partial_s)\\
    &= \sum_s (\Gamma^s_{ij}g(\partial_s, \partial_k)+\Gamma^s_{ik}g(\partial_j, \partial_s)) \\ & \textrm{(since $g$ is $C^\infty(U)$ bilinear and $\Gamma^s_{ij}\in C^\infty(U)$)}
\end{align*}
In short $\partial_i g_{jk} = \sum_s (\Gamma^s_{ij}g_{sk}+\Gamma^s_{ik}g_{js})$. Now we find the torsion-free condition:
\begin{align*}
    [\partial_i, \partial_j] & = \nabla_{\partial_i}\partial_j - \nabla_{\partial_j}\partial_i \\
    & = \sum_s (\Gamma^s_{ij} - \Gamma^s_{ji})\partial_s
\end{align*}
and noting that the lie bracket $[\partial_i, \partial_j]=0$ for all $i,j$, we get $(\Gamma^s_{ij} - \Gamma^s_{ji})=0$, i.e. $\Gamma^s_{ij}=\Gamma^s_{ji}$.

Our goal now is to obtain a formula for the Christoffel symbols of a Levi-Civita connection, called the second Christoffel identity. We will first derive the first Christoffel identity to help us in this end:
\begin{align*}
    & \partial_i g_{jk} + \partial_j g_{ik} - \partial_k g_{ij} \\
    & = \sum_s [(\Gamma^s_{ij}g_{sk}+\Gamma^s_{ik}g_{js})+
    (\Gamma^s_{ji}g_{sk}+\Gamma^s_{jk}g_{is})-
    (\Gamma^s_{ki}g_{sj}+\Gamma^s_{kj}g_{is})] \\
    & = \sum_s 2\Gamma^s_{ij}g_{sk}
\end{align*}
where we used compatibility with the metric in the first step, and torsion-free and symmetry of the metric in the second step. Now to get the second Christorffel identity, we can contract with the inverse $(g^{tk})$ of the metric $g_{ij}$. Applying the contraction to the right hand side of the first Christoffel identity, we get
\begin{align*}
    \sum_k g^{tk}(2\sum_s \Gamma^s_{ij}g_{sk}) 
    & = 2\sum_s \Gamma^s_{ij}\sum_k g_{sk}g^{kt} \\
    & = 2 \sum_s \Gamma^s_{ij}\delta^t_s \\
    & = 2\Gamma^t_{ij}
\end{align*}
Whence we obtain the second Christoffel identity,
$$\Gamma^t_{ij} = \frac{1}{2}\sum_k g^{tk}(\partial_i g_{jk} + \partial_j g_{ik} - \partial_k g_{ij})$$

Note that an invarient formulation (not depending on a local chart) of the second Christoffel identity also exists called the Koszul formula (stated below), which is also easy to derive from the properties of the Levi-Civita connection.
\begin{align*}
2g(\nabla_XY, Z) = & X(g(Y,Z))+Y(g(X,Z))-Z(g(X,Y)) \\
& -g([Y,X],Z)-g([X,Z],Y)-g([Y,Z],X)    
\end{align*}

\subsubsection{Is a connection tensorial?}
We ask does the connection $\nabla:\Gamma(E)\to \Gamma(T^*M\otimes E)$ come from a bundle morphism $E\to T^*M\otimes E$? For it to be tensorial, we would require $\nabla(fs)=f\nabla s$. However in general (by the axiom for connections), $\nabla(fs)=f\nabla s + df\otimes s$, and $df\otimes s \neq 0$ in general, hence a connection is not tensorial.

Although a connection is not tensorial, we will see now that the difference of two connections is.

Let $\nabla^1, \nabla^2$ be two connections on $E$. Then $\nabla^1-\nabla^2:\Gamma(E)\to \Gamma(T^*M\otimes E)$. Now for $f\in \smooth(M)$, $s\in \Gamma(E)$
\begin{align*}
(\nabla^1 - \nabla^2)(fs)
& = \nabla^1(fs) - \nabla^2(fs) \\
& = (f\nabla^1 s + df \otimes s)
- (f\nabla^2 s + df \otimes s) \\
& = f\nabla^1 s - f\nabla^2 s \\
& = f(\nabla^1 - \nabla^2) s
\end{align*}
so $\nabla^1-\nabla^2$ is tensorial and comes from a bundle morphism $E\to T^*M\otimes E$. Loosely we have shown that $\nabla^1-\nabla^2\in \Hom (E, T^*M\otimes E)$ has a pointwise nature, which will allow us to express connections locally.

Now note that 
\begin{align*}
\Hom (E, T^*M\otimes E) & \simeq \Gamma(E^* \otimes T^*M\otimes E) \\
& \simeq \Gamma(T^*M\otimes E^* \otimes E) \\
& \simeq \Gamma(T^*M \otimes \Hom(E, E)) \\
& = \Gamma(T^*M \otimes \text{End}(E)) \\
& = \Omega^1(M, \text{End}(E))
\end{align*}
so $(\nabla^1-\nabla^2)_p\in T^*_pM\otimes \Hom(E_p, E_p)$ at $p\in M$. Or we can write $A:=\nabla^1-\nabla^2 \in \Omega^1(M, \text{End}(E))$, so that 
$A_p:T_p(M)\to \text{End}(E_p)$ for all $p\in M$. Or in other words for $s\in \Gamma(E), X\in \mathfrak{X}(M)$,
\begin{align*}
& (\nabla^1_X - \nabla^2_X)s = A_X s \\
\Rightarrow & \nabla^1_X s = \nabla^2_X s + A_X s \\
\end{align*}
where $A_X s$ at a point $p\in M$ reads as $(A_p)_{X_p} (s_p)$.

\subsubsection{Local expression of connections}
Let $E\to M$ be a (complex) vector bundle with a connection $\nabla:\Gamma(E)\to \Gamma(T^*M\otimes E)$.  Let $f=(e_1, \ldots, e_r)$ be a frame over $U\subseteq M$, with corresponding local trivialisation $\psi:\pi^{-1}(U)\to U\times \complexnos^r$. 
We want to find a local formula for $\nabla s$,  for an arbitrary section $s\in \Gamma(E, U)$.
Recall we can write $s$ locally as $s=(s^1, \ldots,  s^r)$,  for $s^i\in \smoothCmaps$,  with respect to the local frame $f=(e_1, \ldots e_r)$ (or equivalently a local trivialisation $\psi$).

Note that over $U$, any bundle is isomorphic to the trivial bundle $U\times \complexnos^r$. We have seen that the exterior derivative $d:\smooth(U)\to \Omega^1(U)$ is a connection on the trivial line bundle. We extend this definition of $d$ to the trivial rank $r$ bundle, by defining $d: \smooth(U)^r\to \Omega^1(U)^r$ componentwise as $d(s^1, \ldots, s^r) = (ds^1, \ldots, ds^r)$, which we also saw was a connection on $U\times \complexnos^r$. 

Hence over $U$, we have that $A:=\nabla-d\in \Omega^1(U, \End(E|_U)) = \Omega^1(U, \End(U\times \complexnos^r))$ is tensorial. Note that $\End(U\times \complexnos^r)=U\times M_r(\complexnos)$ (the trivial bundle of rank $r^2$). So we have that $A\in \Omega^1(U, \End(U\times \complexnos^r)) = \Omega^1(U, U\times M_r(\complexnos)) \simeq \Gamma(T^*U\otimes (U\times M_r(\complexnos))) \simeq \Hom(TU, U\times M_r(\complexnos)) \simeq \Hom_{\smooth(M)}(\mathfrak{X}(U), M_r(\smoothCmaps))$. Thus for any $p\in U$, we have a linear map
\begin{align*}
A_p : T_pM & \to \{p\}\times M_r(\complexnos) \simeq M_r(\complexnos) \\
v & \mapsto \begin{pmatrix}
A_{11}(v) & \cdots & A_{11}(v)\\
\vdots & \ddots & \vdots \\
A_{r1}(v) & \cdots & A_{rr}(v)
\end{pmatrix}
\end{align*}
with $A_{ij}\in T_p^*M$. Hence $A\in M_r(\Omega^1(U))$ is a matrix of one-forms on $U$. 

To see this same fact from a slightly different perspective, the connection $\nabla$ can be written locally with respect to the frame $f=(e_1, \ldots, e_r)$ as 
$$\nabla e_j = \sum_{i=1}^r A_{ij}(\nabla, f)\otimes e_i \in \Gamma(T^*U\otimes E|_U)$$
for some $A_{ij}:=A_{ij}(\nabla, f) \in \Omega_\complexnos^1(U)$. Then to find a local formula for $\nabla s$ for an arbitrary section $s=(s^1, \ldots,  s^r)\in \smoothCmaps\simeq \Gamma(E, U)$, we compute
\begin{align*}
\nabla s & = \nabla (\sum_{i=1}^r s^i e_i) \\
& = \sum_{i=1}^r (d s^i \otimes e_i + s^i \nabla e_i) \\
& = \sum_{i=1}^r (d s^i \otimes e_i + s^i (\sum_{k=1}^r A_{ki}\otimes e_k )) \\
& = \sum_{i=1}^r (d s^i + \sum_{j=1}^r s^j A_{ij})\otimes e_i 
\end{align*}
whence we have (in terms of matrix multiplication) that 
$\nabla s = \sum_{i=1}^r (d s(f) + As(f))\otimes e_i$.

To summarise, we have shown that we can write 
$$\nabla = d + A$$ 
where for $s\in \Gamma(E|_U)$,
\begin{align*}
\nabla s & = ds + As \\ 
(d+A)\begin{pmatrix}
    s^1 \\
    \vdots \\
    s^r
\end{pmatrix}
& = \begin{pmatrix}ds^1 \\ \vdots \\ ds^r\end{pmatrix} + \begin{pmatrix}
s^1A_{11} & \cdots & s^rA_{11}\\
\vdots & \ddots & \vdots \\
s^1A_{r1} & \cdots & s^rA_{rr}
\end{pmatrix} 
\end{align*}
for some $A_{ij}\in \Omega^1(U)$. 

Note using this formula in practice to apply $\nabla s$ to a (local) vector field $X\in \Gamma(TU)$, we get
\begin{align*}
    \nabla s (X) & = \left( \begin{pmatrix}ds^1 \\ \vdots \\ ds^r\end{pmatrix} + \begin{pmatrix}
s^1A_{11} & \cdots & s^rA_{11}\\
\vdots & \ddots & \vdots \\
s^1A_{r1} & \cdots & s^rA_{rr}
\end{pmatrix} \right) X \\
& = \begin{pmatrix}
ds^1 & s^1A_{11} & \cdots & s^rA_{11}\\
\vdots & \vdots & \ddots & \vdots \\
ds^r & s^1A_{r1} & \cdots & s^rA_{rr}
\end{pmatrix} X \\
& = \begin{pmatrix}
ds^1(X) & s^1A_{11}(X) & \cdots & s^rA_{11}(X)\\
\vdots & \vdots & \ddots & \vdots \\
ds^r(X) & s^1A_{r1}(X) & \cdots & s^rA_{rr}(X) 
\end{pmatrix} \in \Gamma(U\times \complexnos^r) = \smooth(U)^r
\end{align*}
If we want to compute the $A_{ij}(X)$ given $\nabla$, note that 
$$s^i = \begin{pmatrix}0 \\ \vdots \\ 0 \\ \mathbbm{1} \\ 0 \\ \vdots \\ 0\end{pmatrix}$$
under the isomorphism $\Gamma(E, U)\simeq \smooth(U)^k$, where $\mathbbm{1}:x\mapsto 1$ is the constant function, and is in the $i$'th position of the vector. Whence
\begin{align*}
    \nabla_X s^i & = d_X s^i + A_X s^i \\
    & = d_X \begin{pmatrix}0 \\ \vdots \\ 0 \\ \mathbbm{1} \\ 0 \\ \vdots \\ 0\end{pmatrix} +
     A_X \begin{pmatrix}0 \\ \vdots \\ 0 \\ \mathbbm{1} \\ 0 \\ \vdots \\ 0\end{pmatrix} 
     = \begin{pmatrix}A_{i1}(X) \\ A_{i2}(X) \\ \vdots \\ X\mathbbm{1} + A_{ii}(X)\\ A_{i, (i+1)}(X) \\ \vdots \\ A_{ik}(X)\end{pmatrix} 
     = \begin{pmatrix}A_{i1}(X) \\ A_{i2}(X) \\ \vdots \\ A_{ii}(X)\\ A_{i, (i+1)}(X) \\ \vdots \\ A_{ik}(X)\end{pmatrix} 
\end{align*}
where the last equality holds since the deriviative of a constant function is zero ($X\mathbbm{1}$=0).

\subsubsection{Example: Local expression for connection on the trivial line bundle}
As an example let us consider an arbitrary connection $\nabla$ on the trivial bundle $M\times \complexnos$ and understand its local formula. 

Recall that the exterior derivative $d$ is a connection on $M\times \complexnos$, and hence $\nabla - d$ is tensorial, and comes from a bundle morphism $M\times \complexnos\to T^*M\otimes E \simeq T^*M\otimes (M\times \complexnos)\simeq T^* M$, where the last isomorphism is a globally analogous to the linear isomorphism $V\otimes \complexnos\simeq V$ (where $V$ is a $\complexnos$-vector space). 
Now $\Hom(M\times \complexnos, T^*M)\simeq \Gamma((M\times \complexnos)^*\otimes T^*M)\simeq \Gamma((M\times \complexnos)\otimes T^*M) \simeq \Gamma(T^*M) = \Omega^1(M)$, (where we have used $M\times \complexnos \simeq (M\times \complexnos)^*$). So we have shown that $\nabla - d\in \Omega^1(M)$. To conclude,
$$\nabla = d + \omega$$
for $\omega \in \Omega^1(M)$, where $\omega(f)=f\omega \in \Omega^1(M)$, for $f\in \Gamma(E)=\smooth(M)$.

\subsubsection{Example: Local expression for connection on the trivial rank $r$-bundle}
In a similar fashion to the example above for a line bundle, one can show that for the trivial rank $r$ bundle, we have any connection $\nabla$ satisfies $\nabla - d\in \Omega^1(M)^r$, hence we can write
$$\nabla = d + A$$
for a matrix of $1$-forms $A\in \Omega^1(M)^r$.

\subsubsection{Hermitian connection}
Given a Hermitian vector bundle $(E, h)$, a connection $\nabla$ is a \defn{hermitian connection} with respect to $h$ if for any $X\in \Gamma(TM)$,
$$X(h(s, \sigma))=h(\nabla_X s, \sigma)+h(s,\nabla_X \sigma)$$
for $s, \sigma\in \Gamma(E, U)$ and noting that $h(s, \sigma)\in \smooth_\complexnos (U)$.

Equivalently, one could write this compatability condition as 
$$d(h(s, \sigma))=h(\nabla s, \sigma)+h(s,\nabla\sigma)$$
for $s, \sigma\in \Gamma(E, U)$, and $d:\smooth_\complexnos (U)\to \Omega_\complexnos^1(U)$ is the exterior differential map. Here to understand $h(\nabla s, \sigma)$ where $\nabla s\in \Gamma(T^* M\otimes E)\simeq \Gamma(T^*M)\otimes \Gamma(E)=\Omega^1_\complexnos(M)\otimes \Gamma(E)$, we need to extend the definition of $h$ as follows:
$$h(\alpha \otimes s, \sigma):=\alpha h(s, \sigma)$$
$$h(s, \alpha \otimes \sigma) := \bar{\alpha}h(s, \sigma)$$
for $\alpha \in \Omega_\complexnos^1(M)$ and $s, \sigma\in \Gamma(E)$.

For a hermitian connection $\nabla$ and $A\in \Omega^1(M, \End(E))$ we know $\nabla ' := \nabla + A$ is a connection (recall tensoriality discussion). Now $\nabla'$ is a hermitian connection if and only if
$$h(As, \sigma)+h(s, A\sigma)= 0$$
for all sections $s, \sigma \in \Gamma(E, U)$. (One can show from this that $A$ is contained in the Lie algebra of skew-hermitian matrices at each point, after  diagonalisation of $h$).

\subsubsection{Chern connection}
On a holomorphic vector bundle $E$ over a complex manifold $M$, with hermitian structure $h$, there exists a unique hermitian connection $\nabla$ compatible with the holomorphic bundle, called the \defn{Chern connection}. 

Note the Chern connection $\nabla:\Omega^0(E, M)\to \Omega^1(E, M)=\Omega^{1,0}(E, M) \oplus \Omega^{0,1}(E, M)$ splits as
$\nabla = \nabla' + \nabla''$ where
$$\nabla ': \Omega^0(E, M)\to \Omega^{1,0}(E, M)$$
$$\nabla '': \Omega^0(E, M)\to \Omega^{0,1}(E, M)$$
Then locally over a holomorphic frame $f$, we have that $\nabla=d+A$ splits as  
$$\nabla ' = \partial + A$$
$$\nabla '' = \bar{\partial}$$

Further, the condition that $\nabla$ is compatible with the metric $h$ can be written locally as $dh = hA + \bar{A}^Th$
which can in turn gives 
$\partial h = h A$ and
$\bar{\partial} h = \bar{A}^T h$ by comparing types. From this we get that
$$A= h^{-1}\partial h$$
So to summarise we have shown that 
\begin{align*}
    \nabla & = d + A \\
    & = (\partial + \bar{\partial}) + h^{-1}\partial h
\end{align*}

Finally we state without proof that if $A$ is the connection matrix of $\nabla$ then
$$\partial A= -A \wedge A$$

Note that the Chern connection (a connection conpatible with the holmorphic structure), is not the same as a holomorhpic connection, which has a more restrictive definition. We remark without further explanation that on the tangent bundle of a Kahler manifold, the Chern connection coincides with the Levi-civita connection of the underlying Riemannian metric.  

\subsubsection{Example: Chern connection on a holomorphic line bundle}
Let $E$ be a holomorphic line bundle. 
Since $E$ is rank $1$, at a point $p$, and a base vector $e \in E_p$ we have the hermitian inner product $h(e, e)$ is positive and real (by definition). Hence the matrix representing $h$ at $p$ is a positive real number. Thus over an open neighbourhood $U$, we have that locally $h\in \smooth(U)$ is a positive real function on this bundle. Thus locally the chern connection on this bundle is given by 
\begin{align*}
\nabla & = d + h^{-1}\partial h \\
& = d + \partial \text{log} (h)
\end{align*}
where the second equality follows from recalling calculus, i.e. differentiating the log function ($\frac{d}{dx} \text{log}(x) = \frac{1}{x}$) and thus the chain rule gives ($\frac{d}{dx} \text{log}(f(x)) = \frac{1}{f} f'(x)$ for a (positive, real) function $f$). 

\subsubsection{Example: Chern connection on $\complexnos P^n$}

Recall the Fubini-study metric $h_{i\bar{j}}$ defined previously on a standard chart $U\subseteq \complexnos P^n$ with coordinates $(w_1, \dots, w_n)$. We define the matrix representation of this metric as
\begin{align*}
H & := \frac{1}{2\pi} \left[  h_{i\bar{j}} \right] \\
& = \frac{1}{2\pi} \frac{1}{(1+|w|^2)^2}\begin{pmatrix}
1+|w|^2-|w_1|^2 & -\bar{w}_1w_2 & \cdots & -\bar{w}_1w_n \\
-\bar{w}_2w_1 & 1+|w|^2-|w_1|^2 & \cdots & -\bar{w}_2w_n \\
\vdots & \vdots & \ddots & \vdots \\
-\bar{w}_nw_1 & \cdots & \cdots & 1+|w|^2-|w_n|^2
\end{pmatrix} \\
& \in M_n(\smooth_\complexnos (U))
\end{align*}
where $|w|^2 = |w_1|^2 + \ldots + |w_n|^2$. 
Then the Chern connection locally on $U_i$ is given by 
$$\nabla = d + \bar{H}^{-1}(\partial \bar{H})$$
where $\partial \bar{H}$ means that $\partial$ is applied to each component of the matrix. 

\subsubsection{Connection on the dual bundle}
Let $\pi:E\to M$ be a vector bundle with connection $\nabla$.

We introduce some notation. For $\eta\in \Gamma(E^*), \sigma\in \Gamma(E)$, write
$$\langle \eta, \sigma \rangle := \eta(\sigma)\in \smooth (M)$$
(and recall that $\Gamma(E^*)=\Gamma(E)^* = \Hom_{\smooth(M)}(\Gamma(E), \smooth(M))$ is the dual module of $\Gamma(E)$). Note also that $\langle \eta, \sigma \rangle = \langle \sigma, \eta \rangle $ since it is analogous to thinking of vectors in $V$ as functionals on $V^*$, i.e. $V^{**}=V$.

We define a \defn{connection on the dual bundle} $E^*\to M$ as 
\begin{align*}
    \nabla^* :  \Gamma(TM)\times \Gamma(E^*) & \to \Gamma(E^*) \\
    (X, \eta) & \mapsto \nabla^*_X \eta
\end{align*}
such that for $X\in \Gamma(TM), \eta\in \Gamma(E^*), \sigma\in \Gamma(E)$,
$$X \langle \eta, \sigma \rangle := \langle \nabla^*_X \eta , \sigma \rangle + \langle \eta, \nabla^*_X \sigma \rangle$$
or in other words
$$ \langle \nabla^*_X \eta , \sigma \rangle := X \langle \eta, \sigma \rangle - \langle \eta, \nabla^*_X \sigma \rangle$$
To show this is indeed a connection, one would need to show that $\nabla^*_X \eta \in \Gamma(E^*)$ (i.e. that it is $\smooth(M)$-linear as a map $\Gamma(E)\to \smooth(M)$) and that $\nabla^*$ is $\smooth(M)$-linear in the first argument and satisfies Liebniz rule. 

Note that locally if $\nabla^1 = \nabla^2 + A$, then $(\nabla^1)^* = (\nabla^2)^* + A^*$.

\subsubsection{Connection on tensor product of bundles}
Let $E, F$ be vector bundles over $M$, with connections $\nabla^E, \nabla^F$ respectively.
Define a connection $\nabla$ on $E\otimes F$ as
\begin{align*}
    \nabla : \Gamma(TM)\times \Gamma(E\otimes F) & \to \Gamma(E\otimes F) \\
    (X, s\otimes \sigma) & \mapsto \nabla_X(s\otimes \sigma)
\end{align*}
such that
$$\nabla_X(s\otimes \sigma):=(\nabla^E_X s)\otimes \sigma + s \otimes \nabla^F_X \sigma $$

\subsubsection{More induced connections}
Let $E, F$ be vector bundles over $M$, with connections $\nabla^E, \nabla^F$ respectively. 

For $s\in \Gamma(E), \sigma \in \Gamma(F)$ define \defn{a connection $\nabla$ on $E\oplus F$} as
$$\nabla(s\oplus \sigma)=\nabla^E(s)\oplus \nabla^F(s)$$

We can define a \defn{connection $\nabla$ on $\Hom(E, F)$} locally as
$$\nabla(f)(s)=\nabla^F(f(s))-f(\nabla^E(s))$$
where $f\in \Hom(E, F)|_U$ ($f$ local over $U$), and $s\in \Gamma(U, E)$. Note that $f(s)\in \Gamma(U, F)$, so we can apply $\nabla^F$ to this. To understand $f(\nabla^E(s))$ in this formula, we note that $\Hom(E, F)\simeq \Gamma(E^*\otimes F) \simeq \Omega^1(E)\otimes \Gamma(F)$ and define $f(\alpha\otimes t)=\alpha \otimes f(t)$ for $\alpha \in \Omega^1(E)$ and $t\in \Gamma(F)$.

If $\nabla = d+A$ is a connection on $U\subseteq N$ locally and $f:M\to N$ is smooth map of manifolds, then the \defn{pull back connection} is a connection on $M$ locally given by $f^*\nabla |_{f^-1(U)}=d+f^*A$.

\section{Curvature}

To motivate curvature, we not that a connection $\nabla$ is not a differential (in other words $\nabla^2\neq 0$ in general). We note that the obstruction to $\nabla$ being a differential is the curvature, which is defined with respect to a connection. 

\subsubsection{Classical curvature form}
Given a connection $\nabla$ on $E$, define the \defn{curvature}
\begin{align*}
R^\nabla : & \Gamma(TM)\times \Gamma(TM)\times \Gamma(E)\to \Gamma(E) \\
& (X, Y, s) \mapsto \nabla_X(\nabla_Y s) - \nabla_Y(\nabla_X s) - \nabla_{[X, Y]} s
\end{align*}
which one can show is $\smooth (M)$-trilinear.
A useful identity in relation to curvature (which follows from direct computation) is
$$R^\nabla (X, Y, s)=-R^\nabla (Y, X, s)$$

\subsubsection{Curvature of the trivial line bundle}
Consider the trivial line bundle $E=M\times \complexnos$. Recall that $\nabla:=d:\smooth(M)\to \Omega^1(M)$ is a connection on this bundle, where $\nabla_X f = \nabla f(X) = df(X) = Xf$, for $f\in \smooth(M)$. 
So for all $s\in \Gamma(M\times \complexnos)\simeq \smooth(M)$, we have
\begin{align*}
    R^\nabla(X, Y, s) & = \nabla_X \nabla_Y s - \nabla_Y \nabla_X s - \nabla_{[X, Y]}s \\
    & = XYs - YXs - [X, Y]s \\
    & = 0
\end{align*}
Since the curvature is identically zero, we say that this connection on the trivial bundle is a \defn{flat connection}.

\subsubsection{Curvature of Levi Civita connection on $T\realnos^n$}
On $\realnos^n$ we have global coordinates $(x^1, \ldots, x^n)$. 
Let
$X=\sum_i X^i \frac{\partial}{\partial x^i}$, $Y=\sum_i Y^i \frac{\partial}{\partial x^i}$, $Z=\sum_i Z^i \frac{\partial}{\partial x^i}$ be vector fields on $T\realnos^n$. 
Recall we define the Levi-Civita connection on $T\realnos^n$ as
$\nabla_X Y = \sum_i X(Y^i) \frac{\partial}{\partial x^i}$. Then 
\begin{align*}
      R^\nabla(X, Y, Z) & = \nabla_X \nabla_Y Z - \nabla_Y \nabla_X Z - \nabla_{[X, Y]} Z \\
      & = \sum_i (X(Y(Z^i))-Y(X(Z^i))-[X, Y](Z^i))\frac{\partial}{\partial x^i} \\
      & = 0
\end{align*}
Hence the Levi-Civita connection on $T\realnos^n$ is flat. 

\subsubsection{General curvature form}
Given a connection $\nabla$, define the \defn{curvature form} as
$$F_\nabla := D^2 = D\circ D: \Omega^0(E, M)\to \Omega^1(E, M) \to \Omega^2(E, M)$$
where $D:\Omega^k(E, M)\to \Omega^{k+1}(E, M)$ is the covarient derivative (defined earlier). 

One can show that $F_\nabla$ is $\smooth(M)$-linear and so note $F_\nabla \in \Omega^2(\End(E), M)$. 

One can show that the general curvature form $F_\nabla$ corresponds with the classical curvature form $R^\nabla$, defined previously.

\subsubsection{Example: Curvature of connections on the trivial bundle}
Recall that $d$ is the trivial connection on the trivial bundle $M\times \complexnos^r$. Then the curvature associated with this connection is
$F_d = d^2 = 0$.

Note that any other connection on $M\times \complexnos^r$, is of the form $\nabla = d+A$ (globally) for $A\in M_r(\Omega^1(M))$. Hence (by applying the discussion on local expressions below on $M\times \complexnos^r$), the curvature form is (globally) given by $F_\nabla = dA+A\wedge A$. 

\subsubsection{Curvature form local expression}
Let $A\in M_r(\Omega^1(U))$ be the connection matrix associated to a connection $\nabla$, over a local frame $f=(f^1, \ldots , f^r)$. In other words say $\nabla=d+A$ locally over $f$. Then locally, the curvature form $F_\nabla$ is given by a matrix $\theta \in M_r(\Omega^2(M))$ such that
$$\mathbb{\theta}_{ij} := dA_{ij} + \sum_k A_{ik}\wedge A_{kj}$$
or in short
$$\mathbb{\theta} := dA + A\wedge A$$
To prove this we note that $F_\nabla = \nabla^2 = \nabla \circ \nabla$ and so locally for $s\in \Gamma(E, U)=\Gamma(U\times \complexnos^r)\simeq \smooth(U)^r$
\begin{align*}
    (d+A)(d+A)v & = d^2 v + A dv + d(Av) + (A\wedge A) v \\
    & = A dv + (dA) v - A  dv + (A\wedge A) v \\
    & = (dA) v  + (A\wedge A) v \\
    & = (dA + A\wedge A) v
\end{align*}

Note that if $g:U\to GL_r(\complexnos)$ is a change of frame, so that $f'=fg$ is a new frame, we have the following transformation law
$$\mathfrak{\theta}(fg)=g^{-1} \mathbb{\theta}g$$

One can show that for a holomorphic bundle $E$ with hermitian structure $h$, and Chern connection $\nabla$, the associated curvature $F_\nabla$ is a $(1,1)$-type, real, skew-hermitian form. In other words $F_\nabla\in \Omega^{1,1}_\realnos (\End(E), M)$. Hence locally, if $A$ is the connection matrix (so $\nabla = d+A$), then locally the associated curvature is given by
\begin{align*}
F_\nabla & = dA + (A\wedge A) \\
& = (\partial + \bar{\partial})A + (A\wedge A) \\
& = \bar{\partial} A
\end{align*}
where the last equality holds by comparing types.

\subsubsection{Bianchi identity}
Suppose $F_\nabla\in \Omega^2(\End(E), M)$ is the curvature associated to a connection $\nabla$ on a vector bundle $E$. Suppose that $\tilde{\nabla}$ is the natural connection on $\End(E)$ induced from $\nabla$. Then the Bianchi identity states   
$$\tilde{\nabla}(F_\nabla)=0$$
and note that $\tilde{\nabla}(F_\nabla)\in \Omega^3(\End(E), M)$.

Note that for the Chern connection on a holomorphic hermitian bundle, the Bianchi identity allows us to define a cohomology theory with cohomology classes $[F_\nabla]$ belonging to a Dolbeault cohomology class. One can further show that this does not depend on the choice of connection, i.e. $[F_\nabla]=[F_{\nabla+a}]$, where $\nabla, \nabla+a$ are Chern connections with respect to the hermitian metric. 

\subsubsection{Curvature of induced bundles}
Let $E_1, E_2$ be vector bundles with connections $\nabla^1, \nabla^2$ respectively, and associated curvature forms $F_{\nabla^1}, F_{\nabla^2}$. We can define curvature forms of various bundles in terms of the curvature of these bundles. 

Namely, on $E_1\oplus E_2$ we get a  curvature form
$$F = F_{\nabla^1}\oplus F_{\nabla^2}$$

On $E_1\otimes E_2$ we get the curvature form
$F=F_{\nabla^1}\otimes 1 + 1\otimes F_{\nabla^2}$

On $E^*$ (with induced connection $\nabla^*$) we get
$$F_{\nabla^*}=-F_{\nabla}^T$$

On the pull back connection $f^*\nabla$ (where $f:M\to N$ is a smooth map of manifolds), we get
$$F_{f^* \nabla} = f^* F_\nabla$$

\subsubsection{Example: Cuvature associated with Chern connection on $T (\complexnos P^n)$}

Recall the Chern connection on $\complexnos P^n$ with hermitian structure $H$ given by the fubini-study metric, is given locally by $\nabla = d+\bar{H}^{-1}\partial(\bar{H})$. Let $A=\bar{H}^{-1}\partial(\bar{H})$, then the curvature associated to this Chern connection is locally 
\begin{align*}
    F_\nabla & = dA + A\wedge A \\
    & = \bar{\partial} A + \partial A + A\wedge A \\
    & = \bar{\partial}(\bar{H}^{-1}\partial(\bar{H})) + \partial(\bar{H}^{-1}\partial(\bar{H})) + \bar{H}^{-1}\partial(\bar{H}) \wedge \bar{H}^{-1}\partial(\bar{H}) \\
    & = \bar{\partial}(\bar{H}^{-1}\partial(\bar{H}))
\end{align*}
where the last equality holds, by comparing types. More precisely recall that $A$ is a matrix of $(1, 0)$ forms, so $\partial A$ is a $(2,0)$-form, and $A\wedge A$ is a $(2,0)$-form, and so are zero as curvature on a holomorphic bundle with hermitian structure, is a $(1,1)$ form.

\subsubsection{Example: Curvature of tautological line bundle on $\complexnos P^n$}
Consider the standard coordinates $(z^0, \ldots, z^n)\in \Gamma(\mathcal{O}(1))$ as sections of the dual bundle $\mathcal{O}(1)$ to the tautological line bundle $\mathcal{O}(-1)$ on $M:=\complexnos P^n$. 

 Recall we have a hermitian metric $h=(\sum_i |z^i|^2)^{-1}=(\sum_i 1+|w^i|^2)^{-1}$ on this bundle. Then one can show that the associated curvature to the Chern connection $\nabla$ on this bundle is
$$F_\nabla = \frac{2\pi}{i} \omega_{FS} $$
where $\omega_{FS}$ is the Fubini-study form. Recall the connection matrix $A$ on a holomorphic line bundle with hermitian structure is given by $A = \partial \text{log}(h)$, where $h$ is the hermitian metric (and a positive real function on the line bundle). Hence locally we have that the curvature form is given by 
\begin{align*}
F_\nabla = \bar{\partial}A & =  \bar{\partial}\partial \text{log}(h) \\
& = \bar{\partial}\partial \text{log}(\sum_i 1+|w^i|^2)
\end{align*}

Note that this can be generalised to any holomorphic line bundle on any complex manifold $M$ using the pull-back of $\phi:M\to \complexnos P^n$ to give
$$\frac{i}{2\pi}F_\nabla = \phi^* \omega_{FS}$$
where $\omega_{FS}$ is the Fubini-study form on $\complexnos P^n$.

\subsubsection{Example: Connection matrix of the trivial line bundle is a closed real form}
We state without proof that if $\nabla$ is a hermitian connection on a hermitian vector bundle $(E, h)$, then $F_\nabla$ satisfies $h(F_\nabla s, \sigma)+h(s, F_\nabla \sigma)=0$. In other words $F_\nabla\in \Omega^2(\End(E), M)$.

Recall that locally a hermitian connection $\nabla=d+A$ has the property $\bar{A}^T=-A$ for its connection matrix. Then the associated curvature is locally $F_\nabla = dA+A\wedge A$ and we have $\bar{F_\nabla}^T = d\bar{A}^T + \overline{(A\wedge A)}^T$. Then by the result above one can see that $F_\nabla + \bar{F_\nabla}^T = 0$.

Consider the trivial line bundle $E=M\times \complexnos$, endowed with the constant hermitian structure. Let $\nabla=d+\omega$ be a connection on $E$, for some $\omega \in \Omega^1(M)$. Now for any one form $\omega$ we have that $\omega \wedge \omega =0$ (note which does not hold for forms of higher degree). Using this and the fact that $\bar{\omega}^T = \bar{\omega}$, since $\omega$ is considered a $1\times 1$ matrix, we get that
\begin{align*}
F_\nabla + \bar{F_\nabla}^T & = d(\omega + \bar{\omega}^T) + (\omega\wedge \omega) + (\bar{\omega}^T \wedge \bar{\omega}^T) \\
& = d(\omega + \bar{\omega}^T) \\
& = d(\omega + \bar{\omega}) \\
& = d(2\text{Re}(\omega)) \\
& = 0
\end{align*}
Hence we have shown that $\text{Re}(\omega)=\omega + \bar{\omega}^T$ is closed. 

So in particular we have that $\nabla = d+\omega$ (and $F_\nabla = d\omega + \omega\wedge \omega$), where $\text{Re}(\omega)$ is closed.

\section{Chern classes}
Let $E\to M$ be a (complex) vector bundle of rank $k$ (over $\complexnos$). The $i$'th Chern class is an element of de Rham cohomology
$c^i(E)\in H^{2i}_{dR}(M)$. Our objective is to understand and compute these cohomology classes, in particular on complex projective space.

\subsubsection{De Rham cohomology}
An $r$-form $\omega\in \Omega^r(M)$ is closed if $d\omega=0$ and is exact if there exists a $\eta \in \Omega^{r-1}(M)$ such that $\omega = d\eta$. The closed $r$-forms are denoted $Z^r(M)$ and exact $r$-forms are denoted $B^r(M)$. Now since $d^2=0$, so that $B^r(M)\subseteq Z^r(M)$, we can define the $r$'th \defn{de Rham cohomology} group as 
$$H_{dR}^r(M):=\frac{Z^r(M)}{B^r(M)}$$
Thus two closed forms $\omega_1, \omega_2$ have equivalent cohomology class $[\omega_1]=[\omega_2]$ if $\omega_1 - \omega_2 = d\eta$ for some $\eta \in \Omega^{r-1}(M)$. 

Now $H^*_{dR}(M) = \bigoplus_i^n H^i_{dR}(M)$ is a graded $\realnos$-algebra, with product $\cup:H^p_{dR}(M)\times H^q_{dR}(M)\to H^{p+q}_{dR}(M)$. 

Recall that $H^m_{dR}(M)=0$ if $m>\text{dim}_\realnos M$. Recall if $\omega \in H^{i}_{dR}(M)$, $\eta \in H^{j}_{dR}(M)$, then $\omega \eta = (-1)^{ij} \eta \omega$.

Note that de Rham's theorem says that singular cohomology and de Rham cohomology are equivalent $H(M; \realnos)\simeq H_{dR}(M)$ as cohomology rings.

\subsubsection{Cohomology of $\complexnos P^n$}
Note that $\dim_\realnos \complexnos P^n = 2n$, so that $\Omega^{2n}(M)$ are top forms.  

A fact we will require is the cohomology classes of $\complexnos P^n$ is a $1$-dimensional vector space for even degree, and zero otherwise. In other words,
$$H^k_{dR}(\complexnos P^n)\simeq \begin{cases}
\realnos, &  k \text{ even} \\ 
\{0\}, & k \text{ odd}
\end{cases}$$

Another required fact is that there exists a generator in the second de Rham cohomology class that gives an element in each of the other even degree cohomology classes. To be precise, there exists $h\in H^2_{dR}(\complexnos P^n)$ such that 
$$h^k \in H^{2k}_{dR}(\complexnos P^n), \text{ for } 0\neq h^k := \underbrace{h\wedge h \wedge \cdots \wedge h}_{k \text{ times}}$$
(Note also that since even degree cohomology classes are $1$-dimensional vector spaces, each generator $h^k$ is a basis for $H^{2k}_{dR}(\complexnos P^n)$).

Using this fact we get the $\realnos$-algebra isomorphism
$$H^*_{dR}(\complexnos P^n) \simeq \frac{\realnos [h]}{(h^{n+1})}$$
where $\text{deg}(h)=2$. (So note $h^{n+1}=0$, since $h^n\in H_{dR}^{2n}(M)$ is of top degree). 

Now we can choose the generator $h\in H^2_{dR}(\complexnos P^n)$, so that $h^n\in H^{2n}_{dR}(\complexnos P^n)$ is in top cohomology (i.e. of degree $2n$), and so that if $h^n=[\omega]$, then $\int \omega = 1$. Furthermore note that by Stokes theorem, this integral does not depend on the choice of representative $\omega$ of $h^n$. 

Using this we find that if we took another generator $h'\in H^{2}_{dR}(\complexnos P^n)$ with $(h')^n=[\omega ']$. Then $(h')^n=ah^n$ for some $a\in \realnos$, since recall $h$ is a basis element of the top cohomology class (which is a real one-dimensional vector space). In other words $[\omega ']=a[\omega] = [a\omega]$. Thus we get that $\int \omega' = \int a\omega = a \int \omega = a\cdot 1 = a$. 

To summarise we have shown that for any top form $\omega'\in \Omega^{2n}(\complexnos P^n)$, we can obtain a generator $h^k$ for the cohomology classes $H^{2k}_{dR}(\complexnos P^n)$ specified by 
$$\int \omega' \in \realnos$$
where $[\omega ']=h^{2n}\in H^{2n}_{dR}(\complexnos P^n)$ is a ($1$-dimensional) basis for the top cohomology class.

\subsubsection{Invariant polynomials}

A $k$-multi-linear form $$\tilde{\phi}:\underbrace{M_r(\complexnos)\times \cdots \times M_r(\complexnos)}_{k \text{ times}}\to \complexnos$$
is \defn{invariant} if for any $g\in GL_r(\complexnos)$
$$\tilde{\phi}(A_1, \ldots, A_k) = \tilde{\phi}(gA_1g^{-1}, \ldots, gA_kg^{-1})$$
We write $\tilde{I}_k(M_r(\complexnos))$ for the $\complexnos$-vector space of all \defn{invariant $k$-multilinear forms} on $M_r(\complexnos)$. 

Recall a polynomial is said to be \defn{homogeneous} if all its monomials are of the same degree (for example $x^5 + 7xy^4 + x^3y^2$ is homogeneous of degree 5). Now $\tilde{\phi}\in \tilde{I}_k(M_r(\complexnos))$ induces a homogeneous polynomial $\phi$ of degree $k$ (with entries in $A$) given by $\phi:M_r(\complexnos)\to \complexnos$ such that
$$\phi(A):=\tilde{\phi}(A, A, \ldots, A)$$
and we note then that $\phi(gAg^{-1})=\phi(A)$ (for $g\in GL_r(\complexnos)$), hence we call $\phi$ an \defn{invariant polynomial}. Note that $\phi$ is called the \defn{polarized} form of $\tilde{\phi}$. We write $I_k(M_r(\complexnos))$ for the \defn{space of invariant polynomials of degree $k$}. 

Converesly, given an invariant polynomial $\phi \in I_k(M_r(\complexnos))$, we can get an invariant k-multilinear form $\tilde{\phi} \in \tilde{I}_k(M_r(\complexnos))$, by defining
$$\tilde{\phi}(A, \ldots, A):=\phi(A)$$

An example of a (degree $r$) invariant polynomial is the determinant function
$$\text{det}:M_r(\complexnos)\to \complexnos \in I_r(M_r(\complexnos))$$
and furthermore, the map
$$\text{det}(I+A)=\sum_{k=0}^r \gamma_k(A) $$
give invariant polynomials $\gamma_k\in I_k(M_r(\complexnos))$, where $I$ is the identity, $A \in M_r(\complexnos)$.

\subsubsection{Chern-Weil theory}
Given an invarient polynomial $\phi$, our goal is now to define $\phi(F_\nabla)$ as a $2k$-form which is closed. 

Consider $\mathfrak{g}=\mathfrak{gl}_r(\complexnos)\simeq M_r(\complexnos)$, with a basis $(e_1, \ldots, e_N)$, $N=r^2$. Let $\phi:\mathfrak{g}\to \complexnos$ be an invariant homogeneous polynomial of degree $d$. (So we consider all multi-indices $v=(v_1, \ldots, v_N)$ such that $|v|=\sum_i v_i = d$). Then $\phi$ has the form 
$$\phi(x)=\sum_{|v|=d} a_v x^v, \ \ x^v := (x^1)^{v_1}(x^2)^{v_2}\cdots (x^N)^{v_N}$$
where $x^i\in \complexnos$ is the $i$'th component of $x\in \mathfrak{g}$ with respect to the basis $(e_1, \ldots, e_n)$, i.e. $x=\sum_i x^i e_i$. 

Let $\pi:E\to M$ be a vector bundle with local trivialisations 
$\psi_\alpha:\pi^{-1}(U_\alpha)\to U_\alpha \times \complexnos^k$. Let $\nabla$ be a connection on $E$ with associated curvature $F_\nabla$, locally given by $F_\alpha \in \Omega^2(U_\alpha, \mathfrak{g})$ such that 
$F_\alpha = \sum_{i=1}^N \omega^i_\alpha e_i$
for some $\omega^i_\alpha\in \Omega^2(U_\alpha)$. 

We define
$$\phi(F_\nabla)|_{U_\alpha} : = \sum_{|v|=d} a_v\omega^v_\alpha$$
where $\omega^v_\alpha = (\omega^1_\alpha)^{v_1} \wedge (\omega^2_\alpha)^{v_2} \wedge \cdots \wedge (\omega^N_\alpha)^{v_N} $. 
We remark here that since the $\omega^i_\alpha$ are all of even degree, the wedge product in this case commutes $\omega^i_\alpha \wedge \omega^j_\alpha = \omega^j_\alpha \wedge \omega^i_\alpha$, and so can be written in any order.

Now due to the invariance of the polynomial $\phi$, one can show that the above definitions agree on the intersections $U_\alpha\cap U_\beta$, and further $\phi(F_\nabla)$ is independent of the choice of basis $(e_i)$ used. Hence we have that $\phi(F_\nabla)\in \Omega^{2d}(M)$ is a global even degree form. 

Let $P$ be an invarient polynomial on $\mathfrak{gl}(r, \complexnos)$. Then
$$P(F_\nabla)\in \Omega^{2d}_\complexnos(M)$$
is closed. Hence we can define cohomology classes
$$[P(F_\nabla)]\in H^{2d}(M, \complexnos)$$
which one can show is independent of the chosen connection $\nabla$, i.e.
$$[P(F_\nabla)]=[P(F_{\tilde{\nabla}} )]$$
for any connections $\nabla, \tilde{\nabla}$. 

Note that in the language of principle $G$-bundles, we can consider invarient polynomials from any Lie algebra $\mathfrak{g}:=Lie(G)\subseteq \End(V)$ of the structure group $G\subseteq GL(V)$ (which is a Lie group) of a local trivialisation, and then $\nabla$ becomes a $G$-connection, and $F_\nabla$ its associated curvature, in the definitions of $p(F_\nabla)$ above. 

In fact from this we can obtain an algebra homomorphism from invarient polynomials over the structure group into the graded algebra of even degree cohomology classes $H^{2*}_{dR}(M)$ called the \defn{Chern-Weil homomorphism}. To be precise, fixing a vector bundle $E$ with connection $\nabla$, this is the algebra homomorphism given by
\begin{align*}
    c_E : \text{Inv}(\mathfrak{gl}_r(\complexnos)) & \to H^* (M) \\
     P(X) & \mapsto [P(F_\nabla)]
\end{align*}
and sends $P(X)Q(X)\xmapsto{c_E} [P(F_\nabla)\wedge Q(F_\nabla)]$, (noting that $\text{Inv}(\mathfrak{gl}_r(\complexnos))$ denotes invariant polynomials on $\mathfrak{gl}_r(\complexnos)$).

\subsubsection{Definition of Chern class}
We now consider specific invarient polynomials, to define Chern classes and other useful topological invarients. 

We define $\{P_k\}$ to be the homogeneous invariant polynomial of $\text{deg}(P_k)=k$ defined by 
$$det(I+B)=1+P_k(B)+\ldots +P_r(B)$$
where $I$ is the identity, and $B\in M_k(\complexnos)$. Define the \defn{Chern forms} on a rank $r$ vector bundle $E$ endowed with a connection $\nabla$ as 
$$c_k(E, \nabla):=P_k \left( \frac{i}{2\pi} F_\nabla \right) \in \Omega^{2k}_\complexnos (M)$$

The $k$'th \defn{Chern class} of $E$ is then the induced cohomology class of the respective Chern form, i.e. 
$$c_k(E):= [c_k(E, \nabla)] \in H^{2k}(M, \complexnos)$$

One can note that $c_0(E)=1$ and $c_k(E)=0$ for all $k>\text{rank}(E)$. 

We define the \defn{total Chern class} as 
\begin{align*}
c(E)& :=c_0(E)+c_1(E)+\ldots +c_r(E) \\
& = \text{det} \left( I + \frac{i}{2\pi} F_\nabla \right)
\end{align*}
noting that $c_i(E)\in H^{2i}(M, \complexnos)$.

One can show that in fact a Chern form $c(E, \nabla)$ is a real differentiable form, from which it follows the Chern class $c(E)\in H^*(X, \realnos)\subseteq H^*(X, \complexnos)$.  Hence by recalling the de Rham theorem ($H^*(X, \realnos)\simeq H^*_{dR}(M)$), we have $c(E)\in H^{2*}_{dR}(M)$.

\subsubsection{Computing Chern classes}
To make it less cumbersome to expand the determinant and compute Chern classes, we find formulas in terms of the traces for the $k$-th Chern class. 

We start by diagonalising the curvature form $F_\nabla$ by finding a matrix $g\in GL_r(\complexnos)$ such that
$$\Theta := g^{-1}\left( \frac{i}{2\pi} F_\nabla \right) g = \text{diag}(\omega_1, \ldots, \omega_r)$$
for some $2$-forms $\omega_i$. 

For example, if we choose an anti-Hermitian generator $g\in SU(r)\subset GL_r(\complexnos)$, then we have
\begin{align*}
    \det(I+\Theta) &= \det (\text{diag}(1+\omega_1, \ldots, 1+\omega_r)) \\
    & = \prod_{i=1}^r (1+\omega_i) \\
    & = 1 + (\omega_1 + \ldots + \omega_r) + (\omega_1\omega_2 +\ldots + \omega_{r-1}\omega_r) + \ldots (\omega_1\omega_2\cdots \omega_r) \\
    & = 1 + \tr(\Theta) + \frac{1}{2}(\tr(\Theta)^2 - \tr(\Theta^2)) + \ldots + \det (\Theta) 
\end{align*}
and note that each summand is an elementery symmetric function of the $\omega_i$.
Now since $\det(I+\Theta)$ is an invarient polynomial we have $\det(I+\Theta)=\det(I+g\Theta g^{-1})=\det(I+g (g^{-1} (\frac{i}{2\pi} F_\nabla) g) g^{-1}) = \det (I+ \frac{i}{2\pi} F_\nabla)$. 
Or in other words for the general curvature $F_\nabla$, we have 
$$\det(I+F_\nabla)=\det(I+\frac{2\pi}{i}\Theta)$$
$$\tr(F_\nabla)= \tr( \frac{2\pi}{i} \Theta)= \frac{2\pi}{i} \tr(\Theta)$$
This then gives us the Chern classes with respect to the curvature form $F_\nabla$ as
\begin{align*}
    c_0 & = 1 \\
    c_1 & = \tr(\Theta) = \frac{i}{2\pi} \tr(F_\nabla)
    \\
     c_2 & = \frac{1}{2}(\tr(\Theta)^2 - \tr(\Theta^2))  = \frac{1}{2} \left( \frac{i}{2\pi} \right)^2 (\tr(F_\nabla)\wedge\tr(F_\nabla) - \tr(F_\nabla\wedge F_\nabla)) \\
    \vdots & \\
     c_r & = \det(\Theta)=\left( \frac{i}{2\pi} \right)^r \det(F_\nabla) 
\end{align*}

\subsubsection{Properties of Chern classes}
From the definition, it is possible to derive various useful properties of Chern classes. We now list these properties.
\begin{enumerate}
    \item Given two isomorphic bundles $E\simeq F$ over $M$, then $c(E)=c(F)$ (i.e. $c_i(E)=c_i(F), \forall i$).

    \item Triviality:

    If $E=M\times \complexnos^k$ is the trivial bundle then $c(E)=1$ (i.e. $c_0(E)=1\in H^0_{dR}(M)\simeq \realnos$ (if $M$ connected) and $c_1(E)=c_2(E)=\ldots =c_k(E)=0$). 

    \item Whitney Sum:

    If $E, F$ are two bundles over $M$ then
    $c(E\oplus F) = c(E) \cup c(F)$
    
    Recall that $\cup$ is multiplication in $H^*_{dR}(M)$ defined as
    \begin{align*}
        c(E) \cup c(F) & = (c_0(E)+\ldots + c_{k}(E)) \cup (c_0(E)+\ldots + c_{s}(F)) \\
        & = (c_0(E) c_0(F)) + (c_0(E)\cup c_1(F) + c_1(E)\cup c_0(F)) + \ldots
    \end{align*}
    where in the last equality the first summand is degree 0, the second is degree 2, etc. 
    
    \item Naturality:

    Let $F:M\to N$ be a smooth map of manifolds, and $E\to N$ a vector bundle over $N$. Recall the pull back bundle $F^*E$ is a bundle over $M$, with fibres $(F^*E)_p=E_{F(p)}$ for $p\in M$. 
    
    Then functoriality says that 
    $c(F^* E) = F^*(c(E)) \in H^*_{dR}(M)$, (i.e. $c_i(F^* E) = F^*(c_i(E)) \in H^{2i}_{dR}(M)$ for all $i$).
    
    \item Normalisation:

    $c(L)=1+x$, for the tautological line bundle $L$ on $\complexnos P^k$, where $x$ is the Pioncar{\' e} dual of the hyperplane $\complexnos P^{k-1} \subseteq \complexnos P^k$. 

    \item Duality:

    If $E^*$ is the dual bundle of $E$ then $c_i(E^*)=(-1)^ic_i(E)$

\end{enumerate}

As an application of axiom 1, and the triviality axiom, we observe that if $\forall i\geq 1, c^i(E)\neq 0$ (and so $c(E)\neq 1$), then the bundle $E$ is not trivial. 

\subsubsection{Axiomatic definition of Chern classes}
It is also possible to define Chern classes axiomatically. Denote the set of (complex) vector bundles $E\to M$ of $\complexnos$-rank $k$, over a fixed smooth manifold $M$ as $\mathfrak{E}$. Define
\begin{align*}
    c: \mathfrak{E} & \to H^*_{dR}(M) \\
    E & \mapsto c(E) = c_0(E)+c_1(E)+\ldots +c_k(E)
\end{align*}
where $c_i(E)\in H^{2i}_{dR}(M)$, with the properties of triviality, naturality, Whitney sum, and normailsation (defined above). Then one can show that these axioms uniquely define a Chern class and the axiomatic definition is equivalent to our previous definition in terms of Chern-Weil theory. 

\subsubsection{Chern classes on induced bundles}
Suppose $E, E'$ are complex bundles with respective connections $\nabla_E, \nabla_{E'}$ and associated curvature forms $F_{\nabla_E}, F_{\nabla_{E'}}$ respectively.

If $B=E \oplus E'$, then recall the curvature form on $B$ is given by $F_\nabla = F_{\nabla_E} \oplus F_{\nabla_{E'}}$. 
Now since $\det((I_E + \frac{i}{2\pi}F_{\nabla_E})\oplus (I_{E'} + \frac{i}{2\pi}F_{\nabla_{E'}})) = \det(I_E + \frac{i}{2\pi}F_{\nabla_E})\cdot  \det(I_{E'} + \frac{i}{2\pi}F_{\nabla_{E'}})$, we obtain that the Chern form of $B$ is
$$c(B, \nabla) = c(E, \nabla_E)\cdot c(E', \nabla_{E'})$$
called the \defn{Whitney product formula}. This also translates to the Chern classes of $B$ (see properties of Chern classes section). 

Suppose $L$ is a line bundle, and $B = E\otimes L$. Then one can show the first Chern class is given by $$c_1(E\otimes L) = c_1(E) + \text{rank}(E)c_1(L)$$

If $\nabla^*$ is the associated connection on the dual bundle $E^*$ of $E$. Then recall that $F_{\nabla^*} = -(F_\nabla)^T$. Hence $\det(I + \frac{i}{2\pi}F_{\nabla_{E^*}}) = \det(I - \frac{i}{2\pi}F_{\nabla_E}^T)=\det(I - \frac{i}{2\pi}F_{\nabla_E})$. From this we get the Chern forms 
$$c_k(E^*, \nabla^*)=(-1)^k c_k(E, \nabla)$$

Consider the pull-back bundle $f^*E$ (for a smooth map $f:M\to N$ of manifolds, with a bundle $E$ over $N$). Then we know that $F_{f^*\nabla}=f^* F_\nabla$, from which we obtain
$$c_k(f^*E, f^*\nabla)= f^* c_k(E, \nabla)$$

\subsubsection{Chern characters}
We now use a different invarient homogeneous polynomial to define Chern characters. 

Define the invarient polynomials $\{Q_k\}$ given by 
$$tr(e^B)=Q_0(B)+Q_1(B)+Q_1(B)+\ldots $$
Then we define the $2k$-forms 
$$ch_k(E,\nabla):=Q_k \left( \frac{i}{2\pi}F_\nabla \right) \in \Omega^{2k}_\complexnos (M)$$.
The $k$'th \defn{Chern character} is then defined as the cohomology classes of these forms, namely
$$ch_k(E):= [ch_k(E, \nabla)]\in H^{2k}(M, \complexnos)$$

We note that $ch_0(E)=\text{rank}(E)$.

Finally we define the \defn{total Chern character} as 
$$ch(E):=ch_0(E)+ch_1(E)+ch_2(E)+\ldots$$

\subsubsection{Example: First Chern class of tangent bundle to $\complexnos P^1$}
Note that the tangent bundle $T(\complexnos P^1)$ is $\realnos$-linear isomophic to the real tangent bundle $T(S^2)$. Given coordinates $z$ on $\complexnos P^1$, recall the metric on $T(\complexnos P^1)$ is the Fubini-study metric given by
$$h(z):= h(\partial_z, \partial_z) := \frac{1}{(1+|z|^2)^2}$$
Note that if we view this in $T(S^2)$ with coordinates $w=\frac{1}{z}$ at infinity, then we get the same form 
$$h(\partial_w, \partial_w)=\frac{1}{(1+|w|^2)^2}$$

We now have that locally the connection matrix $A$ (so $\nabla=d+A$) is 
\begin{align*}
A(z) & =h(z)^{-1}\partial h(z) \\
& = (1+|z|^2)^2 \partial \left( \frac{1}{(1+|z|^2)^2} \right) \\
& = (1+|z|^2)^2 \frac{\partial}{\partial z} \left( \frac{1}{(1+z\bar{z})^2} \right)dz \\
& = (1+|z|^2)^2 \left( -2 (1+z\bar{z})^{-3} \bar{z} \right)dz \\
& = -\frac{2\bar{z}}{1+|z|^2} dz
\end{align*}

The associated curvature form $F_\nabla$ is then locally
\begin{align*}
\theta & = \bar{\partial} A \\
& = \frac{\partial}{\partial \bar{z}} \left( -\frac{2\bar{z}}{1+|z|^2} dz \right) d\bar{z} \\
& = 2 \frac{\partial}{\partial \bar{z}} \left(\frac{\bar{z}}{1+z\bar{z}} \right) dz\wedge d\bar{z} \\
& = 2 \left(\frac{1\cdot (1+z\bar{z})-\bar{z}z}{(1+z\bar{z})^2} \right) dz\wedge d\bar{z} \\
& = 2 \left(\frac{1}{(1+z\bar{z})^2} \right) dz\wedge d\bar{z} \\
& = \frac{2}{(1+|z|^2)^2} dz \wedge d\bar{z}
\end{align*}


Hence the first Chern form is given by 
$$c_1(E, \nabla) = \frac{i}{2\pi} \theta = \frac{i}{\pi(1+|z|^2)^2} dz \wedge d\bar{z} = \frac{2dx\wedge dy}{\pi(1+|z|^2)^2}$$

Since on $\complexnos P^1$ the top cohomology class is of degree 2, so the first Chern class $c_1(\complexnos P^1)\in H^2_{dR}(\complexnos P^1)$ is in the top cohomology class. Recall by the discussion of the cohomology on $\complexnos P^n$, that $H^2_{dR}(\complexnos P^1)\simeq \realnos$ as a vector space, and so is generated by a basis element $h$. Then $c_1(\complexnos P^n) = ah$ for some $a\in \realnos$ given by the integral of the first Chern form $a=\int_{\complexnos P^1} c_1(E, \nabla)$. Let us compute this number $a$.
\begin{align*}
    \int_{\complexnos P^1} c_1(E, \nabla) & =  \frac{2}{\pi} \int_{-\infty}^\infty \int_{-\infty}^\infty \frac{dx dy}{(1+||(x,y)||^2)^2} \\
    & = \frac{2}{\pi} \int_{0}^\infty \int_{0}^{2\pi} \frac{\rho}{(1+\rho^2)^2} d\rho d\theta \\
    & = 4 \int_{0}^\infty \frac{\rho}{(1+\rho^2)^2} d\rho  \\
    & = 2 \int_{1}^\infty \frac{du}{u^2}  \\
    & = 2
\end{align*}
Since this integral is non-zero, the closed form $c_1(E,\nabla)$ is not exact (by Stokes theorem). Hence the cohomology class $H^2_{dR}(\complexnos P^1)$ is non-zero. Hence $T(\complexnos P^1)$ is a non-trivial complex line bundle. 

\section*{References}
\begin{enumerate}
\item R.O.Wells Jr.; \textit{Differential Analysis on Complex Manifolds, 3rd ed.} Springer (2008)  
\item D.Huybrechts; \textit{Complex Geometry, an Introduction.} Springer (2005)
\item J.W.Milnor, J.D.Stasheff; \textit{Characteristic Classes.} Princeton University Press (1974)
\item M.Nakahara; \textit{Geometry, Topology and Physics.} IOP Publishing Ltd (1990)
\item J.W.Robbin, D.A.Salamon; \textit{Introduction to Differential Topology.} (2018) 
\item I.Madsen, J.Tornehave; \textit{From Calculus to Cohomology, De Rham Cohomology and characteristic classes.} Cambridge University Press (1997)
\item P.B.Gilkey, R.Ivanova, S.Nik{\u c}evi\'c; \textit{Characteristic classes.} Elsevier Ltd (2006)
\item L.W.Tu; \textit{Differential Geometry; Connections, Curvature and Characteristic Classes.} Springer (2017)
\item J.M.Lee; \textit{Introduction to Riemannian Manifolds, 2nd ed.} Springer (2018)
\item J.M.Lee; \textit{Introduction to Smooth Manifolds, 2nd ed.} Springer (2012)
\item M.Atiyah; \textit{The Geometry and Physics of Knots.} Cambridge University Press (1990)
\item M.H.Freedman, A.Kitaev, M.J.Larsen, Z.Wang; \textit{Topological Quantum Computing.} arxiv:quant-ph/0101025v2 (2008)
\end{enumerate}
\end{document}
