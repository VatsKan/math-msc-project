\documentclass[a4paper]{article}

\usepackage{amsthm, amsmath, latexsym, amssymb}
\usepackage{mathtools}
\usepackage{verbatim}
\usepackage[dvipsnames]{xcolor}

\theoremstyle{definition} \newtheorem*{definition}{Definition}
\theoremstyle{definition} \newtheorem*{definitions}{Definitions}
\theoremstyle{plain} \newtheorem{theorem}{Theorem}[section]
\theoremstyle{plain} \newtheorem{proposition}[theorem]{Proposition}
\theoremstyle{plain} \newtheorem{corollary}[theorem]{Corollary}
\theoremstyle{plain} \newtheorem{lemma}[theorem]{Lemma}
\theoremstyle{plain} \newtheorem{example}[theorem]{Example}

\newcommand{\checkCorrect}[1]{\textcolor{red}{#1}}
\newcommand{\understandBetter}[1]{\textcolor{orange}{#1}}
\newcommand{\question}[1]{\textcolor{orange}{#1}}
\newcommand{\explainFurther}[1]{\textcolor{blue}{#1}}
\newcommand{\finish}[1]{\textcolor{green}{#1}}

\newcommand{\defn}[1]{\textbf{#1}}
\newcommand{\realnos}{\mathbb{R}}
\newcommand{\complexnos}{\mathbb{C}}
\newcommand{\canonicaliso}{\cong}
\newcommand{\id}{\mathtt{id}}
\newcommand{\smoothCmaps}{C^\infty (U, \complexnos)}

\begin{document}
\title{Connections}
\author{Vatsal Kanoria}
\date{2023}
\maketitle

\tableofcontents

\section{Connections}
We will adopt the following notation throughout this section. Let $E\to M$ be a $\complexnos$-vector bundle over a smooth manifold of rank $r$.
Let $f=(e_1,\ldots, e_r)$ be a local frame of the bundle over some open neighbourhood $U\subseteq M$. (Hence $e_j\in \Gamma(U, E)$ are linearly independent at each $p\in U$).

\subsubsection{Local frames}
A smooth map $g:U\to GL_r(\complexnos)$ is called a \defn{change of frame mapping}, since $fg$ is another local frame on $U$ given by 
$$fg=(\sum_{k=1}^r g_{k1}e_k,\ldots, \sum_{k=1}^r g_{kr}e_k)$$
noting that $g_{kj}\in \smoothCmaps$. We may also express this in terms of matrix multiplication as $(fg)(p):=f(x)g(x)$. Note if we wrote $f$ as a column vector instead of a row vector then we would write $gf$ instead of $fg$.

One can show that given any two local frames $f, f'$ of $E$ over $U$, there always exists a change of frame mapping $g:U\to GL_r(\complexnos)$ such that $f'=fg$. 

\subsubsection{Local representation of sections}
Given a local frame $f=(e_1,\ldots, e_r)$ we may express a section $s\in \Gamma(U, E)$ locally as
$$s=\sum_{i=1}^r s^i(f)e_i$$
for some unique $s^i(f)\in \smoothCmaps$ ($i=1,\ldots, r$), and we define 
$$s(f):= \begin{pmatrix}s^1(f) \\ \vdots \\ s^r(f)\end{pmatrix}$$

Suppose $g:U\to GL_r(\complexnos)$ is a change of frame mapping, so that $f'=fg$ is another local frame. Then we obtain the following transformation law $s(f')=s(fg)=g^{-1}s(f)$, or in other words
$$gs(f')=s(f)$$
which follows from direct computation since $s^i(fg)=\sum_{j=1}^r g_{ij}^{-1} s^j(f)$.

\subsubsection{Hermitian Vector bundles}
On a complex vector space $V$, a \defn{Hermitian inner product} is a map $(\cdot, \cdot):V\times V\to \complexnos$ such that for all $u,v\in V$
\begin{align*}
& (u,v)=\overline{(v, u)} \\
& (\lambda u + v, w) = \lambda (u, w) + (v, w) \\
& (u, v)\geq 0 \\
& (u,v)=0, \forall v \implies u=0
\end{align*}
Or in other words $(\cdot, \cdot)$ is a conjugate-symmetric, sesquilinear, positive, non-degenerate map. 

For example on $V=\complexnos^n$, $(x, y)=\sum_i x_i\overline{y_i}$, $\forall x,y\in \complexnos^n$, is a Hermitian inner product. Note in this case we have that for a matrix $A\in M_n(\complexnos)$ we have $(Au, v)=(u, \overline{A}^Tv)$.

A \defn{Hermitian vector bundle} assigns a Hermitian inner product to each fibre $E_p$ of the vector bundle. More formally, for every open neighbourhood $U\subseteq M$, and for every pair of sections $s, \gamma \in \Gamma(U, E)$, \defn{a Hermitian metric} $h$ on $E$ is a map $h(s, \gamma):=\langle s, \gamma \rangle:U\to \complexnos$ such that $\langle s, \gamma \rangle(p)=\langle s(p), \gamma (p) \rangle$ is a Herimitian inner product on $E_p$, and $\langle s, \gamma \rangle$ is smooth. 

One may show that it is possible to construct a Hermitian metric on any complex vector bundle (the construction of which involves a partitions of unity). 

\subsubsection{Hermitian metric local representation}
Given a local frame $f=(e_1, \ldots, e_r)$ over $U$, define 
$$h(f)_{ij}=\langle e_i, e_j \rangle \ (:U\to \complexnos)$$
Whence $h(f)\in M_r(\smoothCmaps)$ is a positive definite, Hermitian, symmetric matrix, that represents $h$ locally with respect to the frame $f$. 

Recall given $s, \gamma\in \Gamma(U, E)$, we can write $s(f)=(s^1(f), \ldots, s^r(f))^T$, $\gamma(f)=(\gamma^1(f), \ldots, \gamma^r(f))^T$ locally with respect to the frame $f$. Then one may show that 
$$\langle s, \gamma \rangle = \overline{\gamma(f)^T}h(f)s(f)$$
(where the product here is matrix multipilicaiton).

Furthermore, if $g:U\to GL_r(\complexnos)$ is a change of frame mapping with $f'=fg$. Then we get the transformation law
$$h(f')=\overline{g^T}h(f)g$$
for the local representation of $h$ with respect to the frames $f, f'$.

\subsubsection{Differential forms with vector coefficients}
Define $\bigwedge^p_\complexnos(T^\ast M):=\bigwedge^p (T^\ast M)\otimes \complexnos$, i.e.\ the complexification of $\bigwedge^p_\realnos(T^\ast M)$. Note that we can express the elements of $\Omega^p_\complexnos(M)=\Gamma(\bigwedge^p_\complexnos(T^\ast M))$ as $\omega+i\eta \in \Omega^p_\complexnos(M)$ for $\omega, \eta \in \bigwedge^p_\realnos(T^\ast M)$, and $i$ the imaginary unit. 

We define the differential $p$-forms with coeffecients in $E$ as 
\begin{align*}
\Omega^p(M, E) & :=\Gamma(M, \bigwedge {}^p_\complexnos(T^\ast M) \otimes_\complexnos E) \\
& \simeq Hom(\bigwedge {}^p_\realnos (TM), E)
\end{align*}
\understandBetter{UNDERSTAND THE ISO PROPERLY!}
In other words the elements $\omega\in \Omega^p(M, E)$ at a point $x\in M$ are skew-symmetric multilinear forms 
$$\omega_x:\underbrace{T_xM\times T_xM\times \ldots T_xM}_{p \text{ times}} \to E_x$$
assigning a vector in the fibre $E_x$ to the $p$-tuples of tangent vectors at $x$. 

\subsubsection{Connections}

\section*{References}
\begin{enumerate}
\item Wells
\end{enumerate}
\end{document}
