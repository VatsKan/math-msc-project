\documentclass[a4paper]{article}

\usepackage{amsthm, amsmath, latexsym, amssymb, verbatim}
\usepackage[dvipsnames]{xcolor}

\theoremstyle{definition} \newtheorem*{definition}{Definition}
\theoremstyle{definition} \newtheorem*{definitions}{Definitions}
\theoremstyle{plain} \newtheorem{theorem}{Theorem}[section]
\theoremstyle{plain} \newtheorem{proposition}[theorem]{Proposition}
\theoremstyle{plain} \newtheorem{corollary}[theorem]{Corollary}
\theoremstyle{plain} \newtheorem{lemma}[theorem]{Lemma}
\theoremstyle{plain} \newtheorem{example}[theorem]{Example}

\newcommand{\done}[1]{\textcolor{ForestGreen}{#1}}
\newcommand{\finish}[1]{\textcolor{orange}{#1}}
\newcommand{\checkCorrect}[1]{\textcolor{red}{#1}}
\newcommand{\understandBetter}[1]{\textcolor{red}{#1}}


\begin{document}
\title{Cobordism, Chern Simons, Topological Quantum Computing}
\author{Vatsal Kanoria}
\date{August 2022}
\maketitle
\begin{abstract}
This is the abstract
\end{abstract}
\tableofcontents

\section{Introduction}

\finish{About the project}

Acknowledgements

\section{Manifolds}

\done{Define Smooth Manifold, differential structure/(maximal) atlas.
Define manifold with boundary}, 
compact manifold, closed manifold, oriented manifold.

Define complex manifold.

submanifolds submersions immersions? partition of unity? regular level set theorem?

\subsection{Examples}

Manifolds with boundaries

New manifolds from old. Subsets of $R^n$??
topology on $T^2$, $S^n$...smooth manifolds.

$RP^n, CP^n, HP^n, OP^n$

\section{Fibre Bundles}

\done{Define Fibre Bundle.}

\done{Define Bundle morphism.}

\subsection{Examples}
Mobius band (see sinha youtube + milnor ch2)

Double twisted mobius band (see sinha youtube)

Hopf fibrations. 
The torus $T^2 = S^1 \times S^1$ over $S^1$, with fibres $E_p \cong S^1$. Is a trivial bundle/trivial fibration. \understandBetter{(Hopf invarient is invarient of homotopy group).}

Hopf line over $CP^1$. 
Hopf line bundle. 
line bundle on $RP^1$ isomorphic to $S^1$. 

The 4 hopf fibrations from the division algebras R, C, H, O.
double cover of $S^1$ (figure 8). 
$\mathbb{Z}^2 = S^0 \to S^1 \xrightarrow{2:1} S^1$;; note last $S^1$ comes from $RP^1$, middle one is boundary of mobius strip;; twisted fibration, so non trivial.
$(S^1\to) S^3$ over $S^2$; ($S^2$ comes from $CP^1$. $S^3$ all lines in $C^2 - \{0\}$).
(gives homotopy groups of spheres. (homotopy group is equivalence class of loops in a space -- throw a loop (sphere) into space and can you catch something with it))

\section{Vector Bundles}

\done{Define Vector Bundle (local trivs).}

\done{Define Complex vector bundle??} [DO ALL FOLLOWING DEFS HOLD IN SAME WAY?]. (Do i need oriented, riemannian, hermitian vector bundle? -- salamon)

\checkCorrect{Explain that complex vector bundles can be on real manifolds.}

Define euclidean vect bundles (and riem manifolds)? (milnor pg21-23)

\done{Define transition maps between local trivs and Structure group??} ...[Do this for fibre bundles instead??]

\done{Define sections, module of sections over smooth functions. Define space of local sections $\Gamma (U, E)$ and local expression for this (see exposition notes).}

\done{Explain relationship between local trivialisations with local frames, and global trivs with non-vanishing sections. (milnor T2.2, or perhaps gilkey)}
[See later: G-bundles can have vanishing sections with corresponding global trivs --- true?? --check.]. 

Define Tangent Bundle. 
Explain briefly that tangent vectors are point derivations which form an (algebra?).
Explain given chart of M, $\frac{\partial}{\partial{x_i}} |_p$ for all $i$ is a basis of $T_p M$.

Go into derivatives/jacobians etc??

Define vector field (and notation for space of all vect fields)...Q: is there a name for sections of general vector bundle (like is it called a tensor field or something)?

Define Vector Bundle Morphism.

Identification between vect bundle morphims and module-homos of sections. (See Ivan notes).

\subsection{De Rham Cohomology}

Define Differential forms.

Define chain complex/homology/cohomology in more general setting??

Define de Rham differential/Exterior Derivative.

Define De Rham Cohomology.

\subsection{Integration??}

Define integration of forms??

Stokes theorem??

\subsection{Examples (of vect bundles)}

Zero section

Trivial line-bundle.

Trivial bundle.

Is tangent bundle always a trivial bundle? -- see milnor pg 7.

Normal bundle (milnor)

Canonical line bundle over real projective space (milnor). (is a non-trivial bundle).

New bundles from old. [see Hatcher]
e.g. tensor bundle -- see notes.
dual bundle (see ivan notes 21 june).
endomorphism bundle ($End(E)\to M$).
complexification/realification

Frame bundle

Perhaps grassmanians.

Tangent bundle over $S^n$. Write up transition functions etc.
(Explain global triv on $TS^2$ does not exist by hairy ball theorem.--- for which $n$ does it/(not) exist?).
\understandBetter{(Note: Sits inside $R^5$, invarient - curvature, integrating over invarient to tangent bundle gives euler number (quantum/topological invarient). leads to gauss-bonet theorem)}

$R^n/Z^n$ quotient spaces. 

Example of computing de rham cohom/integration??

\section{Principle G-Bundles}

Lie groups, lie algebras, lie alegbra of a lie group --> Appendix. (see Lee)

left/right G-actions on manifold. Orbits (and example of orbit with $M=R^2, G=SO(2)$. Orbit space. Stabiliser (example of stabiliser in $SO(2)$ example.)
Free G-action and examples. (see Schuller)

Explain diffeo of orbit space at point with the Lie group, given a free G-action. (see Schuller)

Define G-Bundle??? Q: is this different to principle G-bundle??

Define Principle G-Bundle. (see Schuller notes/youtube)

Define Principle Bundle Morphism. (see Schuller notes/youtube)

Define Trivial principle G-bundle. (Schuller notes)

Theorem that trivial principle G-bundle iff exists smooth section. (Proof -- appendix. See schuller).  (Did simon also mention this in meeting?--20 june pics)

(Define associated bundle?)

\subsection{Examples}
See also gilkey for some examples

$SO(2)$ on $R^2$

Frame bundle as a principle G bundle.

Some example of trival principle G-bundle?

$GL(2)$ on $LS^2$ non-trivial principle G-bundle (due to hairy ball theorem).

$U(n)$ with base $C^n$ ?

$SU(2)$ with base $C^2$ ? 

\understandBetter{The unit quaternions can be thought of as a choice of a group structure on the 3-sphere S3 that gives the group Spin(3), which is isomorphic to SU(2) and also to the universal cover of SO(3).} (wikipedia quaternion)

\section{Connections on Vector Bundles}

Define connection in 2 ways.

Explain various linear/module isomorphisms --> Appendix (linear algebra). (see Exposition notes and Ivan notes)
More generally, write the isos in diff geo/global setting. Bundle homo characterisation lemma (see Lee) etc. [Move this to vector bundle section???] 

Equivalence of connections. 
Appendix --> Show equivalence of 2-4 different defs of connections from the various refs.  (see Ivan notes 21 june and exposition )

Define affine/linear connection.

Explain $\bigtriangledown = d +A$

\subsection{Examples}

Directional derivative.

Show $\bigtriangledown^1 - \bigtriangledown^2$ is tensorial (i.e. comes from bundle homo even though $\bigtriangledown^1,  \bigtriangledown^2$ is not neccessarily). (See exposition notes). (See Salamon notes for why relevant).

Standard trivial connection on trival bundle. (For line bundle, and more generally for $MxR^k$). Various perspectives on this (i.e. using the two defs). Also obtain d+A formula in this setting (i.e. $\bigtriangledown = \omega + d$ (see exposition notes), also noting that diff of connections is tensorial from previous example).

Riemannian metric, Levi-Civita connection (example of affine connection). Christoffel symbols.

\section{Connections on Principle G-Bundles}

G-connections. (see Salamon)

Pullback connections? (see Salamon)

Gauge transformations, gauge group. Gauge group action on G-connections. (see Salamon)

\section{Equivalence of Connections}
Show equivalence between vector bundles and principle G-bundles (see pictures from Simon meeting 20 June)
and connections on these spaces??

\section{Curvature}

Define curvature. 

(Alt defs of curvature and show equivalent). (See  lecture notes. and also salamon notes).

Show curvature is module-trilinear [to do]. --> Appendix.

Explain $R^\bigtriangledown = dA + A\wedge A$. (see Salamon notes).

\subsection{Flat connections and more general De Rham Cohomology}

Define vector valued differential forms (see salamon notes). 
Show that $\Omega^1 (E, M) = \Gamma (T^*M\otimes E)$ (See exposition notes). 
(Rewrite connection def in terms of this).

(see Salamon notes):
Define vector valued exterior derivative. That it doesn't form a chain complex, but Bianchi identity holds. 

Aside: show how $d^\bigtriangledown$ relates to the usual exterior derivative $d$.

Define flat connection (i.e. when curvature vanishes).

Define cohomology for when have a flat connection.

TODO: Understand correspondance $\rho^\bigtriangledown -> \bigtriangledown$ in salamon pg 235. (Looks like some kind of Rep theory I think).

\subsection{Examples}

Horizontal sub-bundles of tangent bundle of total space of vector bundle. (for intuition on flat connection). (see Salamon)



\section{Torsion??}
[Is torsion needed? Do torsion before curvature?]

Define torsion on affine connection. 

Show torsion is module-bilinear/tensorial. (see exposition notes) --> Appendix

\section{Characteristic classes}
Cohomology --> appendix. (what do i need from cohomology theory?)

Euler characteristic. Euler characteristic zero means parralelizable (i.e. exists trivial bundle --check) (milnor).

Chern classes

Review characters from repr theory? Explain how this is pointwise version of characteristic classes (if true)?

Secondary character class?

\section{Bordism and Cobordism}

\section{Chern Simons}


\section{Topological Quantum Computing}
SU(2)-Chern Simons. Yang-Mills theory.

Word lines of non-abelian anyons.

Define TQFT. [Understand associativity axiom in Atiyah related to Cobordism.]

\section{Appendix: Functors and Categories}

\section*{References}
\begin{enumerate}
\item This is a reference
\end{enumerate}
\end{document}
